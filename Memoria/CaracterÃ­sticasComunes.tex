\section{Características de las heurísticas}
	En este apartado, se describirán las características comunes empleadas para resolver
	el problema.
	
	\subsection{Representación de la solución}
		Representaremos la solución como un vector de booleanos de longitud $n$, con $n$
		el número de características del problema dado, en el cuál se especifica si la
		característica $f_i$ está en la solución o no.
	
	\subsection{Función objetivo y función evaluación}
		Como función objetivo lógica para nuestro problema, se tomaría la función:
		$$tasa_{aciertos} = 100 \cdot \frac{nº\ instancias\ bien\ clasificadas}{nº\ total\ de\ instancias}$$
		
		Sin embargo, como el número total de instancias es una constante y para la máquina
		es más sencillo trabajar con valores enteros que con valores decimales, optaremos
		por maximizar la función que nos calcula el número de aciertos y los transformaremos
		en la tasa de aciertos cuando queramos evaluar el buen funcionamiento de la
		clasificación realizada sobre el conjunto de evaluación.
		
		Un pequeño esquema del funcionamiento de nuestra función de evaluación sería:

		\begin{algorithm}[H]
			\begin{algorithmic}[1]

				\REQUIRE \ \\
		        	\texttt{c\_sel}, vector de características seleccionadas\\
		        	\texttt{inst}, instancias sobre las que evaluar\\ \

		     	\STATE{\texttt{aciertos} = 0}\\
		     	\FORALL{\texttt{instancia} $\in$ \texttt{inst}}
			  		\STATE{Genera \texttt{K-NN} de \texttt{instancia[i]}}
			  		\IF{\texttt{Hay clase predominante}}
						\STATE{\text{clase\ esperada} = Clase predominante de \texttt{K-NN} de \texttt{instancia[i]}}
					\ELSE
						\STATE{\text{clase\ esperada} = Clase de \texttt{1-NN} de \texttt{instancia[i]}}
					\ENDIF
			  		
					\IF{\texttt{clase\ esperada} == Clase de \texttt{instancia[i]}}
						\STATE{\texttt{aciertos++}}
					\ENDIF
				\ENDFOR
	  
				\RETURN{aciertos}			
			\end{algorithmic}
   \caption{Función de evaluación}
   \label{Evaluate}
		\end{algorithm}
		
		Este algoritmo tiene algunas variantes, como que si no le pasas instancia, toma la que
		usó para realizar el entrenamiento, si no le pasas vector de características seleccionadas,
		usa la que tiene almacenada y que si le pasas un entero $i$, usa el vector de características
		resultante de alternar el valor de $f_i$ de la solución almacenada.
		
		En nuestro caso, tomamos $K=3$ para realizar el KNN. Además, para el cálculo del KNN
		utilizamos el \textit{framework} que ya estaba implementado en \cite{KNN}.
	
	\subsection{Operador de vecinos}
		Para la generación de una solución vecina, basta con alterar la pertenencia de una
		característica al conjunto de características seleccionadas, y alterar el contador
		de las mismas ya que me parecía más eficiente de este modo llevar la cuenta de cuántas
		fueron escogidas. Una aproximación de sería la siguiente:
		
		\begin{algorithm}[H]
			\begin{algorithmic}[1]
				\REQUIRE \ \\
		        	\texttt{index}, índice a cambiar\\ \

		     	\STATE{\texttt{car[index]} = \texttt{!car[index]}}\\
		  		\IF{\texttt{car[index]} == \texttt{true}}
					\STATE{Incrementamos número de características seleccionadas}
				\ELSE
					\STATE{Decrementamos número de características seleccionadas}
				\ENDIF
			\end{algorithmic}
		\caption{Generación de solución vecina}
		\label{Flip}
		\end{algorithm}
	
		Y para el caso de la generación de un vecino aleatorio, bastaría con generarlo aleatoriamente,
		invocar a la función anterior y devolver el número generado (por si fuese necesario
		para algo en el ámbito en el que se invoca la función):
	
		\begin{algorithm}[H]
			\begin{algorithmic}[1]
				\REQUIRE \ \\
					 \

		     	\STATE{\texttt{index} = Entero Aleatorio}\\
		  		\STATE{\texttt{Flip(index)}}\\
		  		
				\RETURN{index}
			\end{algorithmic}
		\caption{Generación aleatoria de solución vecina}
		\label{Neighbour}
		\end{algorithm}
	
	\subsection{\textit{5x2-Cross Validation}}
		Para realizar la comprobación del buen funcionamiento de nuestras heurísticas, realizaremos
		una serie de particiones del conjunto de instancias que nos pasan como parámetro, algunas
		las usaremos como entrenamiento y otras como validación.
		
		En concreto, utilizaremos la conocida como \textit{5x2-Cross Validation}, que consiste en
		dividir el conjunto en dos mitades, tomar una como entrenamiento y la otra como validación,
		y luego invertir la que se usó como entrenamiento y la que se usó como validación. Este
		proceso se realiza 5 veces:
		
		\begin{algorithm}[H]
			\begin{algorithmic}[1]
				\REQUIRE \ \\
					 \
					 
				\STATE{\texttt{i = 0}}
		     	\FOR{\texttt{i} $<$ \texttt{5}}
			  		\STATE{\texttt{part1} = Genera partición}
			  		\STATE{\texttt{part2} = Complementario(\texttt{part1})}

					\STATE{\texttt{Heurística.entrenar(part1)}}
					\STATE{\texttt{evaluaciones[2*i]} = \texttt{Heurística.evaluar(part2)}}
									  		
					\STATE{\texttt{Heurística.entrenar(part2)}}
					\STATE{\texttt{evaluaciones[2*i+1]} = \texttt{Heurística.evaluar(part1)}}
				\ENDFOR
				
				\RETURN{evaluaciones}
			\end{algorithmic}
		\caption{\textit{5x2-Cross Validation}}
		\label{Cross-Validation}
		\end{algorithm}			
		
	\subsection{Variables globales}
		Existen las siguientes variables globales en el problema:
		\begin{itemize}
			\item \textbf{Semillas aleatorias:} Éstas dan cabida a la generación de los números
			aleatorios necesarios para ciertos procesos como la generación de la solución inicial
			o la generación de soluciones vecinas. Disponemos de 11 semillas aleatorias en esta
			implementación:
				\begin{itemize}
					\item \textbf{Principal:} Sólo existe una semilla aleatoria principal, a la
					que le dí el valor $103$.
					\item \textbf{Generadas:} Estas semillas están generadas de forma aleatoria
					a partir de la semilla principal. Cada una de ellas se pasa los algoritmos
					que la requieren\footnote{Todos menos \textit{SFS}.} a la hora de la
					inicialización del objeto. Existen 10 para pasar una en cada prueba de las
					heurísticas.
				\end{itemize}
			\item \textbf{Número máximo de evaluaciones:} Es un valor que nos permite limitar
			las iteraciones de los algoritmos(ya que, de otro modo, podrían no terminar o tardar
			mucho más tiempo)\footnote{Usada por todas excepto la \textit{SFS} porque necesitaría
			una cantidad tan desproporcionada de características en la instancia pasada que
			dicha variable carecería de sentido usarla.}. Le hemos dado el valor de 15000.
		\end{itemize}
	
	\subsection{Entrada y salida}
		Para recoger los datos a través de la entrada, he implementado una clase extra basándome
		en \cite{ARFFReader}.
		Para generar los archivos \textit{.csv}, he utilizado una clase implementada por un
		compañero y algo adaptada a mi estilo de programación.
