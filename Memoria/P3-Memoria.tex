%%
% Plantilla de Memoria
% Modificación de una plantilla de Latex de Nicolas Diaz para adaptarla 
% al castellano y a las necesidades de escribir informática y matemáticas.
%
% Editada por: Mario Román
%
% License:
% CC BY-NC-SA 3.0 (http://creativecommons.org/licenses/by-nc-sa/3.0/)
%%

%%%%%%%%%%%%%%%%%%%%%
% Thin Sectioned Essay
% LaTeX Template
% Version 1.0 (3/8/13)
%
% This template has been downloaded from:
% http://www.LaTeXTemplates.com
%
% Original Author:
% Nicolas Diaz (nsdiaz@uc.cl) with extensive modifications by:
% Vel (vel@latextemplates.com)
%
% License:
% CC BY-NC-SA 3.0 (http://creativecommons.org/licenses/by-nc-sa/3.0/)
%
%%%%%%%%%%%%%%%%%%%%%

%----------------------------------------------------------------------------------------
%	PAQUETES Y CONFIGURACIÓN DEL DOCUMENTO
%----------------------------------------------------------------------------------------

%% Configuración del papel.
% microtype: Tipografía.
% mathpazo: Usa la fuente Palatino.
\documentclass[a4paper, 11pt]{article}
\usepackage[protrusion=true,expansion=true]{microtype}
\usepackage{mathpazo}

% Indentación de párrafos para Palatino
\setlength{\parindent}{0pt}
  \parskip=8pt
\linespread{1.05} % Change line spacing here, Palatino benefits from a slight increase by default


%% Castellano.
% noquoting: Permite uso de comillas no españolas.
% lcroman: Permite la enumeración con numerales romanos en minúscula.
% fontenc: Usa la fuente completa para que pueda copiarse correctamente del pdf.
\usepackage[spanish,es-noquoting,es-lcroman]{babel}
\usepackage[utf8]{inputenc}
\usepackage[T1]{fontenc}
\selectlanguage{spanish}


%% Gráficos
\usepackage{graphicx} % Required for including pictures
\usepackage{wrapfig} % Allows in-line images
\usepackage[usenames,dvipsnames]{color} % Coloring code

% % Enlaces
\usepackage[hidelinks]{hyperref}

%% Matemáticas
\usepackage{amsmath}

% Para algoritmos
\usepackage{algorithm}
\usepackage{algorithmic}
\usepackage{amsthm}
\floatname{algorithm}{Algoritmo}
\renewcommand{\listalgorithmname}{Lista de algoritmos}
\renewcommand{\algorithmicrequire}{\textbf{Entrada:}}
\renewcommand{\algorithmicensure}{\textbf{Salida:}}
\renewcommand{\algorithmicend}{\textbf{fin}}
\renewcommand{\algorithmicif}{\textbf{si}}
\renewcommand{\algorithmicthen}{\textbf{entonces}}
\renewcommand{\algorithmicelse}{\textbf{en otro caso}}
\renewcommand{\algorithmicelsif}{\algorithmicelse,\ \algorithmicif}
\renewcommand{\algorithmicendif}{\algorithmicend\ \algorithmicif}
\renewcommand{\algorithmicfor}{\textbf{para }}
\renewcommand{\algorithmicforall}{\textbf{para cada}}
\renewcommand{\algorithmicdo}{\textbf{}}
\renewcommand{\algorithmicendfor}{\algorithmicend\ \algorithmicfor}
\renewcommand{\algorithmicwhile}{\textbf{mientras}}
\renewcommand{\algorithmicendwhile}{\algorithmicend\ \algorithmicwhile}
\renewcommand{\algorithmicloop}{\textbf{repetir}}
\renewcommand{\algorithmicendloop}{\algorithmicend\ \algorithmicloop}
\renewcommand{\algorithmicrepeat}{\textbf{repetir}}
\renewcommand{\algorithmicuntil}{\textbf{hasta que}}
\renewcommand{\algorithmicprint}{\textbf{imprimir}} 
\renewcommand{\algorithmicreturn}{\textbf{devolver}} 
\renewcommand{\algorithmictrue}{\textbf{true }} 
\renewcommand{\algorithmicfalse}{\textbf{false }} 
\renewcommand{\algorithmicand}{\textbf{y}}
\renewcommand{\algorithmicor}{\textbf{o}}


%% Bibliografía
\makeatletter
\renewcommand\@biblabel[1]{\textbf{#1.}} % Change the square brackets for each bibliography item from '[1]' to '1.'
\renewcommand{\@listI}{\itemsep=0pt} % Reduce the space between items in the itemize and enumerate environments and the bibliography


%----------------------------------------------------------------------------------------
%	TÍTULO
%----------------------------------------------------------------------------------------
% Configuraciones para el título.
% El título no debe editarse aquí.
\renewcommand{\maketitle}{
  \begin{flushright} % Center align
  {\LARGE\@title} % Increase the font size of the title
  
  \vspace{50pt} % Some vertical space between the title and author name
  
  {\large\@author} % Author name
  \\\@date % Date
  \vspace{40pt} % Some vertical space between the author block and abstract
  \end{flushright}
}

% Título
\title{\textbf{Metaheurísticas: Selección de Características}\\ % Title
P-3: Algoritmos Genéticos} % Subtitle

\author{\textsc{Óscar Bermúdez Garrido\\
\href{http://www.github.com/oxcar103}{@oxcar103}} % Author
\\{\textit{Universidad de Granada}}} % Institution

\date{\today} % Date


%----------------------------------------------------------------------------------------
%	DOCUMENTO
%----------------------------------------------------------------------------------------

\begin{document}

\maketitle % Print the title section

% Resumen (Descomentar para usarlo)
\renewcommand{\abstractname}{Resumen} % Uncomment to change the name of the abstract to something else
%\begin{abstract}
% Resumen aquí
%\end{abstract}

% Palabras clave
%\hspace*{3,6mm}\textit{Keywords:} lorem , ipsum , dolor , sit amet , lectus % Keywords
%\vspace{30pt} % Some vertical space between the abstract and first section


% Índice
{\parskip=2pt
  \tableofcontents
}
\pagebreak

%% Inicio del documento
	\input{Introducción.tex}
	\input{CaracterísticasComunes.tex}
	
	\section{Heurísticas implementadas}\footnote{En la implementación de las heurísticas desarrolladas
	en esta práctica, se utilizan la búsqueda de soluciones por la heurística \textbf{SFS} para comparar
	los resultados obtenidos y las optimizaciones de la \textbf{LS} para mejorar los resultados obtenidos
	por estas nuevas heurísticas multiarranque. Dado que ya se explicaron en \textbf{P1-Memoria}, he
	considerado ahorrar su explicación en este documento para permitir una mayor limpieza del mismo.}
	
		\subsection{Algoritmos Genéticos (\textbf{GA})}
			A grandes rasgos, podemos realizar una descripción general de los Algoritmos Genéticos o
			\textit{Genetic Algorithm}(\textit{GA}) que consiste en:
			
			\begin{itemize}
				\item Generar una población inicial
				\item Evolucionar dicha población, o equivalentemente se puede expresar mediante los
				siguientes operadores:
					\begin{itemize}
						\item Selección: Se eligen parejas de la población para reproducirse.
						\item Cruce: Se recombinan las parejas seleccionadas con una probabilidad de
						cruce característica del problema haciendo tomar a las soluciones sucesoras
						características de ambos padres.
						\item Mutación: Se modifican aleatoriamente los genes de la nueva generación
						con una probabilidad de mutación, normalmente, bastante reducida.
						\item Reemplazo: Se procede a incluir la nueva generación en la población
						que actualmente tenemos.
					\end{itemize}
				\item Tras el paso de algunas generaciones, nos quedamos con el mejor individuo que
				haya existido.
			\end{itemize}
			
			Este proceso se puede reflejar en el siguiente algoritmo\footnote{Cabría destacar que,
			realmente, \texttt{Padres\_Seleccionados} no se devuelve por el Operador de Selección, si
			no que se trata como una variable global que se ve modificada por cierto número de factores}
			donde la población para nuestro modelo es de 30 individuos:
			
			\begin{algorithm}[H]
				\begin{algorithmic}[1]
				\REQUIRE \ \\
						 \
			 
				\STATE{i = 0}
				\STATE{\texttt{Índice\_Mejor}}
				\FOR{i < \texttt{Población}}
					\STATE{\texttt{Padres}.add(RandomSolution())}
					\IF{\texttt{Padres[Índice\_Mejor]} < \texttt{Padres[i]}}
						\STATE{\texttt{Índice\_Mejor} = i}
					\ENDIF
				\ENDFOR
				
				\WHILE{\texttt{Evaluaciones} < 15000}
					\STATE{Limpiamos \texttt{Hijos}}
					\STATE{\texttt{Padres\_Seleccionados} = Selection()}
					\STATE{i = 0}
					\FOR{i < $\displaystyle \frac{\texttt{Padres\_Seleccionados}.size()}{2}$}
						\STATE{Crossover(2 $\cdot$ i, 2 $\cdot$ i+1)}
					\ENDFOR
					\STATE{Mutation()}
					\STATE{Inheritance()}
				\ENDWHILE
				
				\STATE{\texttt{Solución\_actual = Padres[Índice\_Mejor]}}
				\end{algorithmic}
			\caption{Algoritmos Genéticos - Entrenamiento(\textit{Train})}
			\label{GA-Train}
			\end{algorithm}
			
			Para una selección óptima de los individuos de la población que pasarán a cruzarse que nos
			permita una rápida convergencia de la población a una buena solución del problema, los
			elegiremos mediante el Torneo Binario consitente en:
			
			\begin{algorithm}[H]
				\begin{algorithmic}[1]
				\REQUIRE \ \\
						 \
					
				\STATE{\texttt{Candidato\_1} = Genera valor aleatorio entre 0 y \texttt{Población}-1}
				\STATE{\texttt{Candidato\_2} = Genera valor aleatorio entre 0 y \texttt{Población}-1}
				
				\WHILE{\texttt{Candidato\_1} == \texttt{Candidato\_2}}
					\STATE{\texttt{Candidato\_2} = Genera valor aleatorio entre 0 y \texttt{Población}-1}
				\ENDWHILE
				\IF{Evaluate(\texttt{Padres[Candidato\_1]}) > Evaluate(\texttt{Padres[Candidato\_2]})}
					\RETURN{\texttt{Candidato\_1}}
				\ELSE
					\RETURN{\texttt{Candidato\_2}}
				\ENDIF
				\STATE{\texttt{Solución\_actual = Padres[Índice\_Mejor]}}
				\end{algorithmic}
			\caption{Algoritmos Genéticos - Torneo Binario(\textit{BinaryTournament})}
			\label{GA-BT}
			\end{algorithm}
			
			Aunque, \textit{a priori}, los 4 operadores necesarios para evolucionar la población son
			esencialmente iguales con pequeños detalles a lo largo del proceso, el más fácil de abstraer
			para los Algoritmos Genéticos Generacional con Elitismo o \textit{Generational Genetic
			Algorithm}(\textit{GGA}) y Estacionario o \textit{Steady-State}(\textit{SSGA}) resulta ser
			el Operador de Cruce que quedaría como:
			
			\begin{algorithm}[H]
				\begin{algorithmic}[1]
				\REQUIRE \ \\
						\texttt{Padre\_1}, un padre \\
						\texttt{Padre\_2}, el otro \\ \
						
				\STATE{\texttt{Primero} = Genera valor aleatorio entre 0 y \texttt{Número\_Características}-1}
				\STATE{\texttt{Segundo} = Genera valor aleatorio entre 0 y \texttt{Número\_Características}-1}
				
				\WHILE{\texttt{Segundo} == \texttt{Primero}}
					\STATE{\texttt{Segundo} = Genera valor aleatorio entre 0 y \texttt{Número\_Características}-1}
				\ENDWHILE
				
				\IF{\texttt{Primero} > \texttt{Segundo}}
					\STATE{swap(\texttt{Primero, Segundo})}
				\ENDIF
				
				\STATE{i = 0}
				\FOR{i < \texttt{Número\_Características}}
					\IF{i < \texttt{Primero}}
						\STATE{\texttt{Hijo\_1}[i] = \texttt{Padre\_1}[i]}
						\STATE{\texttt{Hijo\_2}[i] = \texttt{Padre\_2}[i]}
					\ELSIF{i < \texttt{Segundo}}
						\STATE{\texttt{Hijo\_1}[i] = \texttt{Padre\_2}[i]}
						\STATE{\texttt{Hijo\_2}[i] = \texttt{Padre\_1}[i]}
					\ELSE
						\STATE{\texttt{Hijo\_1}[i] = \texttt{Padre\_1}[i]}
						\STATE{\texttt{Hijo\_2}[i] = \texttt{Padre\_2}[i]}
					\ENDIF
				\ENDFOR
				
				\STATE{\texttt{Hijos}.add{\texttt{Hijo\_1}}}
				\STATE{\texttt{Hijos}.add{\texttt{Hijo\_2}}}
				\end{algorithmic}
			\caption{Algoritmos Genéticos - Cruce(\textit{Crossover})}
			\label{GA-Cross}
			\end{algorithm}
			
		\subsection{Algoritmo Genético Gerenacional (\textbf{GGA})}
			Como ya hemos visto en el apartado anterior, los algoritmos genéticos se caracterizan por:
			
			\begin{itemize}
				\item Probabilidad de cruce: que en este caso particular tomará el valor de 0.7
				\item Probabilidad de mutación: que en esta ocasión será de 0.001
				\item Operadores de Evolución:
				\begin{itemize}
					\item Selección
					\item Cruce\footnote{Éste ya fue explicado en el apartado anterior.}
					\item Mutación
					\item Reemplazo
				\end{itemize}
			\end{itemize}
			
			Antes de explicar el Operador de Selección, cabe resaltar que el proceso usual para realizar
			dicha implementación consiste en dividir la población en parejas y comprobar para cada una
			de ellas si, aleatoriamente, cruzarán o no. Claramente, este proceso es muy costoso, por lo
			que se ha optado por la selección únicamente de las parejas que cruzarán sí o sí, lo que
			requiere previamente del cálculo de la esperanza matemática de individuos de la población
			que mutarán y la almacenaremos en una variable \textit{Número\_de\_Cruces} con expresión
			igual a $\displaystyle \frac{\textit{Probabilidad\_de\_Cruce} \cdot \textit{Población}}{2}$.
			
			Análogamente al caso de la probabilidad de cruce, y con mayor motivo pues la probabilidad
			de que ocurra este hecho se produce con menor frecuencia que el anterior, se calcula la
			cantidad de genes que mutarán como la esperanza matemática de dicho suceso, es decir
			$\textit{Número\_de\_Mutaciones} = \textit{Probabilidad\_de\_Mutación} \cdot \textit{Población}
			\cdot \textit{Número\_de\_Características}$
			
			Llegados a este punto, estamos preparados para representar el Operador de Selección de
			esta variante de los \textit{AG} como:
			
			\begin{algorithm}[H]
				\begin{algorithmic}[1]
				\REQUIRE \ \\
						 \
						
				\WHILE{\texttt{Padres\_Seleccionados}.size() < \texttt{Número\_de\_Cruces}}
					\STATE{\texttt{Candidato} = BinaryTournament}
					\IF{\texttt{Candidato} $\notin$ \texttt{Padres\_Seleccionados}}
						\STATE{\texttt{Padres\_Seleccionados}.add(\texttt{Candidato})}
					\ENDIF
				\ENDWHILE
				
				\RETURN{\texttt{Padres\_Seleccionados}}
				\end{algorithmic}
			\caption{AG Generacional - Selección(\textit{Selection})}
			\label{GGA-Selec}
			\end{algorithm}
			
			En el caso de \textit{GGA}, la mutación se realiza tanto a los nuevos miembros de la comunidad
			como a los miembros de la anterior generación que quedaron sin modificar. Esto se podría
			interpretar como que los padres que no se cruzaron han generado en la nueva generación
			una serie de individuos sucesores idénticos a los predecesores.
			
			De igual manera, obtenemos resultado de la variable \textit{Número\_de\_Mutaciones} recién
			creada mediante el Operador de Mutación que se describe:
			
			\begin{algorithm}[H]
				\begin{algorithmic}[1]
				\REQUIRE \ \\
						 \
						
				\STATE{i = 0}
				\FOR{i < \texttt{Población}}
					\IF{i $\notin$ \texttt{Padres\_Seleccionados}}
						\STATE{\texttt{Hijos}.add(\texttt{Padres}[i])}
					\ENDIF
				\ENDFOR
				\WHILE{\texttt{Genes\_Mutados}.size() < \texttt{Número\_de\_Mutaciones}}
					\STATE{\texttt{Gen} = Genera valor aleatorio entre 0 y (\texttt{Población} $\times$
							\texttt{Número\_de\_Características})-1}
					\STATE{\texttt{Solución} = $\displaystyle \frac{\texttt{Gen}}{\texttt{Número\_de\_Características}}$}
					\STATE{\texttt{Característica} = \texttt{Gen} $\mod{\texttt{Número\_de\_Características}}$}
					
					\IF{\texttt{Gen} $\notin$ \texttt{Genes\_Mutados}}
						\STATE{\texttt{Genes\_Mutados}.add(\texttt{Gen})}
						
						\IF{\texttt{Solución} $\notin$ \texttt{Soluciones\_Mutadas}}
							\STATE{\texttt{Soluciones\_Mutadas}.add(\texttt{Solución})}
						\ENDIF
						
						\STATE{\texttt{Solución}.Flip(\texttt{Gen})}
					\ENDIF
				\ENDWHILE
				\end{algorithmic}
			\caption{AG Generacional - Mutación(\textit{Mutation})}
			\label{GGA-Mut}
			\end{algorithm}
			
			Finalmente, queda el paso de mezclar la antigua generación con la nueva, esto se realiza
			mediante la inserción del mejor candidato o élite de la antigua generación en la nueva
			generación sustituyendo a la peor solución de ésta.
			
			En el caso de \textit{GGA}, el proceso es simple, la nueva generación sustituye por completo
			a la antigua, y posteriormente, se realiza una búsqueda para obtener la mejor solución o
			élite de la nueva:
			
			\begin{algorithm}[H]
				\begin{algorithmic}[1]
				\REQUIRE \ \\
						 \
						
				\STATE{\texttt{Índice\_Peor} = 0}
				\STATE{i = 0}
				\FOR{i < \texttt{Población}}
					\IF{Evaluate(\texttt{Hijos[i]}) < Evaluate(\texttt{Hijos[Índice\_Peor]})}
						\STATE{\texttt{Índice\_Peor} = i}
					\ENDIF
				\ENDFOR
				
				\STATE{\texttt{Hijos[Índice\_Peor]} = \texttt{Padres[Índice\_Mejor]}}
				\STATE{\texttt{Índice\_Mejor} = \texttt{Índice\_Peor}}
				\STATE{\texttt{Padres} = \texttt{Hijos}}
				
				\STATE{i = 0}
				\FOR{i < \texttt{Población}}
					\IF{Evaluate(\texttt{Padres[i]}) > Evaluate(\texttt{Padres[Índice\_Mejor]})}
						\STATE{\texttt{Índice\_Mejor} = i}
					\ENDIF
				\ENDFOR
				\end{algorithmic}
			\caption{AG Generacional - Reemplazo(\textit{Inheritance})}
			\label{GGA-Inher}
			\end{algorithm}
			
		\subsection{Algoritmo Genético Estacionario (\textbf{SSGA})}
			Para el caso del Algoritmo Genético Estacionario o \textit{Steady-State}(\textit{SSGA}),
			tomaremos la misma probabilidad de mutación que para \textit{GGA}, es decir, 0.001 y un
			valor de 1 para la probabilidad de cruce, estableciendo de este modo que todas las parejas
			cruzarán, además, para este \textit{AG} necesitamos indicar el número de hijos que intentará
			entrar a formar parte de la población, le asignaremos el valor 2 aunque los algoritmos
			estarán diseñados para funcionar con cualquier valor\footnote{De hecho, corre por cuenta
			del programador asignarle un valor par ya que de tomar un valor impar el Operador de Cruce
			accedería a zonas de memoria no asignada y en ningún lugar se comprueba.}
			
			El Operador de Selección de los \textit{SSGA} se expresaría de la forma:
			
			\begin{algorithm}[H]
				\begin{algorithmic}[1]
				\REQUIRE \ \\
						 \
				
				\WHILE{\texttt{Padres\_Seleccionados}.size() < \texttt{Número\_de\_Hijos}}
					\STATE{\texttt{Candidato} = BinaryTournament()}
					\IF{\texttt{Candidato} $\notin$ \texttt{Padres\_Seleccionados}}
						\STATE{\texttt{Padres\_Seleccionados}.add(\texttt{Candidato})}
					\ENDIF
				\ENDWHILE
				
				\RETURN{\texttt{Padres\_Seleccionados}}
				\end{algorithmic}
			\caption{AG Estacionario - Selección(\textit{Selection})}
			\label{SSGA-Selec}
			\end{algorithm}
			
			En esta ocasión, a diferencia de \textit{GGA}, para cada característica tomaremos números
			aleatorios para establecer si se produce la mutación de la misma que, aunque es bastante más
			ineficiente, la forma correcta de hacerlo.
			
			Y como resultado, el Operador de Mutación quedaría como:
			
			\begin{algorithm}[H]
				\begin{algorithmic}[1]
				\REQUIRE \ \\
						 \
						
				\STATE{i = 0}
				\FOR{i < \texttt{Hijos}.size()}
					\STATE{j = 0}
					\FOR{j < \texttt{Número\_de\_Características}}
						\STATE{\texttt{rnd} = Genera valor aleatorio entre 0 y 1}
						\IF{rnd < \texttt{Probabilidad\_Mutación}}
							\STATE{i.Flip(j)}
						\ENDIF
					\ENDFOR
				\ENDFOR
				\end{algorithmic}
			\caption{AG Estacionario - Mutación(\textit{Mutation})}
			\label{SSGA-Mut}
			\end{algorithm}
			
			En este caso, se intenta integrar a los hijos a la población uno a uno, esto es, intentamos
			que entren si y solo si mejoran alguna solución ya existente en la generación anterior de
			la población. Una forma de realizarlo sería:
			
			\begin{itemize}
				\item Buscar la peor solución de la anterior generación
				\item Recorrer los hijos hasta que alguno lo mejore
				\item Recordar el índice de este hijo
				\item Buscar la siguiente peor solución
				\item Recorrer los hijos a partir del siguiente al índice que recordamos. Esto se debe
				a que si no superaron la primera solución peor encontrada, no van a superar a la segunda.
				\item Repetir el proceso hasta que no queden nuevos hijos por recorrer(en este caso,
				serían sólo 2 pero el algoritmo está planteado para el caso de que sean cualquier número).
			\end{itemize}
			
			Esta idea queda perfectamente reflejada en el siguiente algoritmo:
			
			\begin{algorithm}[H]
				\begin{algorithmic}[1]
				\REQUIRE \ \\
						 \
						
				\STATE{\texttt{Índice\_Hijo} = 0}
				\WHILE{\texttt{Índice\_Hijo} < \texttt{Hijos}.size()}
					\STATE{\texttt{Índice\_Peor} = 0}
					\STATE{\texttt{Reemplazado\_Peor} = \FALSE}
					\STATE{i = 0}
					\FOR{i < \texttt{Padres}.size()}
						\IF{Evaluate(\texttt{Padres[i]}) < Evaluate(\texttt{Padres[Índice\_Peor]})}
							\STATE{\texttt{Índice\_Peor} = i}
						\ENDIF
					\ENDFOR
					
					\WHILE{\texttt{Índice\_Hijo} < \texttt{Hijos}.size() \AND \NOT \texttt{Reemplazado\_Peor}}
						\IF{Evaluate({Índice\_Hijo}) > Evaluate(\texttt{Padres(Índice\_Peor)})}
							\STATE{\texttt{Padres[Índice\_Peor]} = \texttt{Hijos[Índice\_Hijo]}}
							\STATE{\texttt{Reemplazado\_Peor} = \TRUE}
							\IF{Evaluate(\texttt{Padres[Índice\_Peor]}) > Evaluate(\texttt{Padres[Índice\_Mejor]})}
								\STATE{\texttt{Índice\_Mejor} = \texttt{Índice\_Peor}}
							\ENDIF
						\ENDIF
						
						\STATE{\texttt{Índice\_Hijo}}
					\ENDWHILE
				\ENDWHILE
				\end{algorithmic}
			\caption{AG Estacionario - Reemplazo(\textit{Inheritance})}
			\label{SSGA-Inher}
			\end{algorithm}
			
		
	\section{Resultados}\footnote{En este documento, se comparan los resultados obtenidos por las
	heurísticas desarrolladas en esta práctica con la búsqueda de soluciones por la heurística
	\textbf{SFS} cuyos resultados ya se expusieron en \textbf{P1-Memoria}, por lo que también he
	obviado su inclusión en este documento en toda su extensión mas sí han sido incluidas las medias
	de ejecución en la tabla comparativa.}
		\subsection{Algoritmo Genético Gerenacional (\textbf{GGA})}
			\begin{table}[H]
	\centering
	\begin{tabular}{l|lll}
		Nombre        & Tasa de acierto(\%) & Tasa de reducción(\%) & Tiempo(s)          \\ \hline
		Partición 1-1 & 96.12676056338029   & 33.333333333333336    & 337.041            \\
		Partición 1-2 & 92.63157894736842   & 40.0                  & 403.86             \\
		Partición 2-1 & 96.83098591549296   & 53.333333333333336    & 521.958            \\
		Partición 2-2 & 94.3859649122807    & 56.666666666666664    & 555.581            \\
		Partición 3-1 & 97.53521126760563   & 46.666666666666664    & 451.382            \\
		Partición 3-2 & 94.3859649122807    & 40.0                  & 476.275            \\
		Partición 4-1 & 95.4225352112676    & 46.666666666666664    & 304.983            \\
		Partición 4-2 & 95.43859649122807   & 50.0                  & 342.083            \\
		Partición 5-1 & 98.59154929577464   & 50.0                  & 416.544            \\
		Partición 5-2 & 91.2280701754386    & 40.0                  & 475.407            \\ \hline
		Media         & 95.25772176921176   & 45.66666666666667     & 428.51140000000004
	\end{tabular}
	\caption{WDBC - GGA}
	\label{WDBC-GGA}
\end{table}
			\begin{table}[H]
	\centering
	\begin{tabular}{l|lll}
		Nombre        & Tasa de acierto(\%) & Tasa de reducción(\%) & Tiempo(s) \\ \hline
		Partición 1-1 & 68.88888888888889   & 45.55555555555556     & 561.348   \\
		Partición 1-2 & 70.55555555555556   & 42.22222222222222     & 583.544   \\
		Partición 2-1 & 73.88888888888889   & 50.0                  & 512.32    \\
		Partición 2-2 & 67.77777777777777   & 47.77777777777778     & 527.348   \\
		Partición 3-1 & 73.33333333333333   & 54.44444444444444     & 493.55    \\
		Partición 3-2 & 62.22222222222222   & 53.333333333333336    & 546.685   \\
		Partición 4-1 & 67.77777777777777   & 54.44444444444444     & 517.759   \\
		Partición 4-2 & 68.33333333333333   & 45.55555555555556     & 553.199   \\
		Partición 5-1 & 65.55555555555556   & 51.111111111111114    & 519.908   \\
		Partición 5-2 & 66.66666666666667   & 37.77777777777778     & 566.458   \\ \hline
		Media         & 68.49999999999999   & 48.22222222222222     & 538.2119
	\end{tabular}
	\caption{Movement Libras - GGA}
	\label{MLIB-GGA}
\end{table}
			\begin{table}[H]
	\centering
	\begin{tabular}{l|lll}
		Nombre        & Tasa de acierto(\%) & Tasa de reducción(\%) & Tiempo(s) \\ \hline
		Partición 1-1 & 56.994818652849744  & 52.51798561151079     & 1791.576  \\
		Partición 1-2 & 71.50259067357513   & 59.35251798561151     & 1795.453  \\
		Partición 2-1 & 63.73056994818653   & 51.798561151079134    & 1797.915  \\
		Partición 2-2 & 61.6580310880829    & 44.60431654676259     & 1873.818  \\
		Partición 3-1 & 63.21243523316062   & 51.07913669064748     & 1716.266  \\
		Partición 3-2 & 57.512953367875646  & 46.76258992805755     & 1898.207  \\
		Partición 4-1 & 63.73056994818653   & 50.35971223021583     & 1877.303  \\
		Partición 4-2 & 67.35751295336787   & 48.201438848920866    & 1839.052  \\
		Partición 5-1 & 63.73056994818653   & 55.39568345323741     & 1900.16   \\
		Partición 5-2 & 63.21243523316062   & 47.12230215827338     & 2002.21   \\ \hline
		Media         & 63.26424870466322   & 50.719424460431654    & 1849.196
	\end{tabular}
	\caption{Arrhythmia - GGA}
	\label{ARRH-GGA}
\end{table}
		\subsection{Algoritmo Genético Estacionario (\textbf{SSGA})}
			\begin{table}[H]
	\centering
	\begin{tabular}{l|lll}
		Nombre        & Tasa de acierto(\%) & Tasa de reducción(\%) & Tiempo(s)          \\ \hline
		Partición 1-1 & 95.07042253521126   & 50.0                  & 501.66             \\
		Partición 1-2 & 95.78947368421052   & 33.333333333333336    & 401.151            \\
		Partición 2-1 & 96.47887323943662   & 40.0                  & 464.078            \\
		Partición 2-2 & 95.08771929824562   & 50.0                  & 389.117            \\
		Partición 3-1 & 97.1830985915493    & 60.0                  & 466.005            \\
		Partición 3-2 & 95.43859649122807   & 30.0                  & 335.521            \\
		Partición 4-1 & 95.77464788732394   & 30.0                  & 470.225            \\
		Partición 4-2 & 94.03508771929825   & 46.666666666666664    & 352.6              \\
		Partición 5-1 & 96.83098591549296   & 63.333333333333336    & 328.103            \\
		Partición 5-2 & 94.3859649122807    & 43.333333333333336    & 403.805            \\ \hline
		Media         & 95.60748702742771   & 44.66666666666667     & 411.22650000000004
	\end{tabular}
	\caption{WDBC - SSGA}
	\label{WDBC-SSGA}
\end{table}
			\begin{table}[H]
	\centering
	\begin{tabular}{l|lll}
		Nombre        & Tasa de acierto(\%) & Tasa de reducción(\%) & Tiempo(s) \\ \hline
		Partición 1-1 & 70.55555555555556   & 44.44444444444444     & 567.594   \\
		Partición 1-2 & 71.11111111111111   & 53.333333333333336    & 488.135   \\
		Partición 2-1 & 71.11111111111111   & 38.888888888888886    & 614.843   \\
		Partición 2-2 & 68.33333333333333   & 52.22222222222222     & 485.332   \\
		Partición 3-1 & 73.33333333333333   & 44.44444444444444     & 574.318   \\
		Partición 3-2 & 66.66666666666667   & 47.77777777777778     & 533.625   \\
		Partición 4-1 & 71.11111111111111   & 51.111111111111114    & 501.068   \\
		Partición 4-2 & 66.11111111111111   & 60.0                  & 466.463   \\
		Partición 5-1 & 66.66666666666667   & 48.888888888888886    & 592.193   \\
		Partición 5-2 & 70.0                & 52.22222222222222     & 511.983   \\ \hline
		Media         & 69.5                & 49.33333333333333     & 533.5554
	\end{tabular}
	\caption{Movement Libras - SSGA}
	\label{MLIB-SSGA}
\end{table}
			\begin{table}[H]
	\centering
	\begin{tabular}{l|lll}
		Nombre        & Tasa de acierto(\%) & Tasa de reducción(\%) & Tiempo(s)          \\ \hline
		Partición 1-1 & 65.80310880829016   & 50.0                  & 1844.69            \\
		Partición 1-2 & 61.6580310880829    & 51.07913669064748     & 1950.531           \\
		Partición 2-1 & 62.69430051813472   & 48.201438848920866    & 2005.275           \\
		Partición 2-2 & 63.21243523316062   & 53.597122302158276    & 1646.877           \\
		Partición 3-1 & 67.87564766839378   & 43.16546762589928     & 2027.307           \\
		Partición 3-2 & 63.21243523316062   & 53.9568345323741      & 1908.012           \\
		Partición 4-1 & 59.58549222797927   & 45.32374100719424     & 2006.024           \\
		Partición 4-2 & 66.32124352331606   & 51.07913669064748     & 1936.516           \\
		Partición 5-1 & 64.76683937823834   & 51.07913669064748     & 1758.067           \\
		Partición 5-2 & 62.17616580310881   & 49.64028776978417     & 2068.818           \\ \hline
		Media         & 63.73056994818653   & 49.71223021582734     & 1915.2116999999998
	\end{tabular}
	\caption{Arrhythmia - SSGA}
	\label{ARRH-SSGA}
\end{table}
			
		\subsection{Comparación de resultados}
			\begin{table}[H]
	\centering
	\small
	\begin{tabular}{l|lll|lll|lll}
				& 			& WDBC 		&			&	  Mov	& ement 	& Libras	&			& Arrhyt	& hmia		\\ \hline
				& \%\_clas	& \%\_red	& T			& \%\_clas	& \%\_red	& T			& \%\_clas	& \%\_red	& T			\\ \hline
		SFS		& 93.1814	& 88.6667	& 1.8121	& 49.4444	& 94.3333	& 8.0742	& 66.4249	& 97.8058	& 128.5797	\\ \hline
		GGA		& 95.2577	& 45.6667	& 449.2955	& 68.5		& 48.2222	& 538.2119	& 62.8497	& 49.7482	& 2022.5882	\\ \hline
		SSGA	& 95.5374	& 46.6667	& 434.298	& 69.5		& 49.3333	& 533.5554	& 63.2642	& 48.3453	& 1882.0283
		
	\end{tabular}
	\caption{Comparación de resultados}
	\label{Compare}
\end{table}
			
			Viendo los resultados de arriba, podemos concluir que:
			\begin{itemize}
				\item Los mayores índices de reducción se obtienen con el \textit{SFS}, obteniendo en
				todos los casos una tasa de reducción muy alta, mientras que los resultados de los
				\textit{GA} rondan el 50\%.
				
				\item En el caso de \textit{WBDC}, todos los algoritmos tienen una alta tasa de acierto,
				pudiendo observarse que los resultados de los \textit{GA} bailan en torno al 95\% de
				acierto, con una ligera mejora del \textit{SSGA} frente a la variante \textit{GGA}.
				
				\item En el caso de \textit{Movement Libras}, se aprecian unos resultados similares
				al caso de \textit{WBDC} aunque con una menor tasa de aciertos, dando de nuevo como
				resultado que el \textit{SSGA} supera al \textit{GGA}.
				
				\item Sin duda alguna, para el caso de \textit{Arrhythmia}, el mejor algoritmo sería
				el de \textit{SFS}, ya que tiene el mejor índice de acierto y de reducción.
				
				\item En cuanto a los tiempos de ejecución, podemos observar que el más veloz sería el
				algoritmo greedy \textit{SFS} aunque podemos observar que de entre los \textit{GA}(los
				cuáles suelen tener un resultado significativamente mejor que \textit{SFS}), podemos
				ver que \textit{GGA} requiere algo más de tiempo en la mayor parte de los casos
				frente a \textit{SSGA}.
			\end{itemize}
			
			La diferencia entre las tasas de reducción de los distintos algoritmos se deben a que \textit{SFS}
			comienza con una máscara en la que ninguna característica está seleccionada mientras que
			\textit{GGA} y \textit{SSGA} toman una población de padres con unas máscaras en las cuáles
			la probabilidad de que una característica esté seleccionada es 50\%.
			
			Claramente, ni el operador de selección ni el de reemplazo, alteran las máscaras con las
			que trabaja la población. El operador de cruce, sí altera la máscara pero como trabaja con
			los individuos de la población, los padres tienen una reducción en torno al 50\% y el
			operador de cruce no impone condiciones sobre las características seleccionadas, no tenemos
			garantía, \textit{a priori}, de que vaya a aumentar o disminuir dicha tasa. Análogamente,
			el operador de mutación, altera la máscara al mutar una cantidad de genes de la población
			pero como no impone condiciones sobre la tasa de reducción y los padres tenían una tasa
			cercana al 50\%, es muy probable que la mitad de los genes que se alteran sean para seleccionar
			una característica y la otra mitad, para deseleccionarlos, es decir, no condiciona la tasa
			de reducción. Este resultado se respalda con los resultados obtenidos, donde se han obtenido
			tasas de reducción comprendidas entre el 30\% y el 65\%.
			
			Referente a los tiempos de ejecución, estaba claro que \textit{SFS} sería el más veloz pues
			solamente aplica una vez la heurística mientras los \textit{GA} requieren agotar las 15000
			iteraciones para acabar aunque convergen rápidamente a buenas soluciones, por lo que si no
			las agotara e hicieran un número mucho menor de iteraciones, obtendrían una solución de
			similar grado de calidad.
		
	\input{Compilación.tex}
	\newpage
	
	\begin{thebibliography}{10}
	\expandafter\ifx\csname url\endcsname\relax
	  \def\url#1{\texttt{#1}}\fi
	\expandafter\ifx\csname urlprefix\endcsname\relax\def\urlprefix{URL }\fi
	\expandafter\ifx\csname href\endcsname\relax
	  \def\href#1#2{#2} \def\path#1{#1}\fi
	
	\bibitem{KNN}
	WEKA (The University of Waikato)\\
	  \url{http://weka.sourceforge.net/doc.dev/weka/classifiers/lazy/IBk.html}\\
	  \url{http://weka.sourceforge.net/doc.dev/weka/core/neighboursearch/NearestNeighbourSearch.html}
	  
  	\bibitem{ARFFReader}
  	WEKA (The University of Waikato)\\
	  \url{http://weka.sourceforge.net/doc.dev/weka/core/converters/ArffLoader.ArffReader.html}
	  
	\end{thebibliography}

\end{document}
