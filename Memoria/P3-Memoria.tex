%%
% Plantilla de Memoria
% Modificación de una plantilla de Latex de Nicolas Diaz para adaptarla 
% al castellano y a las necesidades de escribir informática y matemáticas.
%
% Editada por: Mario Román
%
% License:
% CC BY-NC-SA 3.0 (http://creativecommons.org/licenses/by-nc-sa/3.0/)
%%

%%%%%%%%%%%%%%%%%%%%%
% Thin Sectioned Essay
% LaTeX Template
% Version 1.0 (3/8/13)
%
% This template has been downloaded from:
% http://www.LaTeXTemplates.com
%
% Original Author:
% Nicolas Diaz (nsdiaz@uc.cl) with extensive modifications by:
% Vel (vel@latextemplates.com)
%
% License:
% CC BY-NC-SA 3.0 (http://creativecommons.org/licenses/by-nc-sa/3.0/)
%
%%%%%%%%%%%%%%%%%%%%%

%----------------------------------------------------------------------------------------
%	PAQUETES Y CONFIGURACIÓN DEL DOCUMENTO
%----------------------------------------------------------------------------------------

%% Configuración del papel.
% microtype: Tipografía.
% mathpazo: Usa la fuente Palatino.
\documentclass[a4paper, 11pt]{article}
\usepackage[protrusion=true,expansion=true]{microtype}
\usepackage{mathpazo}

% Indentación de párrafos para Palatino
\setlength{\parindent}{0pt}
  \parskip=8pt
\linespread{1.05} % Change line spacing here, Palatino benefits from a slight increase by default


%% Castellano.
% noquoting: Permite uso de comillas no españolas.
% lcroman: Permite la enumeración con numerales romanos en minúscula.
% fontenc: Usa la fuente completa para que pueda copiarse correctamente del pdf.
\usepackage[spanish,es-noquoting,es-lcroman]{babel}
\usepackage[utf8]{inputenc}
\usepackage[T1]{fontenc}
\selectlanguage{spanish}


%% Gráficos
\usepackage{graphicx} % Required for including pictures
\usepackage{wrapfig} % Allows in-line images
\usepackage[usenames,dvipsnames]{color} % Coloring code

% % Enlaces
\usepackage[hidelinks]{hyperref}

%% Matemáticas
\usepackage{amsmath}

% Para algoritmos
\usepackage{algorithm}
\usepackage{algorithmic}
\usepackage{amsthm}
\input{spanishAlgorithmic.tex}

%% Bibliografía
\makeatletter
\renewcommand\@biblabel[1]{\textbf{#1.}} % Change the square brackets for each bibliography item from '[1]' to '1.'
\renewcommand{\@listI}{\itemsep=0pt} % Reduce the space between items in the itemize and enumerate environments and the bibliography


%----------------------------------------------------------------------------------------
%	TÍTULO
%----------------------------------------------------------------------------------------
% Configuraciones para el título.
% El título no debe editarse aquí.
\renewcommand{\maketitle}{
  \begin{flushright} % Center align
  {\LARGE\@title} % Increase the font size of the title
  
  \vspace{50pt} % Some vertical space between the title and author name
  
  {\large\@author} % Author name
  \\\@date % Date
  \vspace{40pt} % Some vertical space between the author block and abstract
  \end{flushright}
}

% Título
\title{\textbf{Metaheurísticas: Selección de Características}\\ % Title
P-3: Algoritmos Genéticos} % Subtitle

\author{\textsc{Óscar Bermúdez Garrido\\
\href{http://www.github.com/oxcar103}{@oxcar103}} % Author
\\{\textit{Universidad de Granada}}} % Institution

\date{\today} % Date


%----------------------------------------------------------------------------------------
%	DOCUMENTO
%----------------------------------------------------------------------------------------

\begin{document}

\maketitle % Print the title section

% Resumen (Descomentar para usarlo)
\renewcommand{\abstractname}{Resumen} % Uncomment to change the name of the abstract to something else
%\begin{abstract}
% Resumen aquí
%\end{abstract}

% Palabras clave
%\hspace*{3,6mm}\textit{Keywords:} lorem , ipsum , dolor , sit amet , lectus % Keywords
%\vspace{30pt} % Some vertical space between the abstract and first section


% Índice
{\parskip=2pt
  \tableofcontents
}
\pagebreak

%% Inicio del documento
	\section{Descripción del problema}
	Disponemos de una serie de muestras $w_1, w_2, \dots, w_r$ con un conjunto de caracterísitcas
	asociadas $(x_1(w_i), x_2, \dots, x_n(w_i))$ clasificadas en las distintas clases $\{C_1, C_2,
	\dots, C_M\}$.
	
	Queremos obtener una máscara que nos permita clasificar estas y futuras muestras a partir
	de sus atributos o características de forma automática, con la mayor tasa de acierto posible
	así como con la mayor tasa de reducción que se pueda, es decir, usando el menor número de
	características posible.
	
	Todo ello, en un tiempo razonable, pues a pesar de que podríamos encontrar el óptimo probando
	para todas las máscaras, debido a la dimensionalidad de algunos problemas, dicho coste en
	tiempo sería muy elevado.
	
	Para comprobar su funcionamiento, se suele recurrir a partir el conjunto de instancias dadas
	en dos particiones no necesariamente de igual cardinalidad. Sobre una de ellas se ejecutará
	el \textbf{Entrenamiento} donde se creará la máscara, y sobre la otra se llevará a cabo la
	\textbf{Validación} donde se usará la máscara creada para clasificar el conjunto y ver qué
	tasa de acierto tiene nuestro algoritmo. Por seguridad del buen funcionamiento del algoritmo,
	se realizan varias particiones entrenamiento-validación.
	
	Este problema tiene principalmente dos grandes adversidades que superar:
	\begin{itemize}
		\item La $"$maldición de la dimensión$"$(o efecto Hughes), que son el conjunto de fenómenos
		ocasionados a raíz del estudio y clasificación de datos de múltiple dimensionalidad.
		
		Estos fenómenos se deben porque a medida que la dimensión aumenta, el tamaño del espacio
		asociado se incrementa exponencialmente, complicando de este modo la predicción correcta
		del comportamiento de los individuos que lo componen.
		
		\item El $"$sobre-ajuste$"$ a los datos de entrenamiento, es decir, la creación de una
		máscara que maximiza la tasa de aciertos sobre los datos de entrenamiento pero dificultan
		la generalización sobre el problema, obteniendo malos resultados en la parte de validación
		y, por tanto, posiblemente en la clasificación de las futuras muestras.
	\end{itemize}

	\section{Características de las heurísticas}
	En este apartado, se describirán las características comunes empleadas para resolver
	el problema.
	
	\subsection{Representación de la solución}
		Representaremos la solución como un vector de booleanos de longitud $n$, con $n$
		el número de características del problema dado, en el cuál se especifica si la
		característica $f_i$ está en la solución o no.
	
	\subsection{Función objetivo y función evaluación}
		Como función objetivo lógica para nuestro problema, se tomaría la función:
		$$tasa_{aciertos} = 100 \cdot \frac{nº\ instancias\ bien\ clasificadas}{nº\ total\ de\ instancias}$$
		
		Sin embargo, como el número total de instancias es una constante y para la máquina
		es más sencillo trabajar con valores enteros que con valores decimales, optaremos
		por maximizar la función que nos calcula el número de aciertos y los transformaremos
		en la tasa de aciertos cuando queramos evaluar el buen funcionamiento de la
		clasificación realizada sobre el conjunto de evaluación.
		
		Un pequeño esquema del funcionamiento de nuestra función de evaluación sería:

		\begin{algorithm}[H]
			\begin{algorithmic}[1]

				\REQUIRE \ \\
		        	\texttt{c\_sel}, vector de características seleccionadas\\
		        	\texttt{inst}, instancias sobre las que evaluar\\ \

		     	\STATE{\texttt{aciertos} = 0}\\
		     	\FORALL{\texttt{instancia} $\in$ \texttt{inst}}
			  		\STATE{Genera \texttt{K-NN} de \texttt{instancia[i]}}
			  		\IF{\texttt{Hay clase predominante}}
						\STATE{\text{clase\ esperada} = Clase predominante de \texttt{K-NN} de \texttt{instancia[i]}}
					\ELSE
						\STATE{\text{clase\ esperada} = Clase de \texttt{1-NN} de \texttt{instancia[i]}}
					\ENDIF
			  		
					\IF{\texttt{clase\ esperada} == Clase de \texttt{instancia[i]}}
						\STATE{\texttt{aciertos++}}
					\ENDIF
				\ENDFOR
	  
				\RETURN{aciertos}			
			\end{algorithmic}
   \caption{Función de evaluación}
   \label{Evaluate}
		\end{algorithm}
		
		Este algoritmo tiene algunas variantes, como que si no le pasas instancia, toma la que
		usó para realizar el entrenamiento, si no le pasas vector de características seleccionadas,
		usa la que tiene almacenada y que si le pasas un entero $i$, usa el vector de características
		resultante de alternar el valor de $f_i$ de la solución almacenada.
		
		En nuestro caso, tomamos $K=3$ para realizar el KNN. Además, para el cálculo del KNN
		utilizamos el \textit{framework} que ya estaba implementado en \cite{KNN}.
	
	\subsection{Operador de vecinos}
		Para la generación de una solución vecina, basta con alterar la pertenencia de una
		característica al conjunto de características seleccionadas, y alterar el contador
		de las mismas ya que me parecía más eficiente de este modo llevar la cuenta de cuántas
		fueron escogidas. Una aproximación de sería la siguiente:
		
		\begin{algorithm}[H]
			\begin{algorithmic}[1]
				\REQUIRE \ \\
		        	\texttt{index}, índice a cambiar\\ \

		     	\STATE{\texttt{car[index]} = \texttt{!car[index]}}\\
		  		\IF{\texttt{car[index]} == \texttt{true}}
					\STATE{Incrementamos número de características seleccionadas}
				\ELSE
					\STATE{Decrementamos número de características seleccionadas}
				\ENDIF
			\end{algorithmic}
		\caption{Generación de solución vecina}
		\label{Flip}
		\end{algorithm}
	
		Y para el caso de la generación de un vecino aleatorio, bastaría con generarlo aleatoriamente,
		invocar a la función anterior y devolver el número generado (por si fuese necesario
		para algo en el ámbito en el que se invoca la función):
	
		\begin{algorithm}[H]
			\begin{algorithmic}[1]
				\REQUIRE \ \\
					 \

		     	\STATE{\texttt{index} = Entero Aleatorio}\\
		  		\STATE{\texttt{Flip(index)}}\\
		  		
				\RETURN{index}
			\end{algorithmic}
		\caption{Generación aleatoria de solución vecina}
		\label{Neighbour}
		\end{algorithm}
	
	\subsection{\textit{5x2-Cross Validation}}
		Para realizar la comprobación del buen funcionamiento de nuestras heurísticas, realizaremos
		una serie de particiones del conjunto de instancias que nos pasan como parámetro, algunas
		las usaremos como entrenamiento y otras como validación.
		
		En concreto, utilizaremos la conocida como \textit{5x2-Cross Validation}, que consiste en
		dividir el conjunto en dos mitades, tomar una como entrenamiento y la otra como validación,
		y luego invertir la que se usó como entrenamiento y la que se usó como validación. Este
		proceso se realiza 5 veces:
		
		\begin{algorithm}[H]
			\begin{algorithmic}[1]
				\REQUIRE \ \\
					 \
					 
		     	\FOR{\texttt{i} $<$ \texttt{5}}
			  		\STATE{\texttt{part1} = Genera partición}
			  		\STATE{\texttt{part2} = Complementario(\texttt{part1})}

					\STATE{\texttt{Heurística.entrenar(part1)}}
					\STATE{\texttt{evaluaciones[2*i]} = \texttt{Heurística.evaluar(part2)}}
									  		
					\STATE{\texttt{Heurística.entrenar(part2)}}
					\STATE{\texttt{evaluaciones[2*i+1]} = \texttt{Heurística.evaluar(part1)}}
				\ENDFOR
				
				\RETURN{evaluaciones}
			\end{algorithmic}
		\caption{\textit{5x2-Cross Validation}}
		\label{Cross-Validation}
		\end{algorithm}			
		
	\subsection{Variables globales}
		Existen las siguientes variables globales en el problema:
		\begin{itemize}
			\item \textbf{Semillas aleatorias:} Éstas dan cabida a la generación de los números
			aleatorios necesarios para ciertos procesos como la generación de la solución inicial
			o la generación de soluciones vecinas. Disponemos de 11 semillas aleatorias en esta
			implementación:
				\begin{itemize}
					\item \textbf{Principal:} Sólo existe una semilla aleatoria principal, a la
					que le dí el valor $103$.
					\item \textbf{Generadas:} Estas semillas están generadas de forma aleatoria
					a partir de la semilla principal. Cada una de ellas se pasa los algoritmos
					que la requieren\footnote{Todos menos \textit{SFS}.} a la hora de la
					inicialización del objeto. Existen 10 para pasar una en cada prueba de las
					heurísticas.
				\end{itemize}
			\item \textbf{Número máximo de evaluaciones:} Es un valor que nos permite limitar
			las iteraciones de los algoritmos(ya que, de otro modo, podrían no terminar o tardar
			mucho más tiempo)\footnote{Usada por todas excepto la \textit{SFS} porque necesitaría
			una cantidad tan desproporcionada de características en la instancia pasada que
			dicha variable carecería de sentido usarla.}. Le hemos dado el valor de 15000.
		\end{itemize}
	
	\subsection{Entrada y salida}
		Para recoger los datos a través de la entrada, he implementado una clase extra basándome
		en \cite{ARFFReader}.
		Para generar los archivos \textit{.csv}, he utilizado una clase implementada por un
		compañero y algo adaptada a mi estilo de programación.

	
	\section{Heurísticas implementadas}\footnote{En la implementación de las heurísticas desarrolladas
	en esta práctica, se utilizan la búsqueda de soluciones por la heurística \textbf{SFS} para comparar
	los resultados obtenidos y las optimizaciones de la \textbf{LS} para mejorar los resultados obtenidos
	por estas nuevas heurísticas multiarranque. Dado que ya se explicaron en \textbf{P1-Memoria}, he
	considerado ahorrar su explicación en este documento para permitir una mayor limpieza del mismo.}
	
		\subsection{Algoritmos Genéticos (\textbf{GA})}
			A grandes rasgos, podemos realizar una descripción general de los Algoritmos Genéticos o
			\textit{Genetic Algorithm}(\textit{GA}) que consiste en:
			
			\begin{itemize}
				\item Generar una población inicial
				\item Evolucionar dicha población, o equivalentemente se puede expresar mediante los
				siguientes operadores:
					\begin{itemize}
						\item Selección: Se eligen parejas de la población para reproducirse.
						\item Cruce: Se recombinan las parejas seleccionadas con una probabilidad de
						cruce característica del problema haciendo tomar a las soluciones sucesoras
						características de ambos padres.
						\item Mutación: Se modifican aleatoriamente los genes de la nueva generación
						con una probabilidad de mutación, normalmente, bastante reducida.
						\item Reemplazo: Se procede a incluir la nueva generación en la población
						que actualmente tenemos.
					\end{itemize}
				\item Tras el paso de algunas generaciones, nos quedamos con el mejor individuo que
				haya existido.
			\end{itemize}
			
			Este proceso se puede reflejar en el siguiente algoritmo\footnote{Cabría destacar que,
			realmente, \texttt{Padres\_Seleccionados} no se devuelve por el Operador de Selección, si
			no que se trata como una variable global que se ve modificada por cierto número de factores}
			donde la población para nuestro modelo es de 30 individuos:
			
			\begin{algorithm}[H]
				\begin{algorithmic}[1]
				\REQUIRE \ \\
						 \
			 
				\STATE{i = 0}
				\STATE{\texttt{Índice\_Mejor}}
				\FOR{i < \texttt{Población}}
					\STATE{\texttt{Padres}.add(RandomSolution())}
					\IF{\texttt{Padres[Índice\_Mejor]} < \texttt{Padres[i]}}
						\STATE{\texttt{Índice\_Mejor} = i}
					\ENDIF
				\ENDFOR
				
				\WHILE{\texttt{Evaluaciones} < 15000}
					\STATE{Limpiamos \texttt{Hijos}}
					\STATE{\texttt{Padres\_Seleccionados} = Selection()}
					\STATE{i = 0}
					\FOR{i < $\displaystyle \frac{\texttt{Padres\_Seleccionados}.size()}{2}$}
						\STATE{Crossover(2 $\cdot$ i, 2 $\cdot$ i+1)}
					\ENDFOR
					\STATE{Mutation()}
					\STATE{Inheritance()}
				\ENDWHILE
				
				\STATE{\texttt{Solución\_actual = Padres[Índice\_Mejor]}}
				\end{algorithmic}
			\caption{Algoritmos Genéticos - Entrenamiento(\textit{Train})}
			\label{GA-Train}
			\end{algorithm}
			
			Para una selección óptima de los individuos de la población que pasarán a cruzarse que nos
			permita una rápida convergencia de la población a una buena solución del problema, los
			elegiremos mediante el Torneo Binario consitente en:
			
			\begin{algorithm}[H]
				\begin{algorithmic}[1]
				\REQUIRE \ \\
						 \
					
				\STATE{\texttt{Candidato\_1} = Genera valor aleatorio entre 0 y \texttt{Población}-1}
				\STATE{\texttt{Candidato\_2} = Genera valor aleatorio entre 0 y \texttt{Población}-1}
				
				\WHILE{\texttt{Candidato\_1} == \texttt{Candidato\_2}}
					\STATE{\texttt{Candidato\_2} = Genera valor aleatorio entre 0 y \texttt{Población}-1}
				\ENDWHILE
				\IF{Evaluate(\texttt{Padres[Candidato\_1]}) > Evaluate(\texttt{Padres[Candidato\_2]})}
					\RETURN{\texttt{Candidato\_1}}
				\ELSE
					\RETURN{\texttt{Candidato\_2}}
				\ENDIF
				\STATE{\texttt{Solución\_actual = Padres[Índice\_Mejor]}}
				\end{algorithmic}
			\caption{Algoritmos Genéticos - Torneo Binario(\textit{BinaryTournament})}
			\label{GA-BT}
			\end{algorithm}
			
			Aunque, \textit{a priori}, los 4 operadores necesarios para evolucionar la población son
			esencialmente iguales con pequeños detalles a lo largo del proceso, el más fácil de abstraer
			para los Algoritmos Genéticos Generacional con Elitismo o \textit{Generational Genetic
			Algorithm}(\textit{GGA}) y Estacionario o \textit{Steady-State}(\textit{SSGA}) resulta ser
			el Operador de Cruce que quedaría como:
			
			\begin{algorithm}[H]
				\begin{algorithmic}[1]
				\REQUIRE \ \\
						\texttt{Padre\_1}, un padre \\
						\texttt{Padre\_2}, el otro \\ \
						
				\STATE{\texttt{Primero} = Genera valor aleatorio entre 0 y \texttt{Número\_Características}-1}
				\STATE{\texttt{Segundo} = Genera valor aleatorio entre 0 y \texttt{Número\_Características}-1}
				
				\WHILE{\texttt{Segundo} == \texttt{Primero}}
					\STATE{\texttt{Segundo} = Genera valor aleatorio entre 0 y \texttt{Número\_Características}-1}
				\ENDWHILE
				
				\IF{\texttt{Primero} > \texttt{Segundo}}
					\STATE{swap(\texttt{Primero, Segundo})}
				\ENDIF
				
				\STATE{i = 0}
				\FOR{i < \texttt{Número\_Características}}
					\IF{i < \texttt{Primero}}
						\STATE{\texttt{Hijo\_1}[i] = \texttt{Padre\_1}[i]}
						\STATE{\texttt{Hijo\_2}[i] = \texttt{Padre\_2}[i]}
					\ELSIF{i < \texttt{Segundo}}
						\STATE{\texttt{Hijo\_1}[i] = \texttt{Padre\_2}[i]}
						\STATE{\texttt{Hijo\_2}[i] = \texttt{Padre\_1}[i]}
					\ELSE
						\STATE{\texttt{Hijo\_1}[i] = \texttt{Padre\_1}[i]}
						\STATE{\texttt{Hijo\_2}[i] = \texttt{Padre\_2}[i]}
					\ENDIF
				\ENDFOR
				
				\STATE{\texttt{Hijos}.add{\texttt{Hijo\_1}}}
				\STATE{\texttt{Hijos}.add{\texttt{Hijo\_2}}}
				\end{algorithmic}
			\caption{Algoritmos Genéticos - Cruce(\textit{Crossover})}
			\label{GA-Cross}
			\end{algorithm}
			
		\subsection{Algoritmo Genético Gerenacional (\textbf{GGA})}
			Como ya hemos visto en el apartado anterior, los algoritmos genéticos se caracterizan por:
			
			\begin{itemize}
				\item Probabilidad de cruce: que en este caso particular tomará el valor de 0.7
				\item Probabilidad de mutación: que en esta ocasión será de 0.001
				\item Operadores de Evolución:
				\begin{itemize}
					\item Selección
					\item Cruce\footnote{Éste ya fue explicado en el apartado anterior.}
					\item Mutación
					\item Reemplazo
				\end{itemize}
			\end{itemize}
			
			Antes de explicar el Operador de Selección, cabe resaltar que el proceso usual para realizar
			dicha implementación consiste en dividir la población en parejas y comprobar para cada una
			de ellas si, aleatoriamente, cruzarán o no. Claramente, este proceso es muy costoso, por lo
			que se ha optado por la selección únicamente de las parejas que cruzarán sí o sí, lo que
			requiere previamente del cálculo de la esperanza matemática de individuos de la población
			que mutarán y la almacenaremos en una variable \textit{Número\_de\_Cruces} con expresión
			igual a $\displaystyle \frac{\textit{Probabilidad\_de\_Cruce} \cdot \textit{Población}}{2}$.
			
			Análogamente al caso de la probabilidad de cruce, y con mayor motivo pues la probabilidad
			de que ocurra este hecho se produce con menor frecuencia que el anterior, se calcula la
			cantidad de genes que mutarán como la esperanza matemática de dicho suceso, es decir
			$\textit{Número\_de\_Mutaciones} = \textit{Probabilidad\_de\_Mutación} \cdot \textit{Población}
			\cdot \textit{Número\_de\_Características}$
			
			Llegados a este punto, estamos preparados para representar el Operador de Selección de
			esta variante de los \textit{AG} como:
			
			\begin{algorithm}[H]
				\begin{algorithmic}[1]
				\REQUIRE \ \\
						 \
						
				\WHILE{\texttt{Padres\_Seleccionados}.size() < \texttt{Número\_de\_Cruces}}
					\STATE{\texttt{Candidato} = BinaryTournament}
					\IF{\texttt{Candidato} $\notin$ \texttt{Padres\_Seleccionados}}
						\STATE{\texttt{Padres\_Seleccionados}.add(\texttt{Candidato})}
					\ENDIF
				\ENDWHILE
				
				\RETURN{\texttt{Padres\_Seleccionados}}
				\end{algorithmic}
			\caption{AG Generacional - Selección(\textit{Selection})}
			\label{GGA-Selec}
			\end{algorithm}
			
			En el caso de \textit{GGA}, la mutación se realiza tanto a los nuevos miembros de la comunidad
			como a los miembros de la anterior generación que quedaron sin modificar. Esto se podría
			interpretar como que los padres que no se cruzaron han generado en la nueva generación
			una serie de individuos sucesores idénticos a los predecesores.
			
			De igual manera, obtenemos resultado de la variable \textit{Número\_de\_Mutaciones} recién
			creada mediante el Operador de Mutación que se describe:
			
			\begin{algorithm}[H]
				\begin{algorithmic}[1]
				\REQUIRE \ \\
						 \
						
				\STATE{i = 0}
				\FOR{i < \texttt{Población}}
					\IF{i $\notin$ \texttt{Padres\_Seleccionados}}
						\STATE{\texttt{Hijos}.add(\texttt{Padres}[i])}
					\ENDIF
				\ENDFOR
				\WHILE{\texttt{Genes\_Mutados}.size() < \texttt{Número\_de\_Mutaciones}}
					\STATE{\texttt{Gen} = Genera valor aleatorio entre 0 y (\texttt{Población} $\times$
							\texttt{Número\_de\_Características})-1}
					\STATE{\texttt{Solución} = $\displaystyle \frac{\texttt{Gen}}{\texttt{Número\_de\_Características}}$}
					\STATE{\texttt{Característica} = \texttt{Gen} $\mod{\texttt{Número\_de\_Características}}$}
					
					\IF{\texttt{Gen} $\notin$ \texttt{Genes\_Mutados}}
						\STATE{\texttt{Genes\_Mutados}.add(\texttt{Gen})}
						
						\IF{\texttt{Solución} $\notin$ \texttt{Soluciones\_Mutadas}}
							\STATE{\texttt{Soluciones\_Mutadas}.add(\texttt{Solución})}
						\ENDIF
						
						\STATE{\texttt{Solución}.Flip(\texttt{Gen})}
					\ENDIF
				\ENDWHILE
				\end{algorithmic}
			\caption{AG Generacional - Mutación(\textit{Mutation})}
			\label{GGA-Mut}
			\end{algorithm}
			
			Finalmente, queda el paso de mezclar la antigua generación con la nueva, esto se realiza
			mediante la inserción del mejor candidato o élite de la antigua generación en la nueva
			generación sustituyendo a la peor solución de ésta.
			
			En el caso de \textit{GGA}, el proceso es simple, la nueva generación sustituye por completo
			a la antigua, y posteriormente, se realiza una búsqueda para obtener la mejor solución o
			élite de la nueva:
			
			\begin{algorithm}[H]
				\begin{algorithmic}[1]
				\REQUIRE \ \\
						 \
						
				\STATE{\texttt{Índice\_Peor} = 0}
				\STATE{i = 0}
				\FOR{i < \texttt{Población}}
					\IF{Evaluate(\texttt{Hijos[i]}) < Evaluate(\texttt{Hijos[Índice\_Peor]})}
						\STATE{\texttt{Índice\_Peor} = i}
					\ENDIF
				\ENDFOR
				
				\STATE{\texttt{Hijos[Índice\_Peor]} = \texttt{Padres[Índice\_Mejor]}}
				\STATE{\texttt{Índice\_Mejor} = \texttt{Índice\_Peor}}
				\STATE{\texttt{Padres} = \texttt{Hijos}}
				
				\STATE{i = 0}
				\FOR{i < \texttt{Población}}
					\IF{Evaluate(\texttt{Padres[i]}) > Evaluate(\texttt{Padres[Índice\_Mejor]})}
						\STATE{\texttt{Índice\_Mejor} = i}
					\ENDIF
				\ENDFOR
				\end{algorithmic}
			\caption{AG Generacional - Reemplazo(\textit{Inheritance})}
			\label{GGA-Inher}
			\end{algorithm}
			
		\subsection{Algoritmo Genético Estacionario (\textbf{SSGA})}
			Para el caso del Algoritmo Genético Estacionario o \textit{Steady-State}(\textit{SSGA}),
			tomaremos la misma probabilidad de mutación que para \textit{GGA}, es decir, 0.001 y un
			valor de 1 para la probabilidad de cruce, estableciendo de este modo que todas las parejas
			cruzarán, además, para este \textit{AG} necesitamos indicar el número de hijos que intentará
			entrar a formar parte de la población, le asignaremos el valor 2 aunque los algoritmos
			estarán diseñados para funcionar con cualquier valor\footnote{De hecho, corre por cuenta
			del programador asignarle un valor par ya que de tomar un valor impar el Operador de Cruce
			accedería a zonas de memoria no asignada y en ningún lugar se comprueba.}
			
			El Operador de Selección de los \textit{SSGA} se expresaría de la forma:
			
			\begin{algorithm}[H]
				\begin{algorithmic}[1]
				\REQUIRE \ \\
						 \
				
				\WHILE{\texttt{Padres\_Seleccionados}.size() < \texttt{Número\_de\_Hijos}}
					\STATE{\texttt{Candidato} = BinaryTournament()}
					\IF{\texttt{Candidato} $\notin$ \texttt{Padres\_Seleccionados}}
						\STATE{\texttt{Padres\_Seleccionados}.add(\texttt{Candidato})}
					\ENDIF
				\ENDWHILE
				
				\RETURN{\texttt{Padres\_Seleccionados}}
				\end{algorithmic}
			\caption{AG Estacionario - Selección(\textit{Selection})}
			\label{SSGA-Selec}
			\end{algorithm}
			
			En esta ocasión, a diferencia de \textit{GGA}, para cada característica tomaremos números
			aleatorios para establecer si se produce la mutación de la misma que, aunque es bastante más
			ineficiente, la forma correcta de hacerlo.
			
			Y como resultado, el Operador de Mutación quedaría como:
			
			\begin{algorithm}[H]
				\begin{algorithmic}[1]
				\REQUIRE \ \\
						 \
						
				\STATE{i = 0}
				\FOR{i < \texttt{Hijos}.size()}
					\STATE{j = 0}
					\FOR{j < \texttt{Número\_de\_Características}}
						\STATE{\texttt{rnd} = Genera valor aleatorio entre 0 y 1}
						\IF{rnd < \texttt{Probabilidad\_Mutación}}
							\STATE{i.Flip(j)}
						\ENDIF
					\ENDFOR
				\ENDFOR
				\end{algorithmic}
			\caption{AG Estacionario - Mutación(\textit{Mutation})}
			\label{SSGA-Mut}
			\end{algorithm}
			
			En este caso, se intenta integrar a los hijos a la población uno a uno, esto es, intentamos
			que entren si y solo si mejoran alguna solución ya existente en la generación anterior de
			la población. Una forma de realizarlo sería:
			
			\begin{itemize}
				\item Buscar la peor solución de la anterior generación
				\item Recorrer los hijos hasta que alguno lo mejore
				\item Recordar el índice de este hijo
				\item Buscar la siguiente peor solución
				\item Recorrer los hijos a partir del siguiente al índice que recordamos. Esto se debe
				a que si no superaron la primera solución peor encontrada, no van a superar a la segunda.
				\item Repetir el proceso hasta que no queden nuevos hijos por recorrer(en este caso,
				serían sólo 2 pero el algoritmo está planteado para el caso de que sean cualquier número).
			\end{itemize}
			
			Esta idea queda perfectamente reflejada en el siguiente algoritmo:
			
			\begin{algorithm}[H]
				\begin{algorithmic}[1]
				\REQUIRE \ \\
						 \
						
				\STATE{\texttt{Índice\_Hijo} = 0}
				\WHILE{\texttt{Índice\_Hijo} < \texttt{Hijos}.size()}
					\STATE{\texttt{Índice\_Peor} = 0}
					\STATE{\texttt{Reemplazado\_Peor} = \FALSE}
					\STATE{i = 0}
					\FOR{i < \texttt{Padres}.size()}
						\IF{Evaluate(\texttt{Padres[i]}) < Evaluate(\texttt{Padres[Índice\_Peor]})}
							\STATE{\texttt{Índice\_Peor} = i}
						\ENDIF
					\ENDFOR
					
					\WHILE{\texttt{Índice\_Hijo} < \texttt{Hijos}.size() \AND \NOT \texttt{Reemplazado\_Peor}}
						\IF{Evaluate({Índice\_Hijo}) > Evaluate(\texttt{Padres(Índice\_Peor)})}
							\STATE{\texttt{Padres[Índice\_Peor]} = \texttt{Hijos[Índice\_Hijo]}}
							\STATE{\texttt{Reemplazado\_Peor} = \TRUE}
							\IF{Evaluate(\texttt{Padres[Índice\_Peor]}) > Evaluate(\texttt{Padres[Índice\_Mejor]})}
								\STATE{\texttt{Índice\_Mejor} = \texttt{Índice\_Peor}}
							\ENDIF
						\ENDIF
						
						\STATE{\texttt{Índice\_Hijo}}
					\ENDWHILE
				\ENDWHILE
				\end{algorithmic}
			\caption{AG Estacionario - Reemplazo(\textit{Inheritance})}
			\label{SSGA-Inher}
			\end{algorithm}
			
		
	\section{Resultados}\footnote{En este documento, se comparan los resultados obtenidos por las
	heurísticas desarrolladas en esta práctica con la búsqueda de soluciones por la heurística
	\textbf{SFS} cuyos resultados ya se expusieron en \textbf{P1-Memoria}, por lo que también he
	obviado su inclusión en este documento en toda su extensión mas sí han sido incluidas las medias
	de ejecución en la tabla comparativa.}
		\subsection{Algoritmo Genético Gerenacional (\textbf{GGA})}
			\begin{table}[H]
	\centering
	\begin{tabular}{l|lll}
		Nombre        & Tasa de acierto(\%) & Tasa de reducción(\%) & Tiempo(s) \\ \hline
		Partición 1-1 & 96.12676056338029   & 33.333333333333336    & 360.295   \\
		Partición 1-2 & 92.63157894736842   & 40.0                  & 482.859   \\
		Partición 2-1 & 96.83098591549296   & 53.333333333333336    & 490.472   \\
		Partición 2-2 & 94.3859649122807    & 56.666666666666664    & 517.027   \\
		Partición 3-1 & 97.53521126760563   & 46.666666666666664    & 424.538   \\
		Partición 3-2 & 94.3859649122807    & 40.0                  & 484.934   \\
		Partición 4-1 & 95.4225352112676    & 46.666666666666664    & 389.957   \\
		Partición 4-2 & 95.43859649122807   & 50.0                  & 410.407   \\
		Partición 5-1 & 98.59154929577464   & 50.0                  & 495.343   \\
		Partición 5-2 & 91.2280701754386    & 40.0                  & 437.123   \\ \hline
		Media         & 95.25772176921176   & 45.66666666666667     & 449.2955 
	\end{tabular}
	\caption{WDBC - GGA}
	\label{WDBC-GGA}
\end{table}
			\input{MLIB-GGA}
			\begin{table}[H]
	\centering
	\begin{tabular}{l|lll}
		Nombre        & Tasa de acierto(\%) & Tasa de reducción(\%) & Tiempo(s)          \\ \hline
		Partición 1-1 & 59.067357512953365  & 49.64028776978417     & 1945.927           \\
		Partición 1-2 & 71.50259067357513   & 52.15827338129496     & 2025.191           \\
		Partición 2-1 & 59.58549222797927   & 49.280575539568346    & 1975.14            \\
		Partición 2-2 & 66.32124352331606   & 52.87769784172662     & 2144.821           \\
		Partición 3-1 & 63.73056994818653   & 47.84172661870504     & 1881.088           \\
		Partición 3-2 & 54.40414507772021   & 46.76258992805755     & 2028.0             \\
		Partición 4-1 & 61.13989637305699   & 50.0                  & 2134.335           \\
		Partición 4-2 & 64.76683937823834   & 52.15827338129496     & 2006.075           \\
		Partición 5-1 & 63.21243523316062   & 48.201438848920866    & 1941.654           \\
		Partición 5-2 & 64.76683937823834   & 48.56115107913669     & 2143.651           \\ \hline
		Media         & 62.8497409326425    & 49.74820143884892     & 2022.5881999999997
	\end{tabular}
	\caption{Arrhythmia - GGA}
	\label{ARRH-GGA}
\end{table}
		\subsection{Algoritmo Genético Estacionario (\textbf{SSGA})}
			\begin{table}[H]
	\centering
	\begin{tabular}{l|lll}
		Nombre        & Tasa de acierto(\%) & Tasa de reducción(\%) & Tiempo(s)          \\ \hline
		Partición 1-1 & 95.77464788732394   & 60.0                  & 357.926            \\
		Partición 1-2 & 95.43859649122807   & 50.0                  & 416.568            \\
		Partición 2-1 & 96.47887323943662   & 40.0                  & 474.991            \\
		Partición 2-2 & 94.3859649122807    & 43.333333333333336    & 483.991            \\
		Partición 3-1 & 97.1830985915493    & 60.0                  & 481.247            \\
		Partición 3-2 & 95.43859649122807   & 30.0                  & 380.815            \\
		Partición 4-1 & 96.47887323943662   & 33.333333333333336    & 510.383            \\
		Partición 4-2 & 94.03508771929825   & 46.666666666666664    & 394.098            \\
		Partición 5-1 & 95.77464788732394   & 63.333333333333336    & 364.539            \\
		Partición 5-2 & 94.3859649122807    & 40.0                  & 478.422            \\ \hline
		Media         & 95.53743513713863   & 46.66666666666667     & 434.29799999999994
	\end{tabular}
	\caption{WDBC - SSGA}
	\label{WDBC-SSGA}
\end{table}
			\begin{table}[H]
	\centering
	\begin{tabular}{l|lll}
		Nombre        & Tasa de acierto(\%) & Tasa de reducción(\%) & Tiempo(s) \\ \hline
		Partición 1-1 & 70.55555555555556   & 44.44444444444444     & 567.594   \\
		Partición 1-2 & 71.11111111111111   & 53.333333333333336    & 488.135   \\
		Partición 2-1 & 71.11111111111111   & 38.888888888888886    & 614.843   \\
		Partición 2-2 & 68.33333333333333   & 52.22222222222222     & 485.332   \\
		Partición 3-1 & 73.33333333333333   & 44.44444444444444     & 574.318   \\
		Partición 3-2 & 66.66666666666667   & 47.77777777777778     & 533.625   \\
		Partición 4-1 & 71.11111111111111   & 51.111111111111114    & 501.068   \\
		Partición 4-2 & 66.11111111111111   & 60.0                  & 466.463   \\
		Partición 5-1 & 66.66666666666667   & 48.888888888888886    & 592.193   \\
		Partición 5-2 & 70.0                & 52.22222222222222     & 511.983   \\ \hline
		Media         & 69.5                & 49.33333333333333     & 533.5554
	\end{tabular}
	\caption{Movement Libras - SSGA}
	\label{MLIB-SSGA}
\end{table}
			\begin{table}[H]
	\centering
	\begin{tabular}{l|lll}
		Nombre        & Tasa de acierto(\%) & Tasa de reducción(\%) & Tiempo(s)          \\ \hline
		Partición 1-1 & 65.80310880829016   & 50.0                  & 1844.69            \\
		Partición 1-2 & 61.6580310880829    & 51.07913669064748     & 1950.531           \\
		Partición 2-1 & 62.69430051813472   & 48.201438848920866    & 2005.275           \\
		Partición 2-2 & 63.21243523316062   & 53.597122302158276    & 1646.877           \\
		Partición 3-1 & 67.87564766839378   & 43.16546762589928     & 2027.307           \\
		Partición 3-2 & 63.21243523316062   & 53.9568345323741      & 1908.012           \\
		Partición 4-1 & 59.58549222797927   & 45.32374100719424     & 2006.024           \\
		Partición 4-2 & 66.32124352331606   & 51.07913669064748     & 1936.516           \\
		Partición 5-1 & 64.76683937823834   & 51.07913669064748     & 1758.067           \\
		Partición 5-2 & 62.17616580310881   & 49.64028776978417     & 2068.818           \\ \hline
		Media         & 63.73056994818653   & 49.71223021582734     & 1915.2116999999998
	\end{tabular}
	\caption{Arrhythmia - SSGA}
	\label{ARRH-SSGA}
\end{table}
			
		\subsection{Comparación de resultados}
			\begin{table}[H]
	\centering
	\small
	\begin{tabular}{l|lll|lll|lll}
				& 			& WDBC 		&			&	  Mov	& ement 	& Libras	&			& Arrhyt	& hmia		\\ \hline
				& \%\_clas	& \%\_red	& T			& \%\_clas	& \%\_red	& T			& \%\_clas	& \%\_red	& T			\\ \hline
		SFS		& 93.1814	& 88.6667	& 1.8121	& 49.4444	& 94.3333	& 8.0742	& 66.4249	& 97.8058	& 128.5797	\\ \hline
		GGA		& 95.2577	& 45.6667	& 449.2955	& 68.5		& 48.2222	& 538.2119	& 62.8497	& 49.7482	& 2022.5882	\\ \hline
		SSGA	& 95.5374	& 46.6667	& 434.298	& 69.5		& 49.3333	& 533.5554	& 63.2642	& 48.3453	& 1882.0283
		
	\end{tabular}
	\caption{Comparación de resultados}
	\label{Compare}
\end{table}
			
			Viendo los resultados de arriba, podemos concluir que:
			\begin{itemize}
				\item Los mayores índices de reducción se obtienen con el \textit{SFS}, obteniendo en
				todos los casos una tasa de reducción muy alta, mientras que los resultados de los
				\textit{GA} rondan el 50\%.
				
				\item En el caso de \textit{WBDC}, todos los algoritmos tienen una alta tasa de acierto,
				pudiendo observarse que los resultados de los \textit{GA} bailan en torno al 95\% de
				acierto, con una ligera mejora del \textit{SSGA} frente a la variante \textit{GGA}.
				
				\item En el caso de \textit{Movement Libras}, se aprecian unos resultados similares
				al caso de \textit{WBDC} aunque con una menor tasa de aciertos, dando de nuevo como
				resultado que el \textit{SSGA} supera al \textit{GGA}.
				
				\item Sin duda alguna, para el caso de \textit{Arrhythmia}, el mejor algoritmo sería
				el de \textit{SFS}, ya que tiene el mejor índice de acierto y de reducción.
				
				\item En cuanto a los tiempos de ejecución, podemos observar que el más veloz sería el
				algoritmo greedy \textit{SFS} aunque podemos observar que de entre los \textit{GA}(los
				cuáles suelen tener un resultado significativamente mejor que \textit{SFS}), podemos
				ver que \textit{GGA} requiere algo más de tiempo en la mayor parte de los casos
				frente a \textit{SSGA}.
			\end{itemize}
			
			La diferencia entre las tasas de reducción de los distintos algoritmos se deben a que \textit{SFS}
			comienza con una máscara en la que ninguna característica está seleccionada mientras que
			\textit{GGA} y \textit{SSGA} toman una población de padres con unas máscaras en las cuáles
			la probabilidad de que una característica esté seleccionada es 50\%.
			
			Claramente, ni el operador de selección ni el de reemplazo, alteran las máscaras con las
			que trabaja la población. El operador de cruce, sí altera la máscara pero como trabaja con
			los individuos de la población, los padres tienen una reducción en torno al 50\% y el
			operador de cruce no impone condiciones sobre las características seleccionadas, no tenemos
			garantía, \textit{a priori}, de que vaya a aumentar o disminuir dicha tasa. Análogamente,
			el operador de mutación, altera la máscara al mutar una cantidad de genes de la población
			pero como no impone condiciones sobre la tasa de reducción y los padres tenían una tasa
			cercana al 50\%, es muy probable que la mitad de los genes que se alteran sean para seleccionar
			una característica y la otra mitad, para deseleccionarlos, es decir, no condiciona la tasa
			de reducción. Este resultado se respalda con los resultados obtenidos, donde se han obtenido
			tasas de reducción comprendidas entre el 30\% y el 65\%.
			
			Referente a los tiempos de ejecución, estaba claro que \textit{SFS} sería el más veloz pues
			solamente aplica una vez la heurística mientras los \textit{GA} requieren agotar las 15000
			iteraciones para acabar aunque convergen rápidamente a buenas soluciones, por lo que si no
			las agotara e hicieran un número mucho menor de iteraciones, obtendrían una solución de
			similar grado de calidad.
		
	\section{Implementación y compilación del proyecto}
	Este proyecto se ha realizado en Java en el entorno de desarrollo NetBeans y se han utilizado
	los \textit{frameworks} de \textit{Weka} principalmente para el tratamiento previo del
	conjunto de instancias(normalización de los mismos), la separación de las instancias en
	instancias de Entrenamiento y de Valoración y para la llamada al $K-NN$ utilizada en la
	función de valoración.
	
	Para la compilación del proyecto, se recomienda el uso de la plataforma NetBeans por su
	sencillez a la hora de compilar, \textit{$"$debuggear$"$} y ejecutar el proyecto(aunque,
	como usa parámetros de entrada, en lo personal, prefiero ejecutarlo por terminal).
	
	No obstante, se puede realizar la compilación del proyecto con \textit{ant -f} <nombre del
	proyecto> y su ejecución mediante \textit{java -jar} <nombre del proyecto> <parámetros>.
	
	Los parámetros serían:
		\begin{itemize}
			\item Número de archivos en los que se le pedirá aplicar las heurísticas.
			\item Párametros de cada fichero, es decir:
				\begin{itemize}
					\item Dirección al fichero.
					\item Columna en la que se encuentra la característica de clase.
					\item Base del nombre del fichero de salida\footnote{Por cada fichero
					de entrada, se generan 4 de salida, uno por cada algoritmo}.
				\end{itemize}
		\end{itemize}

	\newpage
	
	\begin{thebibliography}{10}
	\expandafter\ifx\csname url\endcsname\relax
	  \def\url#1{\texttt{#1}}\fi
	\expandafter\ifx\csname urlprefix\endcsname\relax\def\urlprefix{URL }\fi
	\expandafter\ifx\csname href\endcsname\relax
	  \def\href#1#2{#2} \def\path#1{#1}\fi
	
	\bibitem{KNN}
	WEKA (The University of Waikato)\\
	  \url{http://weka.sourceforge.net/doc.dev/weka/classifiers/lazy/IBk.html}\\
	  \url{http://weka.sourceforge.net/doc.dev/weka/core/neighboursearch/NearestNeighbourSearch.html}
	  
  	\bibitem{ARFFReader}
  	WEKA (The University of Waikato)\\
	  \url{http://weka.sourceforge.net/doc.dev/weka/core/converters/ArffLoader.ArffReader.html}
	  
	\end{thebibliography}

\end{document}
