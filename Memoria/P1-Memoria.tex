%%
% Plantilla de Memoria
% Modificación de una plantilla de Latex de Nicolas Diaz para adaptarla 
% al castellano y a las necesidades de escribir informática y matemáticas.
%
% Editada por: Mario Román
%
% License:
% CC BY-NC-SA 3.0 (http://creativecommons.org/licenses/by-nc-sa/3.0/)
%%

%%%%%%%%%%%%%%%%%%%%%
% Thin Sectioned Essay
% LaTeX Template
% Version 1.0 (3/8/13)
%
% This template has been downloaded from:
% http://www.LaTeXTemplates.com
%
% Original Author:
% Nicolas Diaz (nsdiaz@uc.cl) with extensive modifications by:
% Vel (vel@latextemplates.com)
%
% License:
% CC BY-NC-SA 3.0 (http://creativecommons.org/licenses/by-nc-sa/3.0/)
%
%%%%%%%%%%%%%%%%%%%%%

%----------------------------------------------------------------------------------------
%	PAQUETES Y CONFIGURACIÓN DEL DOCUMENTO
%----------------------------------------------------------------------------------------

%% Configuración del papel.
% microtype: Tipografía.
% mathpazo: Usa la fuente Palatino.
\documentclass[a4paper, 11pt]{article}
\usepackage[protrusion=true,expansion=true]{microtype}
\usepackage{mathpazo}

% Indentación de párrafos para Palatino
\setlength{\parindent}{0pt}
  \parskip=8pt
\linespread{1.05} % Change line spacing here, Palatino benefits from a slight increase by default


%% Castellano.
% noquoting: Permite uso de comillas no españolas.
% lcroman: Permite la enumeración con numerales romanos en minúscula.
% fontenc: Usa la fuente completa para que pueda copiarse correctamente del pdf.
\usepackage[spanish,es-noquoting,es-lcroman]{babel}
\usepackage[utf8]{inputenc}
\usepackage[T1]{fontenc}
\selectlanguage{spanish}


%% Gráficos
\usepackage{graphicx} % Required for including pictures
\usepackage{wrapfig} % Allows in-line images
\usepackage[usenames,dvipsnames]{color} % Coloring code

% % Enlaces
\usepackage[hidelinks]{hyperref}

%% Matemáticas
\usepackage{amsmath}

% Para algoritmos
\usepackage{algorithm}
\usepackage{algorithmic}
\usepackage{amsthm}
\floatname{algorithm}{Algoritmo}
\renewcommand{\listalgorithmname}{Lista de algoritmos}
\renewcommand{\algorithmicrequire}{\textbf{Entrada:}}
\renewcommand{\algorithmicensure}{\textbf{Salida:}}
\renewcommand{\algorithmicend}{\textbf{fin}}
\renewcommand{\algorithmicif}{\textbf{si}}
\renewcommand{\algorithmicthen}{\textbf{entonces}}
\renewcommand{\algorithmicelse}{\textbf{en otro caso}}
\renewcommand{\algorithmicelsif}{\algorithmicelse,\ \algorithmicif}
\renewcommand{\algorithmicendif}{\algorithmicend\ \algorithmicif}
\renewcommand{\algorithmicfor}{\textbf{para }}
\renewcommand{\algorithmicforall}{\textbf{para cada}}
\renewcommand{\algorithmicdo}{\textbf{}}
\renewcommand{\algorithmicendfor}{\algorithmicend\ \algorithmicfor}
\renewcommand{\algorithmicwhile}{\textbf{mientras}}
\renewcommand{\algorithmicendwhile}{\algorithmicend\ \algorithmicwhile}
\renewcommand{\algorithmicloop}{\textbf{repetir}}
\renewcommand{\algorithmicendloop}{\algorithmicend\ \algorithmicloop}
\renewcommand{\algorithmicrepeat}{\textbf{repetir}}
\renewcommand{\algorithmicuntil}{\textbf{hasta que}}
\renewcommand{\algorithmicprint}{\textbf{imprimir}} 
\renewcommand{\algorithmicreturn}{\textbf{devolver}} 
\renewcommand{\algorithmictrue}{\textbf{true }} 
\renewcommand{\algorithmicfalse}{\textbf{false }} 
\renewcommand{\algorithmicand}{\textbf{y}}
\renewcommand{\algorithmicor}{\textbf{o}}


%% Bibliografía
\makeatletter
\renewcommand\@biblabel[1]{\textbf{#1.}} % Change the square brackets for each bibliography item from '[1]' to '1.'
\renewcommand{\@listI}{\itemsep=0pt} % Reduce the space between items in the itemize and enumerate environments and the bibliography


%----------------------------------------------------------------------------------------
%	TÍTULO
%----------------------------------------------------------------------------------------
% Configuraciones para el título.
% El título no debe editarse aquí.
\renewcommand{\maketitle}{
  \begin{flushright} % Center align
  {\LARGE\@title} % Increase the font size of the title
  
  \vspace{50pt} % Some vertical space between the title and author name
  
  {\large\@author} % Author name
  \\\@date % Date
  \vspace{40pt} % Some vertical space between the author block and abstract
  \end{flushright}
}

% Título
\title{\textbf{Metaheurísticas: Selección de Características}\\ % Title
P-1: Búsquedas por Trayectorias Simples} % Subtitle

\author{\textsc{Óscar Bermúdez Garrido\\
\href{http://www.github.com/oxcar103}{@oxcar103}} % Author
\\{\textit{Universidad de Granada}}} % Institution

\date{\today} % Date


%----------------------------------------------------------------------------------------
%	DOCUMENTO
%----------------------------------------------------------------------------------------

\begin{document}

\maketitle % Print the title section

% Resumen (Descomentar para usarlo)
\renewcommand{\abstractname}{Resumen} % Uncomment to change the name of the abstract to something else
%\begin{abstract}
% Resumen aquí
%\end{abstract}

% Palabras clave
%\hspace*{3,6mm}\textit{Keywords:} lorem , ipsum , dolor , sit amet , lectus % Keywords
%\vspace{30pt} % Some vertical space between the abstract and first section


% Índice
{\parskip=2pt
  \tableofcontents
}
\pagebreak

%% Inicio del documento
	\input{Introducción.tex}
	\input{CaracterísticasComunes.tex}
	
	\section{Heurísticas implementadas}
		\subsection{\textit{Sequential Forward Selection} (\textbf{SFS})}
			Para realizar la comprobación del buen funcionamiento del resto de heurísticas, realizaremos
			una heurística \textit{Greedy} que nos dé una solución para el problema de clasificación
			de características.
			
			En particular, usaremos el \textit{Sequential Forward Selection} que se encarga de ir
			seleccionando las características a añadir una por una tomando siempre la que mayor mejora
			produce hasta que ya no se produce mejora.
			
			Para la inicialización del objeto, nos basta con aportarle las instancias de entrenamiento
			y el índice de la columna de clase.
			
			El proceso de entrenamiento se puede ilustrar mediante el siguiente algoritmo:

			\begin{algorithm}[H]
				\begin{algorithmic}[1]
					\REQUIRE \ \\
							 \

					\STATE{\texttt{hay\_mejora} = \texttt{true}}
				    \WHILE{\texttt{hay\_mejora} \AND \texttt{Número de características seleccionadas} $<$
				    		\texttt{Número de características}}
				     	\FOR{\texttt{i} $<$ \texttt{Número de características}}
							\IF{\texttt{Característica$_i$} no está seleccionada ni es de clase}
								\STATE{Guardamos \texttt{Evaluate(i)}}
							\ENDIF
							
							\IF{Es mejor que las soluciones vecinas generadas hasta ahora}
								\STATE{Guardamos el valor de la mejor}
								\STATE{Guardamos el índice de la característica mejorada}
							\ENDIF
						\ENDFOR
						
						\IF{Mejora la solución actual}
							\STATE{\texttt{Flip(índice de la característica mejorada)}}
							\STATE{Guardamos el valor de la solución actual}
						\ELSE
							\STATE{\texttt{hay\_mejora} = \texttt{false}}
						\ENDIF
					\ENDWHILE
				\end{algorithmic}
			\caption{\textit{Sequential Forward Selection}}
			\label{SFS}
			\end{algorithm}			

		\subsection{Búsqueda Local (\textbf{LS})}
			Como primera forma de búsqueda de la solución, utilizaremos la búsqueda local del primero
			mejor. Ésta consiste en:
				\begin{enumerate}
					\item Generar una solución inicial aleatoria\footnote{Aunque en nuestro caso, esto
					se hace en la inicialización del objeto.}.
					\item Tras esto, generamos una serie de vecinos aleatorios.
					\item Aceptamos el primero que mejore la solución actual.
					\item Si generamos un número determinado\footnote{En nuestro código, este número
					consiste en 1.5 $\cdot$ número de características de los datos para dar posibilidad
					a que se prueben todas las características aunque no lo comprobamos.} de vecinos y
					no se ha mejorado, acabamos.
				\end{enumerate}

			Para la inicialización del objeto, nos basta con aportarle las instancias de entrenamiento,
			el índice de la columna de clase y una semilla, pues requiere de un generador de números
			aleatorios para su correcto funcionamiento.

			El algoritmo de su función de entrenamiento sería algo así:
			
			\begin{algorithm}[H]
				\begin{algorithmic}[1]
					\REQUIRE \ \\
							 \

					\STATE{Hacemos copia de la solución actual y el número de características seleccionadas}
					\STATE{\texttt{evaluación\_actual} = Evaluate()}
					
					\WHILE{Iteraciones sin mejora < 1.5 $\cdot$ Número de características
							\AND No se exceda el número máximo de evaluaciones}
				     	\STATE{GenerateNeighbour()}
				     	
				     	\IF{Mejora evaluación actual}
				     		\STATE{Hacemos copia de esta solución y el número de características seleccionadas}
				     		\STATE{Iteraciones sin mejora = 0}
				     	\ELSE
				     		\STATE{Restauramos valores de la solución actual}
				     		\STATE{Incrementamos el número de iteraciones sin mejora}
				     	\ENDIF
					\ENDWHILE
				\end{algorithmic}
			\caption{Búsqueda Local}
			\label{LS}
			\end{algorithm}			

		\subsection{Enfriamiento Simulado (\textbf{SA})}
			Este algoritmo nos permite aceptar no sólo soluciones vecinas mejores, si no que también nos
			permite aceptar algunas que serían peores con el fin de salir de óptimos locales. Para ello,
			es necesario el uso de ciertas variables:

			\begin{itemize}
				\item $\sigma = 0.3$, probabilidad de aceptar una solución peor que la actual.
				\item $\mu = 0.3$, coeficiente que regula cómo de mala puede ser una solución peor
				aceptada.
				\item $T_0$ $\equiv$ Temperatura inicial = $\mu$ $\cdot$ Valor de la solución inicial / $\ln(\sigma)$,
				temperatura a la que comenzamos.
				\item $T_f$ $ \equiv$ Temperatura final = $10^{-3}$, cuando se alcance esta temperatura, se parará la
				búsqueda.
				\item Vecinos máximos = 10 $\cdot$ Número de características, que indica cuántos vecinos
				se pueden generar antes de que se produzca un enfriamiento.
				\item Éxitos máximos = 0.1 $\cdot$ Vecinos máximos $\equiv$ Número de características,
				que indica cuántos éxitos se pueden tener antes de que se produzca un enfriamiento.
				\item M $\equiv$ Número de Enfriamientos = Número máximo de evaluaciones / Vecinos máximos,
				establece cuántas veces se enfriará entre la temperatura inicial y la final
				\item $\displaystyle beta = \frac{T_0 - T_f}{M \cdot T_0 \cdot T_f}$
			\end{itemize}

			La variación de la temperatura, que para nuestra práctica se basa en el esquema de
			enfriamiento simulado de Cauchy modificado, se puede calcular como indica este simple
			algoritmo:
			
			\begin{algorithm}[H]
				\begin{algorithmic}[1]
					\REQUIRE \ \\
							 \
							 
				  	\STATE{$\displaystyle Temperatura = \frac{Temperatura}{1 + \beta * \texttt{Temperatura}}$}
				\end{algorithmic}
			\caption{Enfriamiento Simulado - Enfriamiento}
			\label{SA-Annealing}
			\end{algorithm}

			Mientras tanto, el método de aceptación de una solución, conocido como criterio de
			Metrópolis, se basa en este método:

			\begin{algorithm}[H]
				\begin{algorithmic}[1]
					\REQUIRE \ \\
						\texttt{eval\_act}, evaluación actual \\
						\texttt{eval\_asp}, evaluación candidata \\ \
							 
					\STATE{\texttt{Aceptar} = \texttt{true}}
				  	\IF{Empeora la solución}
				  		\STATE{\texttt{rnd} = Genera valor aleatorio entre 0 y 1}

				  		\STATE{$\displaystyle \texttt{exponente} = \frac{eval\_act - eval\_asp}{T}$}
				  		\IF{\texttt{rnd} > $e^{exponente}$}
				  			\STATE{\texttt{Aceptar} = \texttt{false}}
				  		\ENDIF
				  	\ENDIF
				  	
				  	\RETURN{Aceptar}
				\end{algorithmic}
			\caption{Enfriamiento Simulado - Criterio de metrópolis}
			\label{SA-mCrit}
			\end{algorithm}
			
			Finalmente, el algoritmo de búsqueda de solución sería:
			
			\begin{algorithm}[H]
				\begin{algorithmic}[1]
					\REQUIRE \ \\
							 \

					\STATE{Hacemos copia de la solución actual y el número de características seleccionadas}
					\STATE{\texttt{evaluación\_actual} = Evaluate()}
					\STATE{\texttt{Hay\_mejora} = \texttt{true}}
					
					\WHILE{Temperatura > Temperatura final \AND \texttt{Hay\_mejora}
							\AND No se exceda el número máximo de evaluaciones}
						\STATE{Vecinos = 0}
						\STATE{Éxitos = 0}
						
						\WHILE{Vecinos < Máximo Vecinos \AND Éxitos < Máximos Éxitos
								\AND No se exceda el número máximo de evaluaciones}
					     	\STATE{\texttt{índice} = GenerateNeighbour()}
					     	\STATE{Incrementamos Vecinos}
					     	\STATE{\texttt{evaluación\_vecino} = Evaluate()}
					     	
					     	\IF{MetropolisCriterion(\texttt{evaluación\_vecino}, \texttt{evaluación\_actual)}}
					     		\STATE{Hacemos copia de esta solución y el número de características seleccionadas}
					     		\STATE{Incrementamos Éxitos}
					     	\ELSE
					     		\STATE{Restauramos valores de la solución actual}
					     	\ENDIF
   						\ENDWHILE
   						
   						\STATE{CauchyAnnealing()}
   						\STATE{\texttt{Hay\_mejora} = (Éxitos == 0)}
					\ENDWHILE
				\end{algorithmic}
			\caption{Enfriamiento Simulado - Entrenamiento}
			\label{SA-Train}
			\end{algorithm}			

		\subsection{Búsqueda Tabú (\textbf{TS})}
			Este método de búsqueda almacena la lista de los últimos índices usados para no tomar
			esas posiciones en iteraciones futuras pues estaríamos deshaciendo la solución encontrada
			en iteraciones pasadas. Aunque esta es su política principal de actuación, si encuentra
			una solución mejor que la mejor encontrada hasta el momento, se permite tomarla sin
			importar si el índice está en la lista tabú.
			
			Los principales parámetros de esta búsqueda son:
			\begin{itemize}
				\item Número máximo de vecinos, que nos indica cuántas soluciones vecinas se exploran
				antes de tomar la decisión de cuál elegir como siguiente iteración. En nuestra
				implementación, le daremos el valor de 30.
				\item Tamaño de la lista tabú, que nos indica cuántos elementos tenemos como tabú y
				que, salvo que produzca una mejora superior a la mejor solución encontrada, nos indicará
				qué índices no podemos tomar como alternativa. Le daremos un tercio del número de
				características.
			\end{itemize}
		
			\begin{algorithm}[H]
				\begin{algorithmic}[1]
					\REQUIRE \ \\
							 \

					\STATE{Hacemos copia de la solución actual}
					\STATE{Tomamos como mejor solución global, la actual}
					\STATE{\texttt{evaluación\_actual} = Evaluate()}
					
					\WHILE{No se exceda el número máximo de evaluaciones}
						\STATE{Vecinos = 0}
						
						\WHILE{Vecinos < Máximo Vecinos \AND No se exceda el número máximo de evaluaciones}
					     	\STATE{\texttt{índice} = GenerateNeighbour()}
					     	\STATE{Incrementamos Vecinos}
					     	\STATE{\texttt{evaluación\_vecino} = Evaluate()}

					     	\IF{\texttt{evaluación\_vecino} mejora la solución global}
					     		\STATE{Tomamos esta solución y el índice como mejor vecino}
					     		\STATE{Tomamos esta solución como mejor global}
					     	\ELSE
					     		\IF{\texttt{índice} no está en la lista tabú}
						     		\IF{\texttt{evaluación\_vecino} mejora la solución de mejor vecino}
							     		\STATE{Tomamos esta solución y el índice como mejor vecino}
						     		\ENDIF
					     		\ENDIF
					     	\ENDIF
					     	
					     	\STATE{Restauramos valores de la solución actual}
   						\ENDWHILE
   						
						\STATE{Establecemos la mejor solución vecino como la actual}

						\IF{\texttt{índice} está en la lista tabú}
							\STATE{Lo recolocamos al principio}
						\ELSE
							\STATE{Lo añadimos al principio}
							
							\IF{La lista tabú está llena}
								\STATE{Quitamos el último}
							\ENDIF
						\ENDIF
					\ENDWHILE
					
					\STATE{Establecemos la mejor solución global como la actual}
				\end{algorithmic}
			\caption{Búsqueda tabú}
			\label{TS}
			\end{algorithm}			
			
	\section{Resultados}
		\subsection{\textit{Sequential Forward Selection} (\textbf{SFS})}
			\begin{table}[H]
	\centering
	\caption{WDBC - SFS}
	\label{WDBC-SFS}
	\begin{tabular}{l|lll}
		Nombre        & Tasa de acierto(\%) & Tasa de reducción(\%) & Tiempo(s)          \\ \hline
		Partición 1-1 & 95.4225352112676    & 86.66666666666667     & 2.799              \\
		Partición 1-2 & 93.33333333333333   & 86.66666666666667     & 2.157              \\
		Partición 2-1 & 88.73239436619718   & 93.33333333333333     & 1.042              \\
		Partición 2-2 & 91.9298245614035    & 90.0                  & 1.489              \\
		Partición 3-1 & 93.66197183098592   & 90.0                  & 1.615              \\
		Partición 3-2 & 92.98245614035088   & 86.66666666666667     & 2.164              \\
		Partición 4-1 & 93.66197183098592   & 83.33333333333333     & 2.496              \\
		Partición 4-2 & 92.98245614035088   & 90.0                  & 1.466              \\
		Partición 5-1 & 95.4225352112676    & 93.33333333333333     & 1.057              \\
		Partición 5-2 & 93.6842105263158    & 86.66666666666667     & 1.836              \\ \hline
		Media         & 93.18136891524587   & 88.66666666666667     & 1.8120999999999996
	\end{tabular}
\end{table}
			\begin{table}[H]
	\centering
	\caption{Movement Libras - SFS}
	\label{MLIB-SFS}
	\begin{tabular}{l|lll}
		Nombre        & Tasa de acierto(\%) & Tasa de reducción(\%) & Tiempo(s)         \\ \hline
		Partición 1-1 & 36.666666666666664  & 98.88888888888889     & 3.407             \\
		Partición 1-2 & 31.11111111111111   & 98.88888888888889     & 2.531             \\
		Partición 2-1 & 68.88888888888889   & 92.22222222222223     & 9.183             \\
		Partición 2-2 & 37.22222222222222   & 98.88888888888889     & 2.44              \\
		Partición 3-1 & 71.11111111111111   & 84.44444444444444     & 23.011            \\
		Partición 3-2 & 36.666666666666664  & 98.88888888888889     & 1.926             \\
		Partición 4-1 & 61.111111111111114  & 93.33333333333333     & 8.283             \\
		Partición 4-2 & 60.55555555555556   & 90.0                  & 13.016            \\
		Partición 5-1 & 63.333333333333336  & 88.88888888888889     & 14.337            \\
		Partición 5-2 & 27.77777777777778   & 98.88888888888889     & 2.608             \\ \hline
		Media         & 49.44444444444444   & 94.33333333333334     & 8.074200000000001
	\end{tabular}
\end{table}
			\begin{table}[H]
	\centering
	\caption{Arrhythmia - SFS}
	\label{ARRH-SFS}
	\begin{tabular}{l|lll}
		Nombre        & Tasa de acierto(\%) & Tasa de reducción(\%) & Tiempo(s) \\ \hline
		Partición 1-1 & 72.02072538860104   & 98.56115107913669     & 104.438   \\
		Partición 1-2 & 69.94818652849742   & 98.92086330935251     & 74.25     \\
		Partición 2-1 & 64.24870466321244   & 97.84172661870504     & 126.288   \\
		Partición 2-2 & 72.02072538860104   & 97.4820143884892      & 146.733   \\
		Partición 3-1 & 22.797927461139896  & 98.92086330935251     & 72.962    \\
		Partición 3-2 & 72.02072538860104   & 97.12230215827338     & 171.168   \\
		Partición 4-1 & 68.39378238341969   & 96.40287769784173     & 201.284   \\
		Partición 4-2 & 74.09326424870466   & 98.56115107913669     & 90.28     \\
		Partición 5-1 & 75.12953367875647   & 97.12230215827338     & 143.869   \\
		Partición 5-2 & 73.57512953367876   & 97.12230215827338     & 154.525   \\ \hline
		Media         & 66.42487046632125   & 97.80575539568346     & 128.5797 
	\end{tabular}
\end{table}
		
		\subsection{Búsqueda Local (\textbf{LS})}
			\begin{table}[H]
	\centering
	\caption{WDBC - LS}
	\label{WDBC-LS}
	\begin{tabular}{l|lll}
		Nombre        & Tasa de acierto(\%) & Tasa de reducción(\%) & Tiempo(s)          \\ \hline
		Partición 1-1 & 94.71830985915493   & 40.0                  & 3.029              \\
		Partición 1-2 & 96.14035087719299   & 66.66666666666667     & 1.48               \\
		Partición 2-1 & 96.12676056338029   & 43.333333333333336    & 3.144              \\
		Partición 2-2 & 93.6842105263158    & 33.333333333333336    & 1.627              \\
		Partición 3-1 & 94.36619718309859   & 50.0                  & 1.776              \\
		Partición 3-2 & 93.6842105263158    & 56.666666666666664    & 2.326              \\
		Partición 4-1 & 95.77464788732394   & 43.333333333333336    & 3.116              \\
		Partición 4-2 & 94.03508771929825   & 40.0                  & 1.821              \\
		Partición 5-1 & 97.53521126760563   & 46.666666666666664    & 1.839              \\
		Partición 5-2 & 95.78947368421052   & 63.333333333333336    & 1.361              \\ \hline
		Media         & 95.18544600938966   & 48.33333333333333     & 2.1519000000000004
	\end{tabular}
\end{table}
			\begin{table}[H]
	\centering
	\begin{tabular}{l|lll}
		Nombre        & Tasa de acierto(\%) & Tasa de reducción(\%) & Tiempo(s) \\ \hline
		Partición 1-1 & 69.44444444444444   & 52.22222222222222     & 16.5      \\
		Partición 1-2 & 73.88888888888889   & 51.111111111111114    & 7.563     \\
		Partición 2-1 & 69.44444444444444   & 53.333333333333336    & 8.554     \\
		Partición 2-2 & 68.33333333333333   & 42.22222222222222     & 10.246    \\
		Partición 3-1 & 68.88888888888889   & 47.77777777777778     & 8.363     \\
		Partición 3-2 & 68.88888888888889   & 47.77777777777778     & 8.623     \\
		Partición 4-1 & 65.0                & 57.77777777777778     & 6.367     \\
		Partición 4-2 & 70.55555555555556   & 48.888888888888886    & 7.427     \\
		Partición 5-1 & 71.66666666666667   & 48.888888888888886    & 8.756     \\
		Partición 5-2 & 71.66666666666667   & 47.77777777777778     & 5.863     \\ \hline
		Media         & 69.77777777777777   & 49.77777777777778     & 8.8262   
	\end{tabular}
	\caption{Movement Libras - LS}
	\label{MLIB-LS}
\end{table}
			\begin{table}[H]
	\centering
	\begin{tabular}{l|lll}
		Nombre        & Tasa de acierto(\%) & Tasa de reducción(\%) & Tiempo(s) \\ \hline
		Partición 1-1 & 66.32124352331606   & 48.56115107913669     & 171.05    \\
		Partición 1-2 & 61.13989637305699   & 43.16546762589928     & 92.406    \\
		Partición 2-1 & 62.17616580310881   & 56.47482014388489     & 155.303   \\
		Partición 2-2 & 66.83937823834196   & 46.0431654676259      & 169.566   \\
		Partición 3-1 & 64.76683937823834   & 53.9568345323741      & 117.503   \\
		Partición 3-2 & 64.76683937823834   & 46.76258992805755     & 143.886   \\
		Partición 4-1 & 61.13989637305699   & 50.0                  & 136.32    \\
		Partición 4-2 & 66.32124352331606   & 50.35971223021583     & 282.569   \\
		Partición 5-1 & 69.43005181347151   & 43.16546762589928     & 115.493   \\
		Partición 5-2 & 60.10362694300518   & 44.96402877697842     & 129.924   \\ \hline
		Media         & 64.30051813471503   & 48.34532374100719     & 151.402  
	\end{tabular}
	\caption{Arrhythmia - LS}
	\label{ARRH-LS}
\end{table}

		\subsection{Enfriamiento Simulado (\textbf{SA})}
			\begin{table}[H]
	\centering
	\begin{tabular}{l|lll}
		Nombre        & Tasa de acierto(\%) & Tasa de reducción(\%) & Tiempo(s) \\ \hline
		Partición 1-1 & 96.47887323943662   & 60.0                  & 5.783     \\
		Partición 1-2 & 96.84210526315789   & 50.0                  & 6.153     \\
		Partición 2-1 & 97.1830985915493    & 43.333333333333336    & 5.2       \\
		Partición 2-2 & 94.03508771929825   & 46.666666666666664    & 5.489     \\
		Partición 3-1 & 91.19718309859155   & 66.66666666666667     & 4.964     \\
		Partición 3-2 & 96.49122807017544   & 56.666666666666664    & 6.197     \\
		Partición 4-1 & 95.77464788732394   & 40.0                  & 5.941     \\
		Partición 4-2 & 91.57894736842105   & 53.333333333333336    & 4.924     \\
		Partición 5-1 & 96.12676056338029   & 43.333333333333336    & 5.187     \\
		Partición 5-2 & 95.08771929824562   & 53.333333333333336    & 5.252     \\ \hline
		Media         & 95.079565109958     & 51.333333333333336    & 5.509    
	\end{tabular}
	\caption{WDBC - SA}
	\label{WDBC-SA}
\end{table}
			\begin{table}[H]
	\centering
	\begin{tabular}{l|lll}
		Nombre        & Tasa de acierto(\%) & Tasa de reducción(\%) & Tiempo(s) \\ \hline
		Partición 1-1 & 72.22222222222223   & 56.666666666666664    & 15.106    \\
		Partición 1-2 & 65.0                & 53.333333333333336    & 8.536     \\
		Partición 2-1 & 76.66666666666667   & 47.77777777777778     & 12.607    \\
		Partición 2-2 & 67.77777777777777   & 54.44444444444444     & 12.984    \\
		Partición 3-1 & 71.66666666666667   & 46.666666666666664    & 13.953    \\
		Partición 3-2 & 69.44444444444444   & 48.888888888888886    & 12.894    \\
		Partición 4-1 & 73.88888888888889   & 50.0                  & 13.961    \\
		Partición 4-2 & 72.77777777777777   & 44.44444444444444     & 13.677    \\
		Partición 5-1 & 71.11111111111111   & 45.55555555555556     & 12.517    \\
		Partición 5-2 & 68.33333333333333   & 46.666666666666664    & 14.651    \\ \hline
		Media         & 70.8888888888889    & 49.44444444444445     & 13.0886  
	\end{tabular}
	\caption{Movement Libras - SA}
	\label{MLIB-SA}
\end{table}
			\begin{table}[H]
	\centering
	\caption{Arrhythmia - SA}
	\label{ARRH-SA}
	\begin{tabular}{l|lll}
		Nombre        & Tasa de acierto(\%) & Tasa de reducción(\%) & Tiempo(s)          \\ \hline
		Partición 1-1 & 61.6580310880829    & 48.92086330935252     & 123.033            \\
		Partición 1-2 & 60.62176165803109   & 47.84172661870504     & 98.449             \\
		Partición 2-1 & 56.476683937823836  & 47.48201438848921     & 102.757            \\
		Partición 2-2 & 67.87564766839378   & 50.0                  & 100.828            \\
		Partición 3-1 & 57.512953367875646  & 45.68345323741007     & 101.326            \\
		Partición 3-2 & 64.24870466321244   & 52.15827338129496     & 113.503            \\
		Partición 4-1 & 65.80310880829016   & 45.68345323741007     & 111.926            \\
		Partición 4-2 & 63.21243523316062   & 47.84172661870504     & 102.325            \\
		Partición 5-1 & 65.80310880829016   & 50.0                  & 114.916            \\
		Partición 5-2 & 57.512953367875646  & 46.76258992805755     & 102.337            \\ \hline
		Media         & 62.072538860103634  & 48.23741007194244     & 107.14000000000001
	\end{tabular}
\end{table}
		
		\subsection{Búsqueda Tabú (\textbf{TS})}
			\begin{table}[H]
	\centering
	\begin{tabular}{l|lll}
		Nombre        & Tasa de acierto(\%) & Tasa de reducción(\%) & Tiempo(s)         \\ \hline
		Partición 1-1 & 96.83098591549296   & 46.666666666666664    & 442.098           \\
		Partición 1-2 & 95.43859649122807   & 43.333333333333336    & 571.518           \\
		Partición 2-1 & 97.1830985915493    & 56.666666666666664    & 546.654           \\
		Partición 2-2 & 95.43859649122807   & 36.666666666666664    & 492.704           \\
		Partición 3-1 & 95.77464788732394   & 43.333333333333336    & 519.361           \\
		Partición 3-2 & 96.84210526315789   & 53.333333333333336    & 537.588           \\
		Partición 4-1 & 95.07042253521126   & 46.666666666666664    & 587.384           \\
		Partición 4-2 & 97.89473684210526   & 40.0                  & 560.582           \\
		Partición 5-1 & 97.1830985915493    & 53.333333333333336    & 498.821           \\
		Partición 5-2 & 95.78947368421052   & 23.333333333333332    & 603.365           \\ \hline
		Media         & 96.34457622930566   & 44.33333333333333     & 536.0074999999999
	\end{tabular}
	\caption{WDBC - TS}
	\label{WDBC-TS}
\end{table}
			\begin{table}[H]
	\centering
	\begin{tabular}{l|lll}
		Nombre        & Tasa de acierto(\%) & Tasa de reducción(\%) & Tiempo(s) \\ \hline
		Partición 1-1 & 70.0                & 56.666666666666664    & 530.359   \\
		Partición 1-2 & 71.66666666666667   & 44.44444444444444     & 781.64    \\
		Partición 2-1 & 65.0                & 60.0                  & 617.273   \\
		Partición 2-2 & 75.55555555555556   & 47.77777777777778     & 728.861   \\
		Partición 3-1 & 71.66666666666667   & 57.77777777777778     & 615.89    \\
		Partición 3-2 & 71.11111111111111   & 55.55555555555556     & 691.601   \\
		Partición 4-1 & 78.33333333333333   & 53.333333333333336    & 665.66    \\
		Partición 4-2 & 68.33333333333333   & 53.333333333333336    & 524.015   \\
		Partición 5-1 & 71.11111111111111   & 51.111111111111114    & 539.537   \\
		Partición 5-2 & 75.0                & 53.333333333333336    & 460.171   \\ \hline
		Media         & 71.77777777777777   & 53.333333333333336    & 615.5007 
	\end{tabular}
	\caption{Movement Libras - TS}
	\label{MLIB-TS}
\end{table}
			\begin{table}[]
	\centering
	\begin{tabular}{l|lll}
		Nombre        & Tasa de acierto(\%) & Tasa de reducción(\%) & Tiempo(s) \\ \hline
		Partición 1-1 & 63.21243523316062   & 45.32374100719424     & 2717.402  \\
		Partición 1-2 & 62.69430051813472   & 44.96402877697842     & 2324.424  \\
		Partición 2-1 & 61.6580310880829    & 52.15827338129496     & 2185.451  \\
		Partición 2-2 & 61.13989637305699   & 51.07913669064748     & 2238.612  \\
		Partición 3-1 & 61.6580310880829    & 44.96402877697842     & 2337.653  \\
		Partición 3-2 & 67.87564766839378   & 52.87769784172662     & 2008.115  \\
		Partición 4-1 & 63.21243523316062   & 57.194244604316545    & 1858.113  \\
		Partición 4-2 & 62.17616580310881   & 51.43884892086331     & 2296.42   \\
		Partición 5-1 & 62.17616580310881   & 51.798561151079134    & 2276.05   \\
		Partición 5-2 & 66.32124352331606   & 53.23741007194245     & 2233.166  \\ \hline
		Media         & 63.21243523316061   & 50.50359712230216     & 2247.5406
	\end{tabular}
	\caption{Arrhythmia - TS}
	\label{ARRH-TS}
\end{table}
			
		\subsection{Comparación de resultados}
			\begin{table}[H]
	\centering
	\begin{tabular}{l|lll|lll|lll}
			& 			& WDBC 		&			&	  Mov	& ement 	& Libras	&			& Arrhyt	& hmia		\\ \hline
			& \%\_clas	& \%\_red	& T			& \%\_clas	& \%\_red	& T			& \%\_clas	& \%\_red	& T			\\ \hline
		SFS	& 93.1814	& 88.6667 	& 1.8121	& 49.4444	& 94.3333	& 8.0742	& 66.4249	& 97.8058	& 128.5797 	\\ \hline
		LS	& 95.1854	& 48.3333	& 2.1519	& 69.7778	& 49.7778	& 8.8262	& 64.3005	& 48.3453	& 151.402  	\\ \hline
		SA	& 95.0796	& 51.3333	& 5.509		& 70.8889	& 49.4444	& 13.0886	& 62.0725	& 48.2374	& 107.1400	\\ \hline
		TS	& 96.3446	& 44.3333	& 536.0075	& 71.7778	& 53.3333	& 615.5007	& 63.2124	& 50.5036	& 2247.5406
		
	\end{tabular}
	\caption{Comparación de resultados}
	\label{Compare}
\end{table}
			
			Viendo los resultados de arriba, podemos concluir que:
			\begin{itemize}
				\item Los mayores índices de reducción se obtienen con el \textit{SFS}, obteniendo en
				todos los casos una tasa de reducción superior al 80\% y los demás, rondan el 50\%.
				Lo que nos hace pensar que puede ser que los demás tenga sobre-ajuste sobre los datos
				de entrenamiento.
				
				\item En el caso de \textit{WBDC}, todos los algoritmos tienen una alta tasa de acierto,
				quizás la mejor heurística en este caso, sería utilizar \textit{LS} ya que, a pesar
				de que \textit{TS} tienen un valor mayor, su coste en tiempo desproporcional lo descarta
				rápidamente.
				
				\item En el caso de \textit{Movement Libras}, optaría por tomar \textit{SA} por motivos
				similares al caso anterior: aunque no sea el que posee el mayor índice de acierto,
				\textit{TS} lo descartamos por excesivo tiempo de ejecución, tiene un gran índice de
				acierto, un índice de reducción aceptable y se ejecuta en un tiempo razonable.

				\item Sin duda alguna, para el caso de \textit{Arrhythmia}, el mejor algoritmo sería
				el de \textit{SFS}, ya que tiene el mejor índice de acierto(con un único valor muy
				bajo en las 10 iteraciones que provoca una bajada importante en la media), de reducción
				y un tiempo bastante bajo para la cantidad de características e instancias que tiene.
			\end{itemize}
	
	\input{Compilación.tex}
	\newpage
	
	\begin{thebibliography}{10}
	\expandafter\ifx\csname url\endcsname\relax
	  \def\url#1{\texttt{#1}}\fi
	\expandafter\ifx\csname urlprefix\endcsname\relax\def\urlprefix{URL }\fi
	\expandafter\ifx\csname href\endcsname\relax
	  \def\href#1#2{#2} \def\path#1{#1}\fi
	
	\bibitem{KNN}
	WEKA (The University of Waikato)\\
	  \url{http://weka.sourceforge.net/doc.dev/weka/classifiers/lazy/IBk.html}\\
	  \url{http://weka.sourceforge.net/doc.dev/weka/core/neighboursearch/NearestNeighbourSearch.html}
	  
  	\bibitem{ARFFReader}
  	WEKA (The University of Waikato)\\
	  \url{http://weka.sourceforge.net/doc.dev/weka/core/converters/ArffLoader.ArffReader.html}
	  
	\end{thebibliography}

\end{document}
