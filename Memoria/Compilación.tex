\section{Implementación y compilación del proyecto}
	Este proyecto se ha realizado en Java en el entorno de desarrollo NetBeans y se han utilizado
	los \textit{frameworks} de \textit{Weka} principalmente para el tratamiento previo del
	conjunto de instancias(normalización de los mismos), la separación de las instancias en
	instancias de Entrenamiento y de Valoración y para la llamada al $K-NN$ utilizada en la
	función de valoración.
	
	Para la compilación del proyecto, se recomienda el uso de la plataforma NetBeans por su
	sencillez a la hora de compilar, \textit{$"$debuggear$"$} y ejecutar el proyecto(aunque,
	como usa parámetros de entrada, en lo personal, prefiero ejecutarlo por terminal).
	
	No obstante, se puede realizar la compilación del proyecto con \textit{ant -f} <nombre del
	proyecto> y su ejecución mediante \textit{java -jar} <nombre del proyecto> <parámetros>.
	
	Los parámetros serían:
		\begin{itemize}
			\item Número de archivos en los que se le pedirá aplicar las heurísticas.
			\item Párametros de cada fichero, es decir:
				\begin{itemize}
					\item Dirección al fichero.
					\item Columna en la que se encuentra la característica de clase.
					\item Base del nombre del fichero de salida\footnote{Por cada fichero
					de entrada, se generan 4 de salida, uno por cada algoritmo}.
				\end{itemize}
		\end{itemize}
