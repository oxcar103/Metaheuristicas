%%
% Plantilla de Memoria
% Modificación de una plantilla de Latex de Nicolas Diaz para adaptarla 
% al castellano y a las necesidades de escribir informática y matemáticas.
%
% Editada por: Mario Román
%
% License:
% CC BY-NC-SA 3.0 (http://creativecommons.org/licenses/by-nc-sa/3.0/)
%%

%%%%%%%%%%%%%%%%%%%%%
% Thin Sectioned Essay
% LaTeX Template
% Version 1.0 (3/8/13)
%
% This template has been downloaded from:
% http://www.LaTeXTemplates.com
%
% Original Author:
% Nicolas Diaz (nsdiaz@uc.cl) with extensive modifications by:
% Vel (vel@latextemplates.com)
%
% License:
% CC BY-NC-SA 3.0 (http://creativecommons.org/licenses/by-nc-sa/3.0/)
%
%%%%%%%%%%%%%%%%%%%%%

%----------------------------------------------------------------------------------------
%	PAQUETES Y CONFIGURACIÓN DEL DOCUMENTO
%----------------------------------------------------------------------------------------

%% Configuración del papel.
% microtype: Tipografía.
% mathpazo: Usa la fuente Palatino.
\documentclass[a4paper, 11pt]{article}
\usepackage[protrusion=true,expansion=true]{microtype}
\usepackage{mathpazo}

% Indentación de párrafos para Palatino
\setlength{\parindent}{0pt}
  \parskip=8pt
\linespread{1.05} % Change line spacing here, Palatino benefits from a slight increase by default


%% Castellano.
% noquoting: Permite uso de comillas no españolas.
% lcroman: Permite la enumeración con numerales romanos en minúscula.
% fontenc: Usa la fuente completa para que pueda copiarse correctamente del pdf.
\usepackage[spanish,es-noquoting,es-lcroman]{babel}
\usepackage[utf8]{inputenc}
\usepackage[T1]{fontenc}
\selectlanguage{spanish}


%% Gráficos
\usepackage{graphicx} % Required for including pictures
\usepackage{wrapfig} % Allows in-line images
\usepackage[usenames,dvipsnames]{color} % Coloring code

% % Enlaces
\usepackage[hidelinks]{hyperref}

%% Matemáticas
\usepackage{amsmath}

% Para algoritmos
\usepackage{algorithm}
\usepackage{algorithmic}
\usepackage{amsthm}
\floatname{algorithm}{Algoritmo}
\renewcommand{\listalgorithmname}{Lista de algoritmos}
\renewcommand{\algorithmicrequire}{\textbf{Entrada:}}
\renewcommand{\algorithmicensure}{\textbf{Salida:}}
\renewcommand{\algorithmicend}{\textbf{fin}}
\renewcommand{\algorithmicif}{\textbf{si}}
\renewcommand{\algorithmicthen}{\textbf{entonces}}
\renewcommand{\algorithmicelse}{\textbf{en otro caso}}
\renewcommand{\algorithmicelsif}{\algorithmicelse,\ \algorithmicif}
\renewcommand{\algorithmicendif}{\algorithmicend\ \algorithmicif}
\renewcommand{\algorithmicfor}{\textbf{para }}
\renewcommand{\algorithmicforall}{\textbf{para cada}}
\renewcommand{\algorithmicdo}{\textbf{}}
\renewcommand{\algorithmicendfor}{\algorithmicend\ \algorithmicfor}
\renewcommand{\algorithmicwhile}{\textbf{mientras}}
\renewcommand{\algorithmicendwhile}{\algorithmicend\ \algorithmicwhile}
\renewcommand{\algorithmicloop}{\textbf{repetir}}
\renewcommand{\algorithmicendloop}{\algorithmicend\ \algorithmicloop}
\renewcommand{\algorithmicrepeat}{\textbf{repetir}}
\renewcommand{\algorithmicuntil}{\textbf{hasta que}}
\renewcommand{\algorithmicprint}{\textbf{imprimir}} 
\renewcommand{\algorithmicreturn}{\textbf{devolver}} 
\renewcommand{\algorithmictrue}{\textbf{true }} 
\renewcommand{\algorithmicfalse}{\textbf{false }} 
\renewcommand{\algorithmicand}{\textbf{y}}
\renewcommand{\algorithmicor}{\textbf{o}}


%% Bibliografía
\makeatletter
\renewcommand\@biblabel[1]{\textbf{#1.}} % Change the square brackets for each bibliography item from '[1]' to '1.'
\renewcommand{\@listI}{\itemsep=0pt} % Reduce the space between items in the itemize and enumerate environments and the bibliography



%----------------------------------------------------------------------------------------
%	TÍTULO
%----------------------------------------------------------------------------------------
% Configuraciones para el título.
% El título no debe editarse aquí.
\renewcommand{\maketitle}{
  \begin{flushright} % Center align
  {\LARGE\@title} % Increase the font size of the title
  
  \vspace{50pt} % Some vertical space between the title and author name
  
  {\large\@author} % Author name
  \\\@date % Date
  \vspace{40pt} % Some vertical space between the author block and abstract
  \end{flushright}
}

% Título
\title{\textbf{Metaheurísticas: Selección de Características}\\ % Title
P-1: Búsquedas por Trayectorias Simples} % Subtitle

\author{\textsc{Óscar Bermúdez Garrido\\
\href{http://www.github.com/oxcar103}{@oxcar103}} % Author
\\{\textit{Universidad de Granada}}} % Institution

\date{\today} % Date



%----------------------------------------------------------------------------------------
%	DOCUMENTO
%----------------------------------------------------------------------------------------

\begin{document}

\maketitle % Print the title section

% Resumen (Descomentar para usarlo)
\renewcommand{\abstractname}{Resumen} % Uncomment to change the name of the abstract to something else
%\begin{abstract}
% Resumen aquí
%\end{abstract}

% Palabras clave
%\hspace*{3,6mm}\textit{Keywords:} lorem , ipsum , dolor , sit amet , lectus % Keywords
%\vspace{30pt} % Some vertical space between the abstract and first section


% Índice
{\parskip=2pt
  \tableofcontents
}
\pagebreak

%% Inicio del documento
	\section{\textit{Sequential Forward Selection} (\textbf{SFS})}
	\begin{algorithm}[H]
		\begin{algorithmic}[1]
			\REQUIRE \ \\
	        	\texttt{monedas}, vector con los valores de monedas disponibles \\
	        	\texttt{precio}, cantidad de dinero a devolver\\\
	     	\STATE{Ordenar \texttt{monedas} de mayor a menor}
	     	%  He usado un vector aunque en el código final usábamos 
	     	%  un diccionario, si queréis lo cambio -David
	     	\STATE{Crear un vector \texttt{vuelta}}
			\FORALL{\texttt{moneda} en \texttt{monedas}}
				\STATE{\texttt{cantidad} = \texttt{precio} / \texttt{moneda}}
				\STATE{\texttt{precio} = \texttt{precio} $-$ \texttt{cantidad} $\times$ \texttt{moneda}}
				\STATE{Añadir [\texttt{moneda}, \texttt{cantidad}] a \texttt{vuelta}}
			\ENDFOR
			\RETURN{vuelta}
		\end{algorithmic}
	    \caption{Devolución en mínimo número de monedas}
	    \label{monedas}
	\end{algorithm}
	\section{Búsqueda Local (\textbf{LS})}
	    \begin{algorithm}[H]
		\begin{algorithmic}[1]
			\REQUIRE \ \\
	        	\texttt{grafo}, grafo inicial del que se calcula el árbol minimal \\\
			\STATE{\texttt{árbol=}$\emptyset$}
			\FORALL{\texttt{v} en \texttt{grafo.vértices}:}
			   \STATE{Crea componente conexa \texttt{set(v)}}
			\ENDFOR
			\FORALL{arista \texttt{(u,v)} en \texttt{grafo.aristas}:}
			\IF{\texttt{componente(u)} $\neq$ \texttt{componente(v)}}
			      \STATE{\texttt{árbol}.añadir(\texttt{(u,v)})}
			      \STATE{unir(\texttt{componente(u)},\texttt{componente(v)})}
			    \ENDIF
			\ENDFOR
			\RETURN{\texttt{árbol}}
		\end{algorithmic}
	      \caption{Algoritmo de Kruskal}
	      \label{kruskal}
	    \end{algorithm}
	\section{Enfriamiento Simulado (\textbf{SA})}
	\begin{algorithm}[H]
		\begin{algorithmic}[1]
			\REQUIRE \ \\
	        	\texttt{ciudades}, coordenadas de un conjunto de ciudades \\\
			\STATE{\texttt{grafo.vértices} $=$ \texttt{ciudades}}
			\STATE{\texttt{grafo.aristas}  $=$ $[\texttt{(u,v)} \quad | \quad \forall \texttt{(u,v)} \in Delaunay(\texttt{ciudades})]$}
			\RETURN{$Kruskal(\texttt{grafo})$}
		\end{algorithmic}
	      \caption{Red de comunicaciones}
	      \label{red}
	    \end{algorithm}
	    
	    \begin{algorithm}[H]
		\begin{algorithmic}[1]
			\REQUIRE \ \\
	        	\texttt{grafo}, grafo euclídeo generado por los vectores de características \\\
	        	\texttt{k}, número de componentes conexas en las que se quiere partir el grafo \\\
			\STATE{\texttt{componentes=}[[$v_1$],\ldots,[$v_{\#grafo.vertices}$]], diccionario de $\#grafo.vertices$ arrays, $v_i\in$\texttt{grafo.vertices}}
			\STATE{\texttt{edge\_list=sort(grafo.edges)} en función del peso de cada arista, de menor a mayor}
			\FORALL{\texttt{arista (u,v)} en \texttt{edge\_list}}
			  \IF{\texttt{componente[u]} $\neq$ \texttt{componente[v]}}
				\STATE{unir(\texttt{componente(u)},\texttt{componente(v)})}
			      \ENDIF
			  \IF{\texttt{\#componentes == k}}
			    \RETURN{\texttt{componentes}}
			  \ENDIF
			\ENDFOR
		\end{algorithmic}
	      \caption{Algoritmo de Kruskal generalizado}
	      \label{kruskal-generalizado}
	    \end{algorithm}
	
	\section{Búsqueda Tabú (\textbf{TS})}
	\begin{algorithm}[H]
				\begin{algorithmic}[1]
					\REQUIRE \ \\
						\texttt{m}, matriz de costes, y $n$ orden de la matriz\\
					\STATE{\texttt{Crea una Asignacion vacía}}
					\STATE{\texttt{Crea un vector trabajosUsados de tamaño $n$ y lo rellena a false}}
					\FORALL{Trabajador (i)}
					  \STATE{\texttt{Minimo = -1}}
					  \STATE{\texttt{posTrabajo = 0}}
			          \FORALL{Trabajo (j)}
			            \IF{!trabajosUsados[j]}
			            	\IF{Minimo < 0 || m[i][j] < Minimo}
			            		\STATE{\texttt{posTrabajo = j}}
			         			\STATE{\texttt{Minimo = m[i][j]}}
			            	\ENDIF
			            \ENDIF
			          \ENDFOR
			          \STATE{\texttt{ Asignacion.push(pair(i,posTrabajo))} }
			          \STATE{\texttt{trabajosUsados[posTrabajo = true]} }
			        \ENDFOR  	
			        
					\RETURN{\texttt{Asignacion}}
				\end{algorithmic}
			    \caption{Asignación de trabajos por trabajador}
			    \label{Asignación Trabajos}
			\end{algorithm}
	
	
	
	
	
	
			Para recoger los datos a través de la entrada, he implementado una clase
			extra basándome en \cite{ARFFReader}.
			Para generar los archivos \textit{.csv}, he utilizado una clase implementada
			por un compañero y algo adaptada a mi estilo de programación.
			El algoritmo KNN está tomado de \cite{KNN}
	
	
	
			
		
	\begin{algorithm}[H]
				\begin{algorithmic}[1]
					\REQUIRE \ \\
						\texttt{m}, matriz de costes, y $n$ orden de la matriz\\
					\STATE{\texttt{Crea una Asignacion vacía}}
					\STATE{\texttt{Crea dos vectores: trabajosUsados y trabajadoresUsados, de tamaño $n$ y los rellena a false}}
					\FORALL{Fila}
					  \STATE{\texttt{Minimo = -1}}
					  \STATE{\texttt{posTrabajo = 0\\ posTrabajador = 0}}
				      \FORALL{Trabajador (j)}
				        \IF{!trabajadoresUsados[j]}
				          \FORALL{Trabajo (k)}
				            \IF{!trabajosUsados[j]}
				           	  \IF{Minimo < 0 || m[j][k] < Minimo}
				           	    \STATE{\texttt{posTrabajo = k \\ posTrabajador = j\\
				           	    Minimo = m[j][k]}}
				           	  \ENDIF
				           	\ENDIF
				          \ENDFOR
				        \ENDIF
				      \ENDFOR
				      \STATE{\texttt{Asignacion.push(pair(posTrabajador,posTrabajo))}}
				      \STATE{\texttt{trabajosUsados[posTrabajo] = true \\ trabajadoresUsados[posTrabajador] = true}}
				    \ENDFOR  	
					        
					\RETURN{\texttt{Asignacion}}
				\end{algorithmic}
				   \caption{Asignación global de trabajos}
				   \label{Asignación Global}
			\end{algorithm}
			
	\begin{algorithm}[H]
		\begin{algorithmic}[1]
			\REQUIRE \ \\
	        	\texttt{clases}, array de $n$ objetos \texttt{Clase}, que almacena
					 un tiempo de inicialización y otro de 
					 finalización\\
	     	\STATE{\texttt{adyadcentes}=$(\delta_{i,j})_{nxn}, i,j\in \{0\ldots \#clases - 1\}$}
	     	\STATE{\texttt{sin\_colorear}= $\{0\ldots\#clases-1\}$}
	     	\STATE{\texttt{color}=$[]$}
	     	\STATE{\texttt{indice}=0}
		  \WHILE{\texttt{sin\_colorear}$\neq \emptyset$}
		    \STATE{\texttt{color[indice]=[sin\_colorear.pop\_front]}}
		    \FORALL{\texttt{v}$\in$\texttt{sin\_colorear}}
		      \IF{\textbf{no} \texttt{adyacentes[color[indice][0]][v]}}
			\STATE{\texttt{color[indice].push(v)}}
			\STATE{\texttt{sin\_colorear.delete(v)}}
		      \ENDIF
		    \ENDFOR
		    \STATE{\texttt{indice++}}
		  \ENDWHILE
			\RETURN{$\#$\texttt{color}}
		\end{algorithmic}
	    \caption{Asignación de aulas}
	    \label{aulas}
	\end{algorithm}
	\begin{algorithm}[H]
		\begin{algorithmic}[1]
			\REQUIRE \ \\
	        	\texttt{p}, array de $n$ peticiones\\
	        	\texttt{k}, tamaño de la caché\\ \
	     	\STATE{\texttt{fallos} = 0}\\
	     	\STATE{\texttt{cache} = eliminaRepetidos(\texttt{p}).\texttt{slice}(0, k-1)}
		  \FORALL{\texttt{petición} $\in$ \texttt{p}}
		   \IF{\texttt{petición $\notin$ cache}}
			\STATE{\texttt{a\_eliminar} = Elemento de \texttt{p} en \texttt{cache} que más tardará en volver a ser necesitado}
			\STATE{\texttt{cache[cache.find(a\_eliminar)]} = \texttt{peticion}}
			\STATE{\texttt{fallos++}}
		   \ENDIF
		  \ENDFOR
		  
		  \RETURN{fallos}
	
		\end{algorithmic}
	    \caption{Optimización de memoria caché}
	    \label{memory}
	\end{algorithm}
	
	\begin{thebibliography}{7}
	
	\begin{thebibliography}{10}
	\expandafter\ifx\csname url\endcsname\relax
	  \def\url#1{\texttt{#1}}\fi
	\expandafter\ifx\csname urlprefix\endcsname\relax\def\urlprefix{URL }\fi
	\expandafter\ifx\csname href\endcsname\relax
	  \def\href#1#2{#2} \def\path#1{#1}\fi
	
	\bibitem{ARFFReader}
	WEKA (The University of Waikato)\\
	  \url{http://weka.sourceforge.net/doc.dev/weka/core/converters/ArffLoader.ArffReader.html}
	
	\bibitem{KNN}
	WEKA (The University of Waikato)\\
	  \url{http://weka.sourceforge.net/doc.dev/weka/classifiers/lazy/IBk.html}\\
	  \url{http://weka.sourceforge.net/doc.dev/weka/core/neighboursearch/NearestNeighbourSearch.html}
	
%	\bibitem{KNN}
%	Gigas, the cloud hosting company.\\
%	  \url{https://gigas.com/cloud-vps}

\end{document}
