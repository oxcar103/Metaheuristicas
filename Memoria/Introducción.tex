\section{Descripción del problema}
	Disponemos de una serie de muestras $w_1, w_2, \dots, w_r$ con un conjunto de caracterísitcas
	asociadas $(x_1(w_i), x_2, \dots, x_n(w_i))$ clasificadas en las distintas clases $\{C_1, C_2,
	\dots, C_M\}$.
	
	Queremos obtener una máscara que nos permita clasificar estas y futuras muestras a partir
	de sus atributos o características de forma automática, con la mayor tasa de acierto posible
	así como con la mayor tasa de reducción que se pueda, es decir, usando el menor número de
	características posible.
	
	Todo ello, en un tiempo razonable, pues a pesar de que podríamos encontrar el óptimo probando
	para todas las máscaras, debido a la dimensionalidad de algunos problemas, dicho coste en
	tiempo sería muy elevado.
	
	Para comprobar su funcionamiento, se suele recurrir a partir el conjunto de instancias dadas
	en dos particiones no necesariamente de igual cardinalidad. Sobre una de ellas se ejecutará
	el \textbf{Entrenamiento} donde se creará la máscara, y sobre la otra se llevará a cabo la
	\textbf{Validación} donde se usará la máscara creada para clasificar el conjunto y ver qué
	tasa de acierto tiene nuestro algoritmo. Por seguridad del buen funcionamiento del algoritmo,
	se realizan varias particiones entrenamiento-validación.
	
	Este problema tiene principalmente dos grandes adversidades que superar:
	\begin{itemize}
		\item La $"$maldición de la dimensión$"$(o efecto Hughes), que son el conjunto de fenómenos
		ocasionados a raíz del estudio y clasificación de datos de múltiple dimensionalidad.
		
		Estos fenómenos se deben porque a medida que la dimensión aumenta, el tamaño del espacio
		asociado se incrementa exponencialmente, complicando de este modo la predicción correcta
		del comportamiento de los individuos que lo componen.
		
		\item El $"$sobre-ajuste$"$ a los datos de entrenamiento, es decir, la creación de una
		máscara que maximiza la tasa de aciertos sobre los datos de entrenamiento pero dificultan
		la generalización sobre el problema, obteniendo malos resultados en la parte de validación
		y, por tanto, posiblemente en la clasificación de las futuras muestras.
	\end{itemize}
