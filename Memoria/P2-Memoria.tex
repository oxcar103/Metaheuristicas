%%
% Plantilla de Memoria
% Modificación de una plantilla de Latex de Nicolas Diaz para adaptarla 
% al castellano y a las necesidades de escribir informática y matemáticas.
%
% Editada por: Mario Román
%
% License:
% CC BY-NC-SA 3.0 (http://creativecommons.org/licenses/by-nc-sa/3.0/)
%%

%%%%%%%%%%%%%%%%%%%%%
% Thin Sectioned Essay
% LaTeX Template
% Version 1.0 (3/8/13)
%
% This template has been downloaded from:
% http://www.LaTeXTemplates.com
%
% Original Author:
% Nicolas Diaz (nsdiaz@uc.cl) with extensive modifications by:
% Vel (vel@latextemplates.com)
%
% License:
% CC BY-NC-SA 3.0 (http://creativecommons.org/licenses/by-nc-sa/3.0/)
%
%%%%%%%%%%%%%%%%%%%%%

%----------------------------------------------------------------------------------------
%	PAQUETES Y CONFIGURACIÓN DEL DOCUMENTO
%----------------------------------------------------------------------------------------

%% Configuración del papel.
% microtype: Tipografía.
% mathpazo: Usa la fuente Palatino.
\documentclass[a4paper, 11pt]{article}
\usepackage[protrusion=true,expansion=true]{microtype}
\usepackage{mathpazo}

% Indentación de párrafos para Palatino
\setlength{\parindent}{0pt}
  \parskip=8pt
\linespread{1.05} % Change line spacing here, Palatino benefits from a slight increase by default


%% Castellano.
% noquoting: Permite uso de comillas no españolas.
% lcroman: Permite la enumeración con numerales romanos en minúscula.
% fontenc: Usa la fuente completa para que pueda copiarse correctamente del pdf.
\usepackage[spanish,es-noquoting,es-lcroman]{babel}
\usepackage[utf8]{inputenc}
\usepackage[T1]{fontenc}
\selectlanguage{spanish}


%% Gráficos
\usepackage{graphicx} % Required for including pictures
\usepackage{wrapfig} % Allows in-line images
\usepackage[usenames,dvipsnames]{color} % Coloring code

% % Enlaces
\usepackage[hidelinks]{hyperref}

%% Matemáticas
\usepackage{amsmath}

% Para algoritmos
\usepackage{algorithm}
\usepackage{algorithmic}
\usepackage{amsthm}
\floatname{algorithm}{Algoritmo}
\renewcommand{\listalgorithmname}{Lista de algoritmos}
\renewcommand{\algorithmicrequire}{\textbf{Entrada:}}
\renewcommand{\algorithmicensure}{\textbf{Salida:}}
\renewcommand{\algorithmicend}{\textbf{fin}}
\renewcommand{\algorithmicif}{\textbf{si}}
\renewcommand{\algorithmicthen}{\textbf{entonces}}
\renewcommand{\algorithmicelse}{\textbf{en otro caso}}
\renewcommand{\algorithmicelsif}{\algorithmicelse,\ \algorithmicif}
\renewcommand{\algorithmicendif}{\algorithmicend\ \algorithmicif}
\renewcommand{\algorithmicfor}{\textbf{para }}
\renewcommand{\algorithmicforall}{\textbf{para cada}}
\renewcommand{\algorithmicdo}{\textbf{}}
\renewcommand{\algorithmicendfor}{\algorithmicend\ \algorithmicfor}
\renewcommand{\algorithmicwhile}{\textbf{mientras}}
\renewcommand{\algorithmicendwhile}{\algorithmicend\ \algorithmicwhile}
\renewcommand{\algorithmicloop}{\textbf{repetir}}
\renewcommand{\algorithmicendloop}{\algorithmicend\ \algorithmicloop}
\renewcommand{\algorithmicrepeat}{\textbf{repetir}}
\renewcommand{\algorithmicuntil}{\textbf{hasta que}}
\renewcommand{\algorithmicprint}{\textbf{imprimir}} 
\renewcommand{\algorithmicreturn}{\textbf{devolver}} 
\renewcommand{\algorithmictrue}{\textbf{true }} 
\renewcommand{\algorithmicfalse}{\textbf{false }} 
\renewcommand{\algorithmicand}{\textbf{y}}
\renewcommand{\algorithmicor}{\textbf{o}}


%% Bibliografía
\makeatletter
\renewcommand\@biblabel[1]{\textbf{#1.}} % Change the square brackets for each bibliography item from '[1]' to '1.'
\renewcommand{\@listI}{\itemsep=0pt} % Reduce the space between items in the itemize and enumerate environments and the bibliography


%----------------------------------------------------------------------------------------
%	TÍTULO
%----------------------------------------------------------------------------------------
% Configuraciones para el título.
% El título no debe editarse aquí.
\renewcommand{\maketitle}{
  \begin{flushright} % Center align
  {\LARGE\@title} % Increase the font size of the title
  
  \vspace{50pt} % Some vertical space between the title and author name
  
  {\large\@author} % Author name
  \\\@date % Date
  \vspace{40pt} % Some vertical space between the author block and abstract
  \end{flushright}
}

% Título
\title{\textbf{Metaheurísticas: Selección de Características}\\ % Title
P-2: Búsquedas por Trayectorias Múltiples} % Subtitle

\author{\textsc{Óscar Bermúdez Garrido\\
\href{http://www.github.com/oxcar103}{@oxcar103}} % Author
\\{\textit{Universidad de Granada}}} % Institution

\date{\today} % Date


%----------------------------------------------------------------------------------------
%	DOCUMENTO
%----------------------------------------------------------------------------------------

\begin{document}

\maketitle % Print the title section

% Resumen (Descomentar para usarlo)
\renewcommand{\abstractname}{Resumen} % Uncomment to change the name of the abstract to something else
%\begin{abstract}
% Resumen aquí
%\end{abstract}

% Palabras clave
%\hspace*{3,6mm}\textit{Keywords:} lorem , ipsum , dolor , sit amet , lectus % Keywords
%\vspace{30pt} % Some vertical space between the abstract and first section


% Índice
{\parskip=2pt
  \tableofcontents
}
\pagebreak

%% Inicio del documento
	\input{Introducción.tex}
	\input{CaracterísticasComunes.tex}
	
	\section{Heurísticas implementadas}
		\subsection{Búsqueda Multiarranque Básica (\textbf{BMBS})}
			La dinámica de trabajo de esta técnica de búsqueda consiste en:
			\begin{enumerate}
				\item Generar un número $n$ de soluciones iniciales aleatorias.
				\item Optimizarlas usando búsqueda local\footnote{Como necesitamos optimizar mediante
				búsqueda local, haremos que herede de la clase \textit{LocalSearch} para facilitar
				la implementación.} sobre cada una de ellas.
				\item Quedarnos con la mejor solución obtenida tras dicha mejora.
			\end{enumerate}
			
			Para nuestro caso particular, generaremos un total de 25 soluciones iniciales utilizando
			el siguiente método:
			
			\begin{algorithm}[H]
				\begin{algorithmic}[1]
				\REQUIRE \ \\
						 \
				\FOR{\texttt{i} $<$ \texttt{Número de características}}
					\STATE{\texttt{rnd} = Genera valor aleatorio entre 0 y 1}
					\IF{\texttt{rnd} == 1}
						\STATE{Flip(i)}
					\ENDIF
				\ENDFOR
				\end{algorithmic}
			\caption{Solución Aleatoria(\textit{RandomSolution})}
			\label{RandomSolution}
			\end{algorithm}
			
			El algoritmo de su función de entrenamiento sería tan simple como \footnote{Nótese que la
			función \textbf{Optimizar} utilizada se traduce en código como \textit{super.train()} al
			heredar de la clase \textit{LocalSearch}}:
			
			\begin{algorithm}[H]
				\begin{algorithmic}[1]
				\REQUIRE \ \\
						 \
				\STATE{\texttt{i} = 0}
				\STATE{\texttt{Solución\_actual} = RandomSolution()}
				\STATE{\texttt{Solución\_actual}.Optimizar()}
				\STATE{\texttt{Evaluación\_actual} = \texttt{Evaluate()}}
				\STATE{\texttt{Mejor\_Solución} = \texttt{Solución\_actual}}
				\STATE{\texttt{Mejor\_Evaluación} = \texttt{Evaluación\_actual}}
				\STATE{\texttt{i++}}
				\FOR{\texttt{i} $<$ \texttt{Número de Soluciones Iniciales}}
					\STATE{\texttt{Solución\_actual} = RandomSolution()}
					\STATE{\texttt{Solución\_actual}.Optimizar()}
					\STATE{\texttt{Evaluación\_actual} = \texttt{Evaluate()}}
					\IF{\texttt{Evaluación\_actual} > \texttt{Mejor\_Evaluación}}
						\STATE{\texttt{Mejor\_Solución} = \texttt{Solución\_actual}}
						\STATE{\texttt{Mejor\_Evaluación} = \texttt{Evaluación\_actual}}
					\ENDIF
				\ENDFOR
				\STATE{\texttt{Solución\_actual} = \texttt{Mejor\_Solución}}
			\end{algorithmic}
		\caption{Búsqueda Multiarranque Básica}
		\label{BMBS}
		\end{algorithm}

		\subsection{\textit{Greedy Randomized Adaptive Search Procedure} (\textbf{GRASP})}
			El número de soluciones iniciales que tomaremos será 25.
			
			Utilizaremos el valor $0.3$ de la variable para la variable $\alpha$ que representa la
			tolerancia de calidad para la construcción de la lista restringida de candidatos de entre
			la que sacaremos probabilísticamente el índice que añadiremos
			\begin{algorithm}[H]
				\begin{algorithmic}[1]
					\REQUIRE \ \\
						\texttt{Mejor\_Evaluación}, mejor evaluación \\
						\texttt{Peor\_Evaluación}, peor evaluación \\ \
						
				\STATE{$\texttt{Umbral} = \texttt{Mejor\_Evaluación} - \alpha \cdot
					(\texttt{Mejor\_Evaluación} - \texttt{Peor\_Evaluación})$}
				\RETURN{\texttt{Umbral}}
				\end{algorithmic}
			\caption{GRASP - Umbral(\textit{Treshold})}
			\label{Treshold}
			\end{algorithm}
			
			El método de inicialización de soluciones es muy similar a la construcción de soluciones por
			\textbf{SFS}, de hecho, podemos considerar que es un caso particular con $\alpha = 0$ para
			nuestro algoritmo:

			\begin{algorithm}[H]
				\begin{algorithmic}[1]
					\REQUIRE \ \\
							 \
					\STATE{\texttt{Hay\_Mejora} = \TRUE}
					\WHILE{\texttt{Hay\_Mejora} \AND \texttt{Número de características seleccionadas} $<$
							\texttt{Número de características}}
						\STATE{Vaciamos \texttt{Lista\_Aspirantes}}
						\STATE{\texttt{Mejor\_Evaluación} = 0}
						\STATE{\texttt{Peor\_Evaluación} = $\infty$}
						\FOR{\texttt{i} $<$ \texttt{Número de características}}
							\IF{\texttt{Característica$_i$} no está seleccionada ni es de clase}
								\STATE{\texttt{Evaluación\_Actual} = Evaluate(i)}
								\STATE{Guardamos \texttt{Evaluación\_Actual} en \texttt{Lista\_Aspirantes}}
								\IF{\texttt{Evaluación\_Actual} $<$ \texttt{Peor\_Evaluación}}
									\STATE{\texttt{Peor\_Evaluación} = \texttt{Evaluación\_Actual}}
								\ENDIF
								\IF{\texttt{Evaluación\_Actual} $>$ \texttt{Mejor\_Evaluación}}
									\STATE{\texttt{Mejor\_Evaluación} = \texttt{Evaluación\_Actual}}
								\ENDIF
							\ENDIF
						\ENDFOR
						\STATE{\texttt{Umbral} = Treshold(\texttt{Peor\_Evaluación}, \texttt{Mejor\_Evaluación})}
						\FORALL{\texttt{Evaluación} en \texttt{Lista\_Aspirantes}}
							\IF{\texttt{Evaluación} $<$ \texttt{Umbral}}
								\STATE{Eliminar \texttt{Evaluación} de \texttt{Lista\_Aspirantes}}
							\ENDIF
						\ENDFOR
						\STATE{\texttt{rnd} = Genera valor aleatorio entre 0 y \texttt{Lista\_Aspirantes}.size()-1}
						\IF{\texttt{Lista\_Aspirantes[rnd]} mejora \texttt{Solución\_Actual}}
							\STATE{\texttt{Flip(índice de la característica mejorada por \texttt{rnd})}}
							\STATE{Guardamos el valor de \texttt{Solución\_Actual}}
						\ELSE
							\STATE{\texttt{Hay\_Mejora} = \FALSE}
						\ENDIF
					\ENDWHILE
				\end{algorithmic}
			\caption{GRASP - Solución Greedy Probabilista(\textit{NewGreedySolution})}
			\label{GreedySolution}
			\end{algorithm}
		\subsection{Búsqueda Local Reiterada (\textbf{ILS})}
	\section{Resultados}
		\subsection{Búsqueda Multiarranque Básica (\textbf{BMBS})}
			\begin{table}[H]
	\centering
	\begin{tabular}{l|lll}
		Nombre        & Tasa de acierto(\%) & Tasa de reducción(\%) & Tiempo(s)          \\ \hline
		Partición 1-1 & 96.12676056338029   & -430.0                & 32.558             \\
		Partición 1-2 & 92.63157894736842   & -50.0                 & 26.312             \\
		Partición 2-1 & 97.1830985915493    & 33.333333333333336    & 32.857             \\
		Partición 2-2 & 94.03508771929825   & -243.33333333333334   & 32.146             \\
		Partición 3-1 & 96.47887323943662   & -750.0                & 32.604             \\
		Partición 3-2 & 94.73684210526316   & -186.66666666666666   & 24.396             \\
		Partición 4-1 & 96.12676056338029   & 6.666666666666667     & 32.473             \\
		Partición 4-2 & 92.98245614035088   & -1053.3333333333333   & 30.755             \\
		Partición 5-1 & 96.47887323943662   & -376.6666666666667    & 33.917             \\
		Partición 5-2 & 93.6842105263158    & -210.0                & 23.389             \\ \hline
		Media         & 95.04645416357798   & -325.99999999999994   & 30.140700000000002
	\end{tabular}
	\caption{WDBC - BMBS}
	\label{WDBC-BMBS}
\end{table}
			\begin{table}[H]
	\centering
	\begin{tabular}{l|lll}
		Nombre        & Tasa de acierto(\%) & Tasa de reducción(\%) & Tiempo(s)          \\ \hline
		Partición 1-1 & 67.77777777777777   & 46.666666666666664    & 164.808            \\
		Partición 1-2 & 72.22222222222223   & 40.0                  & 191.95             \\
		Partición 2-1 & 75.0                & 48.888888888888886    & 234.277            \\
		Partición 2-2 & 63.333333333333336  & 46.666666666666664    & 198.46             \\
		Partición 3-1 & 77.77777777777777   & 53.333333333333336    & 210.994            \\
		Partición 3-2 & 63.888888888888886  & 56.666666666666664    & 223.015            \\
		Partición 4-1 & 66.66666666666667   & 52.22222222222222     & 219.249            \\
		Partición 4-2 & 64.44444444444444   & 55.55555555555556     & 214.729            \\
		Partición 5-1 & 71.11111111111111   & 54.44444444444444     & 212.195            \\
		Partición 5-2 & 65.55555555555556   & 50.0                  & 143.887            \\ \hline
		Media         & 68.77777777777777   & 50.44444444444444     & 201.35639999999998
	\end{tabular}
	\caption{Movement Libras - BMBS}
	\label{MLIB-BMBS}
\end{table}
			\begin{table}[H]
	\centering
	\begin{tabular}{l|lll}
		Nombre        & Tasa de acierto(\%) & Tasa de reducción(\%) & Tiempo(s)          \\ \hline
		Partición 1-1 & 58.031088082901555  & -352.51798561151077   & 1463.033           \\
		Partición 1-2 & 70.98445595854922   & -720.863309352518     & 1493.901           \\
		Partición 2-1 & 69.94818652849742   & -306.47482014388487   & 1460.158           \\
		Partición 2-2 & 61.13989637305699   & -582.3741007194244    & 1504.323           \\
		Partición 3-1 & 66.32124352331606   & -239.92805755395685   & 1480.615           \\
		Partición 3-2 & 67.35751295336787   & -103.59712230215827   & 1438.002           \\
		Partición 4-1 & 66.83937823834196   & -568.3453237410072    & 1510.381           \\
		Partición 4-2 & 59.58549222797927   & -248.20143884892087   & 1451.753           \\
		Partición 5-1 & 68.39378238341969   & -506.47482014388487   & 1478.733           \\
		Partición 5-2 & 64.76683937823834   & -142.80575539568346   & 1439.837           \\ \hline
		Media         & 65.33678756476684   & -377.1582733812949    & 1472.0736000000002
	\end{tabular}
	\caption{Arrhythmia - BMBS}
	\label{ARRH-BMBS}
\end{table}
		\subsection{\textit{Greedy Randomized Adaptive Search Procedure} (\textbf{GRASP})}
			\begin{table}[H]
	\centering
	\begin{tabular}{l|lll}
		Nombre        & Tasa de acierto(\%) & Tasa de reducción(\%) & Tiempo(s) \\ \hline
		Partición 1-1 & 94.36619718309859   & -566.6666666666666    & 27.247    \\
		Partición 1-2 & 94.73684210526316   & -263.3333333333333    & 24.748    \\
		Partición 2-1 & 94.36619718309859   & -386.6666666666667    & 23.74     \\
		Partición 2-2 & 93.6842105263158    & -343.3333333333333    & 26.132    \\
		Partición 3-1 & 96.47887323943662   & -26.666666666666668   & 24.218    \\
		Partición 3-2 & 96.84210526315789   & -100.0                & 26.572    \\
		Partición 4-1 & 96.83098591549296   & -130.0                & 30.627    \\
		Partición 4-2 & 92.63157894736842   & -606.6666666666666    & 26.602    \\
		Partición 5-1 & 97.53521126760563   & -170.0                & 27.297    \\
		Partición 5-2 & 94.73684210526316   & 6.666666666666667     & 26.75     \\ \hline
		Media         & 95.22090437361008   & -258.6666666666667    & 26.3933  
	\end{tabular}
	\caption{WDBC - GRASP}
	\label{WDBC-GRASP}
\end{table}
			\begin{table}[H]
	\centering
	\begin{tabular}{l|lll}
		Nombre        & Tasa de acierto(\%) & Tasa de reducción(\%) & Tiempo(s)          \\ \hline
		Partición 1-1 & 71.66666666666667   & 82.22222222222223     & 131.396            \\
		Partición 1-2 & 68.88888888888889   & 83.33333333333333     & 177.039            \\
		Partición 2-1 & 71.11111111111111   & 80.0                  & 200.24             \\
		Partición 2-2 & 68.88888888888889   & 74.44444444444444     & 166.653            \\
		Partición 3-1 & 77.22222222222223   & 84.44444444444444     & 165.951            \\
		Partición 3-2 & 70.0                & 83.33333333333333     & 203.596            \\
		Partición 4-1 & 71.11111111111111   & 82.22222222222223     & 203.226            \\
		Partición 4-2 & 71.11111111111111   & 84.44444444444444     & 203.555            \\
		Partición 5-1 & 62.22222222222222   & 84.44444444444444     & 185.319            \\
		Partición 5-2 & 65.55555555555556   & 83.33333333333333     & 166.636            \\ \hline
		Media         & 69.77777777777777   & 82.22222222222223     & 180.36110000000002
	\end{tabular}
	\caption{Movement Libras - GRASP}
	\label{MLIB-GRASP}
\end{table}
			\begin{table}[H]
	\centering
	\begin{tabular}{l|lll}
		Nombre        & Tasa de acierto(\%) & Tasa de reducción(\%) & Tiempo(s) \\ \hline
		Partición 1-1 & 68.9119170984456    & 19.424460431654676    & 569.781   \\
		Partición 1-2 & 72.02072538860104   & 16.18705035971223     & 523.682   \\
		Partición 2-1 & 67.35751295336787   & -1.4388489208633093   & 561.97    \\
		Partición 2-2 & 70.98445595854922   & 19.06474820143885     & 667.75    \\
		Partición 3-1 & 60.10362694300518   & 9.712230215827338     & 668.8     \\
		Partición 3-2 & 73.05699481865285   & 38.84892086330935     & 546.068   \\
		Partición 4-1 & 68.39378238341969   & 43.52517985611511     & 506.674   \\
		Partición 4-2 & 70.98445595854922   & 46.0431654676259      & 519.151   \\
		Partición 5-1 & 68.9119170984456    & 26.258992805755394    & 503.284   \\
		Partición 5-2 & 74.61139896373057   & 46.76258992805755     & 541.513   \\ \hline
		Media         & 69.53367875647669   & 26.438848920863308    & 560.8673
	\end{tabular}
	\caption{Arrhythmia - GRASP}
	\label{ARRH-GRASP}
\end{table}
		\subsection{Búsqueda Local Reiterada (\textbf{ILS})}
			\begin{table}[H]
	\centering
	\begin{tabular}{l|lll}
		Nombre        & Tasa de acierto(\%) & Tasa de reducción(\%) & Tiempo(s)          \\ \hline
		Partición 1-1 & 95.4225352112676    & 46.666666666666664    & 26.633             \\
		Partición 1-2 & 96.49122807017544   & 30.0                  & 29.346             \\
		Partición 2-1 & 95.77464788732394   & 60.0                  & 22.948             \\
		Partición 2-2 & 95.43859649122807   & 30.0                  & 32.588             \\
		Partición 3-1 & 96.47887323943662   & 43.333333333333336    & 27.878             \\
		Partición 3-2 & 97.89473684210526   & 46.666666666666664    & 27.192             \\
		Partición 4-1 & 96.47887323943662   & 26.666666666666668    & 29.217             \\
		Partición 4-2 & 94.73684210526316   & 50.0                  & 25.221             \\
		Partición 5-1 & 96.83098591549296   & 43.333333333333336    & 24.43              \\
		Partición 5-2 & 94.03508771929825   & 66.66666666666667     & 21.995             \\ \hline
		Media         & 95.95824067210279   & 44.333333333333336    & 26.744799999999998
	\end{tabular}
	\caption{WDBC - ILS}
	\label{WDBC-ILS}
\end{table}
			\begin{table}[H]
	\centering
	\begin{tabular}{l|lll}
		Nombre        & Tasa de acierto(\%) & Tasa de reducción(\%) & Tiempo(s) \\ \hline
		Partición 1-1 & 71.66666666666667   & 52.22222222222222     & 117.308   \\
		Partición 1-2 & 65.0                & 60.0                  & 103.43    \\
		Partición 2-1 & 71.66666666666667   & 47.77777777777778     & 107.089   \\
		Partición 2-2 & 74.44444444444444   & 52.22222222222222     & 113.007   \\
		Partición 3-1 & 71.11111111111111   & 51.111111111111114    & 116.439   \\
		Partición 3-2 & 71.66666666666667   & 55.55555555555556     & 110.307   \\
		Partición 4-1 & 73.88888888888889   & 52.22222222222222     & 109.99    \\
		Partición 4-2 & 67.22222222222223   & 47.77777777777778     & 130.149   \\
		Partición 5-1 & 65.55555555555556   & 51.111111111111114    & 103.269   \\
		Partición 5-2 & 71.66666666666667   & 62.22222222222222     & 114.676   \\ \hline
		Media         & 70.38888888888889   & 53.222222222222214    & 112.5664
	\end{tabular}
	\caption{Movement Libras - ILS}
	\label{MLIB-ILS}
\end{table}
			\begin{table}[H]
	\centering
	\begin{tabular}{l|lll}
		Nombre        & Tasa de acierto(\%) & Tasa de reducción(\%) & Tiempo(s)         \\ \hline
		Partición 1-1 & 65.80310880829016   & 50.0                  & 875.551           \\
		Partición 1-2 & 61.6580310880829    & 49.64028776978417     & 871.76            \\
		Partición 2-1 & 62.17616580310881   & 54.31654676258993     & 887.604           \\
		Partición 2-2 & 64.24870466321244   & 51.798561151079134    & 1010.671          \\
		Partición 3-1 & 67.87564766839378   & 56.47482014388489     & 1103.306          \\
		Partición 3-2 & 58.54922279792746   & 45.32374100719424     & 882.833           \\
		Partición 4-1 & 59.58549222797927   & 52.87769784172662     & 1164.788          \\
		Partición 4-2 & 68.39378238341969   & 48.56115107913669     & 1044.289          \\
		Partición 5-1 & 70.46632124352331   & 53.9568345323741      & 942.765           \\
		Partición 5-2 & 61.13989637305699   & 52.51798561151079     & 1017.056          \\ \hline
		Media         & 63.98963730569949   & 51.546762589928065    & 980.0622999999999
	\end{tabular}
	\caption{Arrhythmia - ILS}
	\label{ARRH-ILS}
\end{table}
			
		\subsection{Comparación de resultados}
	
	\input{Compilación.tex}
	\newpage
	
	\begin{thebibliography}{10}
	\expandafter\ifx\csname url\endcsname\relax
	  \def\url#1{\texttt{#1}}\fi
	\expandafter\ifx\csname urlprefix\endcsname\relax\def\urlprefix{URL }\fi
	\expandafter\ifx\csname href\endcsname\relax
	  \def\href#1#2{#2} \def\path#1{#1}\fi
	
	\bibitem{KNN}
	WEKA (The University of Waikato)\\
	  \url{http://weka.sourceforge.net/doc.dev/weka/classifiers/lazy/IBk.html}\\
	  \url{http://weka.sourceforge.net/doc.dev/weka/core/neighboursearch/NearestNeighbourSearch.html}
	  
  	\bibitem{ARFFReader}
  	WEKA (The University of Waikato)\\
	  \url{http://weka.sourceforge.net/doc.dev/weka/core/converters/ArffLoader.ArffReader.html}
	  
	\end{thebibliography}

\end{document}
