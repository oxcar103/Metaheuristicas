%%
% Plantilla de Memoria
% Modificación de una plantilla de Latex de Nicolas Diaz para adaptarla 
% al castellano y a las necesidades de escribir informática y matemáticas.
%
% Editada por: Mario Román
%
% License:
% CC BY-NC-SA 3.0 (http://creativecommons.org/licenses/by-nc-sa/3.0/)
%%

%%%%%%%%%%%%%%%%%%%%%
% Thin Sectioned Essay
% LaTeX Template
% Version 1.0 (3/8/13)
%
% This template has been downloaded from:
% http://www.LaTeXTemplates.com
%
% Original Author:
% Nicolas Diaz (nsdiaz@uc.cl) with extensive modifications by:
% Vel (vel@latextemplates.com)
%
% License:
% CC BY-NC-SA 3.0 (http://creativecommons.org/licenses/by-nc-sa/3.0/)
%
%%%%%%%%%%%%%%%%%%%%%

%----------------------------------------------------------------------------------------
%	PAQUETES Y CONFIGURACIÓN DEL DOCUMENTO
%----------------------------------------------------------------------------------------

%% Configuración del papel.
% microtype: Tipografía.
% mathpazo: Usa la fuente Palatino.
\documentclass[a4paper, 11pt]{article}
\usepackage[protrusion=true,expansion=true]{microtype}
\usepackage{mathpazo}

% Indentación de párrafos para Palatino
\setlength{\parindent}{0pt}
  \parskip=8pt
\linespread{1.05} % Change line spacing here, Palatino benefits from a slight increase by default


%% Castellano.
% noquoting: Permite uso de comillas no españolas.
% lcroman: Permite la enumeración con numerales romanos en minúscula.
% fontenc: Usa la fuente completa para que pueda copiarse correctamente del pdf.
\usepackage[spanish,es-noquoting,es-lcroman]{babel}
\usepackage[utf8]{inputenc}
\usepackage[T1]{fontenc}
\selectlanguage{spanish}


%% Gráficos
\usepackage{graphicx} % Required for including pictures
\usepackage{wrapfig} % Allows in-line images
\usepackage[usenames,dvipsnames]{color} % Coloring code

% % Enlaces
\usepackage[hidelinks]{hyperref}

%% Matemáticas
\usepackage{amsmath}

% Para algoritmos
\usepackage{algorithm}
\usepackage{algorithmic}
\usepackage{amsthm}
\input{spanishAlgorithmic.tex}

%% Bibliografía
\makeatletter
\renewcommand\@biblabel[1]{\textbf{#1.}} % Change the square brackets for each bibliography item from '[1]' to '1.'
\renewcommand{\@listI}{\itemsep=0pt} % Reduce the space between items in the itemize and enumerate environments and the bibliography


%----------------------------------------------------------------------------------------
%	TÍTULO
%----------------------------------------------------------------------------------------
% Configuraciones para el título.
% El título no debe editarse aquí.
\renewcommand{\maketitle}{
  \begin{flushright} % Center align
  {\LARGE\@title} % Increase the font size of the title
  
  \vspace{50pt} % Some vertical space between the title and author name
  
  {\large\@author} % Author name
  \\\@date % Date
  \vspace{40pt} % Some vertical space between the author block and abstract
  \end{flushright}
}

% Título
\title{\textbf{Metaheurísticas: Selección de Características}\\ % Title
P-2: Búsquedas por Trayectorias Múltiples} % Subtitle

\author{\textsc{Óscar Bermúdez Garrido\\
\href{http://www.github.com/oxcar103}{@oxcar103}} % Author
\\{\textit{Universidad de Granada}}} % Institution

\date{\today} % Date


%----------------------------------------------------------------------------------------
%	DOCUMENTO
%----------------------------------------------------------------------------------------

\begin{document}

\maketitle % Print the title section

% Resumen (Descomentar para usarlo)
\renewcommand{\abstractname}{Resumen} % Uncomment to change the name of the abstract to something else
%\begin{abstract}
% Resumen aquí
%\end{abstract}

% Palabras clave
%\hspace*{3,6mm}\textit{Keywords:} lorem , ipsum , dolor , sit amet , lectus % Keywords
%\vspace{30pt} % Some vertical space between the abstract and first section


% Índice
{\parskip=2pt
  \tableofcontents
}
\pagebreak

%% Inicio del documento
	\section{Descripción del problema}
	Disponemos de una serie de muestras $w_1, w_2, \dots, w_r$ con un conjunto de caracterísitcas
	asociadas $(x_1(w_i), x_2, \dots, x_n(w_i))$ clasificadas en las distintas clases $\{C_1, C_2,
	\dots, C_M\}$.
	
	Queremos obtener una máscara que nos permita clasificar estas y futuras muestras a partir
	de sus atributos o características de forma automática, con la mayor tasa de acierto posible
	así como con la mayor tasa de reducción que se pueda, es decir, usando el menor número de
	características posible.
	
	Todo ello, en un tiempo razonable, pues a pesar de que podríamos encontrar el óptimo probando
	para todas las máscaras, debido a la dimensionalidad de algunos problemas, dicho coste en
	tiempo sería muy elevado.
	
	Para comprobar su funcionamiento, se suele recurrir a partir el conjunto de instancias dadas
	en dos particiones no necesariamente de igual cardinalidad. Sobre una de ellas se ejecutará
	el \textbf{Entrenamiento} donde se creará la máscara, y sobre la otra se llevará a cabo la
	\textbf{Validación} donde se usará la máscara creada para clasificar el conjunto y ver qué
	tasa de acierto tiene nuestro algoritmo. Por seguridad del buen funcionamiento del algoritmo,
	se realizan varias particiones entrenamiento-validación.
	
	Este problema tiene principalmente dos grandes adversidades que superar:
	\begin{itemize}
		\item La $"$maldición de la dimensión$"$(o efecto Hughes), que son el conjunto de fenómenos
		ocasionados a raíz del estudio y clasificación de datos de múltiple dimensionalidad.
		
		Estos fenómenos se deben porque a medida que la dimensión aumenta, el tamaño del espacio
		asociado se incrementa exponencialmente, complicando de este modo la predicción correcta
		del comportamiento de los individuos que lo componen.
		
		\item El $"$sobre-ajuste$"$ a los datos de entrenamiento, es decir, la creación de una
		máscara que maximiza la tasa de aciertos sobre los datos de entrenamiento pero dificultan
		la generalización sobre el problema, obteniendo malos resultados en la parte de validación
		y, por tanto, posiblemente en la clasificación de las futuras muestras.
	\end{itemize}

	\section{Características de las heurísticas}
	En este apartado, se describirán las características comunes empleadas para resolver
	el problema.
	
	\subsection{Representación de la solución}
		Representaremos la solución como un vector de booleanos de longitud $n$, con $n$
		el número de características del problema dado, en el cuál se especifica si la
		característica $f_i$ está en la solución o no.
	
	\subsection{Función objetivo y función evaluación}
		Como función objetivo lógica para nuestro problema, se tomaría la función:
		$$tasa_{aciertos} = 100 \cdot \frac{nº\ instancias\ bien\ clasificadas}{nº\ total\ de\ instancias}$$
		
		Sin embargo, como el número total de instancias es una constante y para la máquina
		es más sencillo trabajar con valores enteros que con valores decimales, optaremos
		por maximizar la función que nos calcula el número de aciertos y los transformaremos
		en la tasa de aciertos cuando queramos evaluar el buen funcionamiento de la
		clasificación realizada sobre el conjunto de evaluación.
		
		Un pequeño esquema del funcionamiento de nuestra función de evaluación sería:

		\begin{algorithm}[H]
			\begin{algorithmic}[1]

				\REQUIRE \ \\
		        	\texttt{c\_sel}, vector de características seleccionadas\\
		        	\texttt{inst}, instancias sobre las que evaluar\\ \

		     	\STATE{\texttt{aciertos} = 0}\\
		     	\FORALL{\texttt{instancia} $\in$ \texttt{inst}}
			  		\STATE{Genera \texttt{K-NN} de \texttt{instancia[i]}}
			  		\IF{\texttt{Hay clase predominante}}
						\STATE{\text{clase\ esperada} = Clase predominante de \texttt{K-NN} de \texttt{instancia[i]}}
					\ELSE
						\STATE{\text{clase\ esperada} = Clase de \texttt{1-NN} de \texttt{instancia[i]}}
					\ENDIF
			  		
					\IF{\texttt{clase\ esperada} == Clase de \texttt{instancia[i]}}
						\STATE{\texttt{aciertos++}}
					\ENDIF
				\ENDFOR
	  
				\RETURN{aciertos}			
			\end{algorithmic}
   \caption{Función de evaluación}
   \label{Evaluate}
		\end{algorithm}
		
		Este algoritmo tiene algunas variantes, como que si no le pasas instancia, toma la que
		usó para realizar el entrenamiento, si no le pasas vector de características seleccionadas,
		usa la que tiene almacenada y que si le pasas un entero $i$, usa el vector de características
		resultante de alternar el valor de $f_i$ de la solución almacenada.
		
		En nuestro caso, tomamos $K=3$ para realizar el KNN. Además, para el cálculo del KNN
		utilizamos el \textit{framework} que ya estaba implementado en \cite{KNN}.
	
	\subsection{Operador de vecinos}
		Para la generación de una solución vecina, basta con alterar la pertenencia de una
		característica al conjunto de características seleccionadas, y alterar el contador
		de las mismas ya que me parecía más eficiente de este modo llevar la cuenta de cuántas
		fueron escogidas. Una aproximación de sería la siguiente:
		
		\begin{algorithm}[H]
			\begin{algorithmic}[1]
				\REQUIRE \ \\
		        	\texttt{index}, índice a cambiar\\ \

		     	\STATE{\texttt{car[index]} = \texttt{!car[index]}}\\
		  		\IF{\texttt{car[index]} == \texttt{true}}
					\STATE{Incrementamos número de características seleccionadas}
				\ELSE
					\STATE{Decrementamos número de características seleccionadas}
				\ENDIF
			\end{algorithmic}
		\caption{Generación de solución vecina}
		\label{Flip}
		\end{algorithm}
	
		Y para el caso de la generación de un vecino aleatorio, bastaría con generarlo aleatoriamente,
		invocar a la función anterior y devolver el número generado (por si fuese necesario
		para algo en el ámbito en el que se invoca la función):
	
		\begin{algorithm}[H]
			\begin{algorithmic}[1]
				\REQUIRE \ \\
					 \

		     	\STATE{\texttt{index} = Entero Aleatorio}\\
		  		\STATE{\texttt{Flip(index)}}\\
		  		
				\RETURN{index}
			\end{algorithmic}
		\caption{Generación aleatoria de solución vecina}
		\label{Neighbour}
		\end{algorithm}
	
	\subsection{\textit{5x2-Cross Validation}}
		Para realizar la comprobación del buen funcionamiento de nuestras heurísticas, realizaremos
		una serie de particiones del conjunto de instancias que nos pasan como parámetro, algunas
		las usaremos como entrenamiento y otras como validación.
		
		En concreto, utilizaremos la conocida como \textit{5x2-Cross Validation}, que consiste en
		dividir el conjunto en dos mitades, tomar una como entrenamiento y la otra como validación,
		y luego invertir la que se usó como entrenamiento y la que se usó como validación. Este
		proceso se realiza 5 veces:
		
		\begin{algorithm}[H]
			\begin{algorithmic}[1]
				\REQUIRE \ \\
					 \
					 
		     	\FOR{\texttt{i} $<$ \texttt{5}}
			  		\STATE{\texttt{part1} = Genera partición}
			  		\STATE{\texttt{part2} = Complementario(\texttt{part1})}

					\STATE{\texttt{Heurística.entrenar(part1)}}
					\STATE{\texttt{evaluaciones[2*i]} = \texttt{Heurística.evaluar(part2)}}
									  		
					\STATE{\texttt{Heurística.entrenar(part2)}}
					\STATE{\texttt{evaluaciones[2*i+1]} = \texttt{Heurística.evaluar(part1)}}
				\ENDFOR
				
				\RETURN{evaluaciones}
			\end{algorithmic}
		\caption{\textit{5x2-Cross Validation}}
		\label{Cross-Validation}
		\end{algorithm}			
		
	\subsection{Variables globales}
		Existen las siguientes variables globales en el problema:
		\begin{itemize}
			\item \textbf{Semillas aleatorias:} Éstas dan cabida a la generación de los números
			aleatorios necesarios para ciertos procesos como la generación de la solución inicial
			o la generación de soluciones vecinas. Disponemos de 11 semillas aleatorias en esta
			implementación:
				\begin{itemize}
					\item \textbf{Principal:} Sólo existe una semilla aleatoria principal, a la
					que le dí el valor $103$.
					\item \textbf{Generadas:} Estas semillas están generadas de forma aleatoria
					a partir de la semilla principal. Cada una de ellas se pasa los algoritmos
					que la requieren\footnote{Todos menos \textit{SFS}.} a la hora de la
					inicialización del objeto. Existen 10 para pasar una en cada prueba de las
					heurísticas.
				\end{itemize}
			\item \textbf{Número máximo de evaluaciones:} Es un valor que nos permite limitar
			las iteraciones de los algoritmos(ya que, de otro modo, podrían no terminar o tardar
			mucho más tiempo)\footnote{Usada por todas excepto la \textit{SFS} porque necesitaría
			una cantidad tan desproporcionada de características en la instancia pasada que
			dicha variable carecería de sentido usarla.}. Le hemos dado el valor de 15000.
		\end{itemize}
	
	\subsection{Entrada y salida}
		Para recoger los datos a través de la entrada, he implementado una clase extra basándome
		en \cite{ARFFReader}.
		Para generar los archivos \textit{.csv}, he utilizado una clase implementada por un
		compañero y algo adaptada a mi estilo de programación.

	
	\section{Heurísticas implementadas}
		\subsection{Búsqueda Multiarranque Básica (\textbf{BMBS})}
			La dinámica de trabajo de esta técnica de búsqueda consiste en:
			\begin{enumerate}
				\item Generar un número $n$ de soluciones iniciales aleatorias.
				\item Optimizarlas usando búsqueda local\footnote{Como necesitamos optimizar mediante
				búsqueda local, haremos que herede de la clase \textit{LocalSearch} para facilitar
				la implementación.} sobre cada una de ellas.
				\item Quedarnos con la mejor solución obtenida tras dicha mejora.
			\end{enumerate}
			
			Para nuestro caso particular, generaremos un total de 25 soluciones iniciales utilizando
			el siguiente método:
			
			\begin{algorithm}[H]
				\begin{algorithmic}[1]
				\REQUIRE \ \\
						 \
				\FOR{\texttt{i} $<$ \texttt{Número de características}}
					\STATE{\texttt{rnd} = Genera valor aleatorio entre 0 y 1}
					\IF{\texttt{rnd} == 1}
						\STATE{Flip(i)}
					\ENDIF
				\ENDFOR
				\end{algorithmic}
			\caption{Solución Aleatoria(\textit{RandomSolution})}
			\label{RandomSolution}
			\end{algorithm}
			
			El algoritmo de su función de entrenamiento sería tan simple como \footnote{Nótese que la
			función \textbf{Optimizar} utilizada se traduce en código como \textit{super.train()} al
			heredar de la clase \textit{LocalSearch}}:
			
			\begin{algorithm}[H]
				\begin{algorithmic}[1]
				\REQUIRE \ \\
						 \
				\STATE{\texttt{i} = 0}
				\STATE{\texttt{Solución\_actual} = RandomSolution()}
				\STATE{\texttt{Solución\_actual}.Optimizar()}
				\STATE{\texttt{Evaluación\_actual} = \texttt{Evaluate()}}
				\STATE{\texttt{Mejor\_Solución} = \texttt{Solución\_actual}}
				\STATE{\texttt{Mejor\_Evaluación} = \texttt{Evaluación\_actual}}
				\STATE{\texttt{i++}}
				\FOR{\texttt{i} $<$ \texttt{Número de Soluciones Iniciales}}
					\STATE{\texttt{Solución\_actual} = RandomSolution()}
					\STATE{\texttt{Solución\_actual}.Optimizar()}
					\STATE{\texttt{Evaluación\_actual} = \texttt{Evaluate()}}
					\IF{\texttt{Evaluación\_actual} > \texttt{Mejor\_Evaluación}}
						\STATE{\texttt{Mejor\_Solución} = \texttt{Solución\_actual}}
						\STATE{\texttt{Mejor\_Evaluación} = \texttt{Evaluación\_actual}}
					\ENDIF
				\ENDFOR
				\STATE{\texttt{Solución\_actual} = \texttt{Mejor\_Solución}}
			\end{algorithmic}
		\caption{Búsqueda Multiarranque Básica}
		\label{BMBS}
		\end{algorithm}

		\subsection{\textit{Greedy Randomized Adaptive Search Procedure} (\textbf{GRASP})}
			La dinámica de trabajo de esta técnica de búsqueda consiste en:
			\begin{enumerate}
				\item Generar un número $n$ de soluciones iniciales greedy probabilistas.
				\item Optimizarlas usando búsqueda local\footnote{Como necesitamos optimizar mediante
				búsqueda local, haremos que herede de la clase \textit{LocalSearch} para facilitar
				la implementación.} sobre cada una de ellas.
				\item Quedarnos con la mejor solución obtenida tras dicha mejora.
			\end{enumerate}
			
			Para nuestro caso particular, generaremos un total de 25 soluciones iniciales utilizando
			el siguiente método:
			
			\begin{algorithm}[H]
				\begin{algorithmic}[1]
					\REQUIRE \ \\
						\texttt{Mejor\_Evaluación}, mejor evaluación \\
						\texttt{Peor\_Evaluación}, peor evaluación \\ \
						
				\STATE{$\texttt{Umbral} = \texttt{Mejor\_Evaluación} - \alpha \cdot
					(\texttt{Mejor\_Evaluación} - \texttt{Peor\_Evaluación})$}
				\RETURN{\texttt{Umbral}}
				\end{algorithmic}
			\caption{GRASP - Umbral(\textit{Treshold})}
			\label{Treshold}
			\end{algorithm}
			
			Hemos utilizado la variable $\alpha$ que representa la tolerancia de calidad, esto es, la
			proporción de empeoramiento de la mejor solución para la construcción de la lista restringida
			de candidatos de entre la que sacaremos probabilísticamente el índice que añadiremos. En
			nuestro caso, utilizaremos la variable con un valor de $0.3$.
			
			El método de inicialización de soluciones es muy similar a la construcción de soluciones por
			\textbf{SFS}, de hecho, podemos considerar que es un caso particular con $\alpha = 0$ para
			nuestro algoritmo:

			\begin{algorithm}[H]
				\begin{algorithmic}[1]
					\REQUIRE \ \\
							 \
					\STATE{\texttt{Hay\_Mejora} = \TRUE}
					\WHILE{\texttt{Hay\_Mejora} \AND \texttt{Número de características seleccionadas} $<$
							\texttt{Número de características}}
						\STATE{Vaciamos \texttt{Lista\_Aspirantes}}
						\STATE{\texttt{Mejor\_Evaluación} = 0}
						\STATE{\texttt{Peor\_Evaluación} = $\infty$}
						\FOR{\texttt{i} $<$ \texttt{Número de características}}
							\IF{\texttt{Característica$_i$} no está seleccionada ni es de clase}
								\STATE{\texttt{Evaluación\_Actual} = Evaluate(i)}
								\STATE{Guardamos \texttt{Evaluación\_Actual} en \texttt{Lista\_Aspirantes}}
								\IF{\texttt{Evaluación\_Actual} $<$ \texttt{Peor\_Evaluación}}
									\STATE{\texttt{Peor\_Evaluación} = \texttt{Evaluación\_Actual}}
								\ENDIF
								\IF{\texttt{Evaluación\_Actual} $>$ \texttt{Mejor\_Evaluación}}
									\STATE{\texttt{Mejor\_Evaluación} = \texttt{Evaluación\_Actual}}
								\ENDIF
							\ENDIF
						\ENDFOR
						\STATE{\texttt{Umbral} = Treshold(\texttt{Peor\_Evaluación}, \texttt{Mejor\_Evaluación})}
						\FORALL{\texttt{Evaluación} en \texttt{Lista\_Aspirantes}}
							\IF{\texttt{Evaluación} $<$ \texttt{Umbral}}
								\STATE{Eliminar \texttt{Evaluación} de \texttt{Lista\_Aspirantes}}
							\ENDIF
						\ENDFOR
						\STATE{\texttt{rnd} = Genera valor aleatorio entre 0 y \texttt{Lista\_Aspirantes}.size()-1}
						\IF{\texttt{Lista\_Aspirantes[rnd]} mejora \texttt{Solución\_Actual}}
							\STATE{\texttt{Flip(índice de la característica mejorada por \texttt{rnd})}}
							\STATE{Guardamos el valor de \texttt{Solución\_Actual}}
						\ELSE
							\STATE{\texttt{Hay\_Mejora} = \FALSE}
						\ENDIF
					\ENDWHILE
				\end{algorithmic}
			\caption{GRASP - Solución Greedy Probabilista(\textit{NewGreedySolution})}
			\label{GreedySolution}
			\end{algorithm}
			
			Para el entrenamiento, se ha utilizado el mismo algoritmo que con \textbf{BMBS} pero tomando
			la función \textit{GreedySolution} en lugar de \textit{RandomSolution}:
			
			\begin{algorithm}[H]
				\begin{algorithmic}[1]
				\REQUIRE \ \\
						 \
				\STATE{\texttt{i} = 0}
				\STATE{\texttt{Solución\_actual} = GreedySolution()}
				\STATE{\texttt{Solución\_actual}.Optimizar()}
				\STATE{\texttt{Evaluación\_actual} = \texttt{Evaluate()}}
				\STATE{\texttt{Mejor\_Solución} = \texttt{Solución\_actual}}
				\STATE{\texttt{Mejor\_Evaluación} = \texttt{Evaluación\_actual}}
				\STATE{\texttt{i++}}
				\FOR{\texttt{i} $<$ \texttt{Número de Soluciones Iniciales}}
					\STATE{\texttt{Solución\_actual} = GreedySolution()}
					\STATE{\texttt{Solución\_actual}.Optimizar()}
					\STATE{\texttt{Evaluación\_actual} = \texttt{Evaluate()}}
					\IF{\texttt{Evaluación\_actual} > \texttt{Mejor\_Evaluación}}
						\STATE{\texttt{Mejor\_Solución} = \texttt{Solución\_actual}}
						\STATE{\texttt{Mejor\_Evaluación} = \texttt{Evaluación\_actual}}
					\ENDIF
				\ENDFOR
				\STATE{\texttt{Solución\_actual} = \texttt{Mejor\_Solución}}
			\end{algorithmic}
		\caption{GRASP - Entrenamiento}
		\label{GRASP}
		\end{algorithm}
		
		\subsection{Búsqueda Local Reiterada (\textbf{ILS})}
	\section{Resultados}
		\subsection{Búsqueda Multiarranque Básica (\textbf{BMBS})}
			\begin{table}[H]
	\centering
	\begin{tabular}{l|lll}
		Nombre        & Tasa de acierto(\%) & Tasa de reducción(\%) & Tiempo(s)          \\ \hline
		Partición 1-1 & 96.12676056338029   & -430.0                & 32.558             \\
		Partición 1-2 & 92.63157894736842   & -50.0                 & 26.312             \\
		Partición 2-1 & 97.1830985915493    & 33.333333333333336    & 32.857             \\
		Partición 2-2 & 94.03508771929825   & -243.33333333333334   & 32.146             \\
		Partición 3-1 & 96.47887323943662   & -750.0                & 32.604             \\
		Partición 3-2 & 94.73684210526316   & -186.66666666666666   & 24.396             \\
		Partición 4-1 & 96.12676056338029   & 6.666666666666667     & 32.473             \\
		Partición 4-2 & 92.98245614035088   & -1053.3333333333333   & 30.755             \\
		Partición 5-1 & 96.47887323943662   & -376.6666666666667    & 33.917             \\
		Partición 5-2 & 93.6842105263158    & -210.0                & 23.389             \\ \hline
		Media         & 95.04645416357798   & -325.99999999999994   & 30.140700000000002
	\end{tabular}
	\caption{WDBC - BMBS}
	\label{WDBC-BMBS}
\end{table}
			\begin{table}[H]
	\centering
	\begin{tabular}{l|lll}
		Nombre        & Tasa de acierto(\%) & Tasa de reducción(\%) & Tiempo(s)          \\ \hline
		Partición 1-1 & 67.77777777777777   & 46.666666666666664    & 164.808            \\
		Partición 1-2 & 72.22222222222223   & 40.0                  & 191.95             \\
		Partición 2-1 & 75.0                & 48.888888888888886    & 234.277            \\
		Partición 2-2 & 63.333333333333336  & 46.666666666666664    & 198.46             \\
		Partición 3-1 & 77.77777777777777   & 53.333333333333336    & 210.994            \\
		Partición 3-2 & 63.888888888888886  & 56.666666666666664    & 223.015            \\
		Partición 4-1 & 66.66666666666667   & 52.22222222222222     & 219.249            \\
		Partición 4-2 & 64.44444444444444   & 55.55555555555556     & 214.729            \\
		Partición 5-1 & 71.11111111111111   & 54.44444444444444     & 212.195            \\
		Partición 5-2 & 65.55555555555556   & 50.0                  & 143.887            \\ \hline
		Media         & 68.77777777777777   & 50.44444444444444     & 201.35639999999998
	\end{tabular}
	\caption{Movement Libras - BMBS}
	\label{MLIB-BMBS}
\end{table}
			\input{ARRH-BMBS}
		\subsection{\textit{Greedy Randomized Adaptive Search Procedure} (\textbf{GRASP})}
			\begin{table}[H]
	\centering
	\begin{tabular}{l|lll}
		Nombre        & Tasa de acierto(\%) & Tasa de reducción(\%) & Tiempo(s) \\ \hline
		Partición 1-1 & 94.36619718309859   & -566.6666666666666    & 27.247    \\
		Partición 1-2 & 94.73684210526316   & -263.3333333333333    & 24.748    \\
		Partición 2-1 & 94.36619718309859   & -386.6666666666667    & 23.74     \\
		Partición 2-2 & 93.6842105263158    & -343.3333333333333    & 26.132    \\
		Partición 3-1 & 96.47887323943662   & -26.666666666666668   & 24.218    \\
		Partición 3-2 & 96.84210526315789   & -100.0                & 26.572    \\
		Partición 4-1 & 96.83098591549296   & -130.0                & 30.627    \\
		Partición 4-2 & 92.63157894736842   & -606.6666666666666    & 26.602    \\
		Partición 5-1 & 97.53521126760563   & -170.0                & 27.297    \\
		Partición 5-2 & 94.73684210526316   & 6.666666666666667     & 26.75     \\ \hline
		Media         & 95.22090437361008   & -258.6666666666667    & 26.3933  
	\end{tabular}
	\caption{WDBC - GRASP}
	\label{WDBC-GRASP}
\end{table}
			\input{MLIB-GRASP}
			\begin{table}[H]
	\centering
	\begin{tabular}{l|lll}
		Nombre        & Tasa de acierto(\%) & Tasa de reducción(\%) & Tiempo(s) \\ \hline
		Partición 1-1 & 68.9119170984456    & 19.424460431654676    & 569.781   \\
		Partición 1-2 & 72.02072538860104   & 16.18705035971223     & 523.682   \\
		Partición 2-1 & 67.35751295336787   & -1.4388489208633093   & 561.97    \\
		Partición 2-2 & 70.98445595854922   & 19.06474820143885     & 667.75    \\
		Partición 3-1 & 60.10362694300518   & 9.712230215827338     & 668.8     \\
		Partición 3-2 & 73.05699481865285   & 38.84892086330935     & 546.068   \\
		Partición 4-1 & 68.39378238341969   & 43.52517985611511     & 506.674   \\
		Partición 4-2 & 70.98445595854922   & 46.0431654676259      & 519.151   \\
		Partición 5-1 & 68.9119170984456    & 26.258992805755394    & 503.284   \\
		Partición 5-2 & 74.61139896373057   & 46.76258992805755     & 541.513   \\ \hline
		Media         & 69.53367875647669   & 26.438848920863308    & 560.8673
	\end{tabular}
	\caption{Arrhythmia - GRASP}
	\label{ARRH-GRASP}
\end{table}
		\subsection{Búsqueda Local Reiterada (\textbf{ILS})}
			\input{WDBC-ILS}
			\begin{table}[H]
	\centering
	\begin{tabular}{l|lll}
		Nombre        & Tasa de acierto(\%) & Tasa de reducción(\%) & Tiempo(s) \\ \hline
		Partición 1-1 & 71.66666666666667   & 52.22222222222222     & 117.308   \\
		Partición 1-2 & 65.0                & 60.0                  & 103.43    \\
		Partición 2-1 & 71.66666666666667   & 47.77777777777778     & 107.089   \\
		Partición 2-2 & 74.44444444444444   & 52.22222222222222     & 113.007   \\
		Partición 3-1 & 71.11111111111111   & 51.111111111111114    & 116.439   \\
		Partición 3-2 & 71.66666666666667   & 55.55555555555556     & 110.307   \\
		Partición 4-1 & 73.88888888888889   & 52.22222222222222     & 109.99    \\
		Partición 4-2 & 67.22222222222223   & 47.77777777777778     & 130.149   \\
		Partición 5-1 & 65.55555555555556   & 51.111111111111114    & 103.269   \\
		Partición 5-2 & 71.66666666666667   & 62.22222222222222     & 114.676   \\ \hline
		Media         & 70.38888888888889   & 53.222222222222214    & 112.5664
	\end{tabular}
	\caption{Movement Libras - ILS}
	\label{MLIB-ILS}
\end{table}
			\begin{table}[H]
	\centering
	\begin{tabular}{l|lll}
		Nombre        & Tasa de acierto(\%) & Tasa de reducción(\%) & Tiempo(s)         \\ \hline
		Partición 1-1 & 65.80310880829016   & 50.0                  & 875.551           \\
		Partición 1-2 & 61.6580310880829    & 49.64028776978417     & 871.76            \\
		Partición 2-1 & 62.17616580310881   & 54.31654676258993     & 887.604           \\
		Partición 2-2 & 64.24870466321244   & 51.798561151079134    & 1010.671          \\
		Partición 3-1 & 67.87564766839378   & 56.47482014388489     & 1103.306          \\
		Partición 3-2 & 58.54922279792746   & 45.32374100719424     & 882.833           \\
		Partición 4-1 & 59.58549222797927   & 52.87769784172662     & 1164.788          \\
		Partición 4-2 & 68.39378238341969   & 48.56115107913669     & 1044.289          \\
		Partición 5-1 & 70.46632124352331   & 53.9568345323741      & 942.765           \\
		Partición 5-2 & 61.13989637305699   & 52.51798561151079     & 1017.056          \\ \hline
		Media         & 63.98963730569949   & 51.546762589928065    & 980.0622999999999
	\end{tabular}
	\caption{Arrhythmia - ILS}
	\label{ARRH-ILS}
\end{table}
			
		\subsection{Comparación de resultados}
	
	\section{Implementación y compilación del proyecto}
	Este proyecto se ha realizado en Java en el entorno de desarrollo NetBeans y se han utilizado
	los \textit{frameworks} de \textit{Weka} principalmente para el tratamiento previo del
	conjunto de instancias(normalización de los mismos), la separación de las instancias en
	instancias de Entrenamiento y de Valoración y para la llamada al $K-NN$ utilizada en la
	función de valoración.
	
	Para la compilación del proyecto, se recomienda el uso de la plataforma NetBeans por su
	sencillez a la hora de compilar, \textit{$"$debuggear$"$} y ejecutar el proyecto(aunque,
	como usa parámetros de entrada, en lo personal, prefiero ejecutarlo por terminal).
	
	No obstante, se puede realizar la compilación del proyecto con \textit{ant -f} <nombre del
	proyecto> y su ejecución mediante \textit{java -jar} <nombre del proyecto> <parámetros>.
	
	Los parámetros serían:
		\begin{itemize}
			\item Número de archivos en los que se le pedirá aplicar las heurísticas.
			\item Párametros de cada fichero, es decir:
				\begin{itemize}
					\item Dirección al fichero.
					\item Columna en la que se encuentra la característica de clase.
					\item Base del nombre del fichero de salida\footnote{Por cada fichero
					de entrada, se generan 4 de salida, uno por cada algoritmo}.
				\end{itemize}
		\end{itemize}

	\newpage
	
	\begin{thebibliography}{10}
	\expandafter\ifx\csname url\endcsname\relax
	  \def\url#1{\texttt{#1}}\fi
	\expandafter\ifx\csname urlprefix\endcsname\relax\def\urlprefix{URL }\fi
	\expandafter\ifx\csname href\endcsname\relax
	  \def\href#1#2{#2} \def\path#1{#1}\fi
	
	\bibitem{KNN}
	WEKA (The University of Waikato)\\
	  \url{http://weka.sourceforge.net/doc.dev/weka/classifiers/lazy/IBk.html}\\
	  \url{http://weka.sourceforge.net/doc.dev/weka/core/neighboursearch/NearestNeighbourSearch.html}
	  
  	\bibitem{ARFFReader}
  	WEKA (The University of Waikato)\\
	  \url{http://weka.sourceforge.net/doc.dev/weka/core/converters/ArffLoader.ArffReader.html}
	  
	\end{thebibliography}

\end{document}
