%%
% Plantilla de Memoria
% Modificación de una plantilla de Latex de Nicolas Diaz para adaptarla 
% al castellano y a las necesidades de escribir informática y matemáticas.
%
% Editada por: Mario Román
%
% License:
% CC BY-NC-SA 3.0 (http://creativecommons.org/licenses/by-nc-sa/3.0/)
%%

%%%%%%%%%%%%%%%%%%%%%
% Thin Sectioned Essay
% LaTeX Template
% Version 1.0 (3/8/13)
%
% This template has been downloaded from:
% http://www.LaTeXTemplates.com
%
% Original Author:
% Nicolas Diaz (nsdiaz@uc.cl) with extensive modifications by:
% Vel (vel@latextemplates.com)
%
% License:
% CC BY-NC-SA 3.0 (http://creativecommons.org/licenses/by-nc-sa/3.0/)
%
%%%%%%%%%%%%%%%%%%%%%

%----------------------------------------------------------------------------------------
%	PAQUETES Y CONFIGURACIÓN DEL DOCUMENTO
%----------------------------------------------------------------------------------------

%% Configuración del papel.
% microtype: Tipografía.
% mathpazo: Usa la fuente Palatino.
\documentclass[a4paper, 11pt]{article}
\usepackage[protrusion=true,expansion=true]{microtype}
\usepackage{mathpazo}

% Indentación de párrafos para Palatino
\setlength{\parindent}{0pt}
  \parskip=8pt
\linespread{1.05} % Change line spacing here, Palatino benefits from a slight increase by default


%% Castellano.
% noquoting: Permite uso de comillas no españolas.
% lcroman: Permite la enumeración con numerales romanos en minúscula.
% fontenc: Usa la fuente completa para que pueda copiarse correctamente del pdf.
\usepackage[spanish,es-noquoting,es-lcroman]{babel}
\usepackage[utf8]{inputenc}
\usepackage[T1]{fontenc}
\selectlanguage{spanish}


%% Gráficos
\usepackage{graphicx} % Required for including pictures
\usepackage{wrapfig} % Allows in-line images
\usepackage[usenames,dvipsnames]{color} % Coloring code

% % Enlaces
\usepackage[hidelinks]{hyperref}

%% Matemáticas
\usepackage{amsmath}

% Para algoritmos
\usepackage{algorithm}
\usepackage{algorithmic}
\usepackage{amsthm}
\floatname{algorithm}{Algoritmo}
\renewcommand{\listalgorithmname}{Lista de algoritmos}
\renewcommand{\algorithmicrequire}{\textbf{Entrada:}}
\renewcommand{\algorithmicensure}{\textbf{Salida:}}
\renewcommand{\algorithmicend}{\textbf{fin}}
\renewcommand{\algorithmicif}{\textbf{si}}
\renewcommand{\algorithmicthen}{\textbf{entonces}}
\renewcommand{\algorithmicelse}{\textbf{en otro caso}}
\renewcommand{\algorithmicelsif}{\algorithmicelse,\ \algorithmicif}
\renewcommand{\algorithmicendif}{\algorithmicend\ \algorithmicif}
\renewcommand{\algorithmicfor}{\textbf{para }}
\renewcommand{\algorithmicforall}{\textbf{para cada}}
\renewcommand{\algorithmicdo}{\textbf{}}
\renewcommand{\algorithmicendfor}{\algorithmicend\ \algorithmicfor}
\renewcommand{\algorithmicwhile}{\textbf{mientras}}
\renewcommand{\algorithmicendwhile}{\algorithmicend\ \algorithmicwhile}
\renewcommand{\algorithmicloop}{\textbf{repetir}}
\renewcommand{\algorithmicendloop}{\algorithmicend\ \algorithmicloop}
\renewcommand{\algorithmicrepeat}{\textbf{repetir}}
\renewcommand{\algorithmicuntil}{\textbf{hasta que}}
\renewcommand{\algorithmicprint}{\textbf{imprimir}} 
\renewcommand{\algorithmicreturn}{\textbf{devolver}} 
\renewcommand{\algorithmictrue}{\textbf{true }} 
\renewcommand{\algorithmicfalse}{\textbf{false }} 
\renewcommand{\algorithmicand}{\textbf{y}}
\renewcommand{\algorithmicor}{\textbf{o}}


%% Bibliografía
\makeatletter
\renewcommand\@biblabel[1]{\textbf{#1.}} % Change the square brackets for each bibliography item from '[1]' to '1.'
\renewcommand{\@listI}{\itemsep=0pt} % Reduce the space between items in the itemize and enumerate environments and the bibliography


%----------------------------------------------------------------------------------------
%	TÍTULO
%----------------------------------------------------------------------------------------
% Configuraciones para el título.
% El título no debe editarse aquí.
\renewcommand{\maketitle}{
  \begin{flushright} % Center align
  {\LARGE\@title} % Increase the font size of the title
  
  \vspace{50pt} % Some vertical space between the title and author name
  
  {\large\@author} % Author name
  \\\@date % Date
  \vspace{40pt} % Some vertical space between the author block and abstract
  \end{flushright}
}

% Título
\title{\textbf{Metaheurísticas: Selección de Características}\\ % Title
P-5: Algoritmos Meméticos} % Subtitle

\author{\textsc{Óscar Bermúdez Garrido\\
\href{http://www.github.com/oxcar103}{@oxcar103}} % Author
\\{\textit{Universidad de Granada}}} % Institution

\date{\today} % Date


%----------------------------------------------------------------------------------------
%	DOCUMENTO
%----------------------------------------------------------------------------------------

\begin{document}

\maketitle % Print the title section

% Resumen (Descomentar para usarlo)
\renewcommand{\abstractname}{Resumen} % Uncomment to change the name of the abstract to something else
%\begin{abstract}
% Resumen aquí
%\end{abstract}

% Palabras clave
%\hspace*{3,6mm}\textit{Keywords:} lorem , ipsum , dolor , sit amet , lectus % Keywords
%\vspace{30pt} % Some vertical space between the abstract and first section


% Índice
{\parskip=2pt
  \tableofcontents
}
\pagebreak

%% Inicio del documento
	\input{Introducción.tex}
	\input{CaracterísticasComunes.tex}
	
	\section{Heurísticas implementadas}\footnote{En la implementación de las heurísticas desarrolladas
	en esta práctica, se utilizan la búsqueda de soluciones por la heurística \textbf{SFS} para comparar
	los resultados obtenidos y las optimizaciones de la \textbf{LS} para mejorar los resultados obtenidos
	por estas nuevas heurísticas multiarranque. Dado que ya se explicaron en \textbf{P1-Memoria}, he
	considerado ahorrar su explicación en este documento para permitir una mayor limpieza del mismo.}
	
		\subsection{Algoritmos Meméticos (\textbf{MA})}
			A grandes rasgos, podemos realizar una descripción general de los Algoritmos Meméticos o
			\textit{Memetic Algorithm}(\textit{MA}) que consiste en optimizar a toda o parte de la
			población de los algoritmos genéticos(\textit{GA}) cada cierto tiempo:
			
			\begin{itemize}
				\item Generar una población inicial
				\item Evolucionar dicha población, o equivalentemente se puede expresar mediante los
				siguientes operadores:
					\begin{itemize}
						\item Selección: Se eligen parejas de la población para reproducirse.
						\item Cruce: Se recombinan las parejas seleccionadas con una probabilidad de
						cruce característica del problema haciendo tomar a las soluciones sucesoras
						características de ambos padres.
						\item Mutación: Se modifican aleatoriamente los genes de la nueva generación
						con una probabilidad de mutación, normalmente, bastante reducida.
						\item Reemplazo: Se procede a incluir la nueva generación en la población
						que actualmente tenemos.
					\end{itemize}
				\item Periódicamente, cada \textit{k} generaciones, optimizamos a toda la población o
				parte de la misma, en función de una probabilidad de mejora por cada cromosoma.
				\item Tras el paso de varias generaciones, nos quedamos con el mejor individuo que
				haya existido.
			\end{itemize}
			
			Este proceso se puede reflejar en el siguiente algoritmo\footnote{Cabría destacar que,
			realmente, \texttt{Padres\_Seleccionados} no se devuelve por el Operador de Selección, si
			no que se trata como una variable global que se ve modificada por cierto número de factores}
			donde la población para nuestro modelo es de 10 individuos:
			
			\begin{algorithm}[H]
				\begin{algorithmic}[1]
				\REQUIRE \ \\
				 \
	 
				\STATE{i = 0}
				\STATE{\texttt{Índice\_Mejor}}
				\FOR{\texttt{i} $<$ \texttt{Población}}
					\STATE{\texttt{Padres}.add(RandomSolution())}
					\IF{\texttt{Padres[Índice\_Mejor]} < \texttt{Padres[i]}}
						\STATE{\texttt{Índice\_Mejor} = i}
					\ENDIF
				\ENDFOR
				
				\STATE{i = 0}
				
				\WHILE{\texttt{Evaluaciones} < 15000}
					\STATE{Limpiamos \texttt{Hijos}}
					\STATE{\texttt{Padres\_Seleccionados} = Selection()}
					\STATE{j = 0}
					\FOR{j < $\displaystyle \frac{\texttt{Padres\_Seleccionados}.size()}{2}$}
						\STATE{Crossover(2 $\cdot$ j, 2 $\cdot$ j+1)}
					\ENDFOR
					\STATE{Mutation()}
					\STATE{Inheritance()}
					\IF{i+1 \% \textit{k} == 0}
						\STATE{Improve()}
					\ENDIF
				\ENDWHILE
				
				\STATE{\texttt{Solución\_actual = Padres[Índice\_Mejor]}}
				\end{algorithmic}
			\caption{Algoritmos Meméticos - Entrenamiento(\textit{Train})}
			\label{MA-Train}
			\end{algorithm}
			
			Gran parte de los algoritmos usados son heredados de los algoritmos genéticos (\textbf{GA}),
			en concreto, del Algoritmo Genético Gerenacional (\textit{GGA}) explicado detalladamente en
			\textbf{P3-Memoria}.
			
			El único elemento nuevo, es el operador de mejora. Este operador se compone básicamente
			de dos partes:
			\begin{itemize}
				\item Selección de un subconjunto de la población, que tendrá dos variantes que veremos
				en los siguientes apartados.
				\item Aplicación de una búsqueda local de baja intensidad, esto es, una única iteración
				de la misma sobre el subconjunto de la población seleccionado.
			\end{itemize}
			
			Tenemos en común que los distintos algoritmos meméticos que usaremos en esta práctica comparten
			el valor de la mayoría de sus variables:
			\begin{itemize}
				\item Probabilidad de cruce = 0.7
				\item Probabilidad de mutación = 0.001
				\item Tamaño de la población = 10
				\item Número de iteraciones \textit{k} tras el cuál se aplicará la mejora = 10
			\end{itemize}
			
		\subsection{Algoritmo Memético Total (\textbf{TMA})}
			El algoritmo memético total se ha tomado como un caso particular de algoritmo memético parcial
			en el cuál, la probabilidad de mejora de un cromosoma es de 1. Por tanto, el subconjunto de
			la población seleccionado será toda la población.
		
		\subsection{Algoritmo Memético Parcial (\textbf{PMA})}
		\subsection{Algotitmo Memético Parcial Elitista (\textbf{EPMA})}
		
	\section{Resultados}\footnote{En este documento, se comparan los resultados obtenidos por las
	heurísticas desarrolladas en esta práctica con la búsqueda de soluciones por la heurística
	\textbf{SFS} cuyos resultados ya se expusieron en \textbf{P1-Memoria}, por lo que también he
	obviado su inclusión en este documento en toda su extensión mas sí han sido incluidas las medias
	de ejecución en la tabla comparativa.}
	
		\subsection{Algoritmo Memético Total (\textbf{TMA})}
			\begin{table}[H]
	\centering
	\begin{tabular}{l|lll}
		Nombre        & Tasa de acierto(\%) & Tasa de reducción(\%) & Tiempo(s)          \\ \hline
		Partición 1-1 & 95.77464788732394   & 46.666666666666664    & 393.624            \\
		Partición 1-2 & 93.6842105263158    & 40.0                  & 341.458            \\
		Partición 2-1 & 95.4225352112676    & 46.666666666666664    & 367.4              \\
		Partición 2-2 & 93.33333333333333   & 56.666666666666664    & 353.936            \\
		Partición 3-1 & 93.30985915492958   & 46.666666666666664    & 353.209            \\
		Partición 3-2 & 94.03508771929825   & 40.0                  & 348.396            \\
		Partición 4-1 & 97.53521126760563   & 40.0                  & 399.221            \\
		Partición 4-2 & 94.73684210526316   & 50.0                  & 370.773            \\
		Partición 5-1 & 95.4225352112676    & 43.333333333333336    & 313.611            \\
		Partición 5-2 & 90.87719298245614   & 50.0                  & 373.271            \\ \hline
		Media         & 94.41314553990608   & 45.99999999999999     & 361.48990000000003
	\end{tabular}
	\caption{WDBC - TMA}
	\label{WDBC-TMA}
\end{table}
			\begin{table}[H]
	\centering
	\begin{tabular}{l|lll}
		Nombre        & Tasa de acierto(\%) & Tasa de reducción(\%) & Tiempo(s)          \\ \hline
		Partición 1-1 & 72.22222222222223   & 47.77777777777778     & 483.078           \\
		Partición 1-2 & 68.33333333333333   & 51.111111111111114    & 413.394           \\
		Partición 2-1 & 69.44444444444444   & 51.111111111111114    & 425.151           \\
		Partición 2-2 & 71.11111111111111   & 55.55555555555556     & 407.325           \\
		Partición 3-1 & 75.0                & 44.44444444444444     & 365.033           \\
		Partición 3-2 & 66.66666666666667   & 52.22222222222222     & 602.561           \\
		Partición 4-1 & 66.66666666666667   & 43.333333333333336    & 568.858           \\
		Partición 4-2 & 72.77777777777777   & 54.44444444444444     & 531.985           \\
		Partición 5-1 & 65.0                & 55.55555555555556     & 590.151           \\
		Partición 5-2 & 69.44444444444444   & 55.55555555555556     & 559.975           \\ \hline
		Media         & 69.66666666666666   & 51.11111111111111     & 494.7511000000001
	\end{tabular}
	\caption{Movement Libras - TMA}
	\label{MLIB-TMA}
\end{table}
			\begin{table}[H]
	\centering
	\begin{tabular}{l|lll}
		Nombre        & Tasa de acierto(\%) & Tasa de reducción(\%) & Tiempo(s)          \\ \hline
		Partición 1-1 & 65.28497409326425   & 48.56115107913669     & 2804.111           \\
		Partición 1-2 & 62.69430051813472   & 47.12230215827338     & 2239.5             \\
		Partición 2-1 & 59.58549222797927   & 55.03597122302158     & 2203.75            \\
		Partición 2-2 & 66.32124352331606   & 48.201438848920866    & 1997.665           \\
		Partición 3-1 & 61.6580310880829    & 50.35971223021583     & 1916.014           \\
		Partición 3-2 & 65.80310880829016   & 45.68345323741007     & 1894.98            \\
		Partición 4-1 & 61.13989637305699   & 43.52517985611511     & 1854.196           \\
		Partición 4-2 & 58.031088082901555  & 48.201438848920866    & 1854.773           \\
		Partición 5-1 & 61.13989637305699   & 51.798561151079134    & 1746.643           \\
		Partición 5-2 & 63.21243523316062   & 49.280575539568346    & 1754.397           \\ \hline
		Media         & 62.48704663212435   & 48.77697841726619     & 2026.6028999999999
	\end{tabular}
	\caption{Arrhythmia - TMA}
	\label{ARRH-TMA}
\end{table}
		\subsection{Algoritmo Memético Parcial (\textbf{PMA})}
			\begin{table}[H]
	\centering
	\begin{tabular}{l|lll}
		Nombre        & Tasa de acierto(\%) & Tasa de reducción(\%) & Tiempo(s)          \\ \hline
		Partición 1-1 & 94.71830985915493   & 50.0                  & 408.058            \\
		Partición 1-2 & 95.78947368421052   & 60.0                  & 342.7              \\
		Partición 2-1 & 95.4225352112676    & 40.0                  & 381.02             \\
		Partición 2-2 & 94.3859649122807    & 36.666666666666664    & 394.217            \\
		Partición 3-1 & 93.66197183098592   & 36.666666666666664    & 448.106            \\
		Partición 3-2 & 95.78947368421052   & 53.333333333333336    & 442.366            \\
		Partición 4-1 & 96.12676056338029   & 50.0                  & 375.013            \\
		Partición 4-2 & 95.08771929824562   & 33.333333333333336    & 372.708            \\
		Partición 5-1 & 95.4225352112676    & 50.0                  & 322.973            \\
		Partición 5-2 & 93.6842105263158    & 33.333333333333336    & 420.37             \\ \hline
		Media         & 95.00889547813196   & 44.33333333333333     & 390.75309999999996
	\end{tabular}
	\caption{WDBC - PMA}
	\label{WDBC-PMA}
\end{table}
			\begin{table}[H]
	\centering
	\begin{tabular}{l|lll}
		Nombre        & Tasa de acierto(\%) & Tasa de reducción(\%) & Tiempo(s) \\ \hline
		Partición 1-1 & 70.0                & 51.111111111111114    & 463.453   \\
		Partición 1-2 & 72.77777777777777   & 51.111111111111114    & 441.972   \\
		Partición 2-1 & 67.77777777777777   & 46.666666666666664    & 552.651   \\
		Partición 2-2 & 66.11111111111111   & 50.0                  & 500.743   \\
		Partición 3-1 & 63.333333333333336  & 47.77777777777778     & 418.51    \\
		Partición 3-2 & 80.55555555555556   & 46.666666666666664    & 453.781   \\
		Partición 4-1 & 71.11111111111111   & 44.44444444444444     & 542.437   \\
		Partición 4-2 & 70.0                & 54.44444444444444     & 576.512   \\
		Partición 5-1 & 70.0                & 48.888888888888886    & 622.136   \\
		Partición 5-2 & 74.44444444444444   & 50.0                  & 520.55    \\ \hline
		Media         & 70.61111111111111   & 49.11111111111111     & 509.2745
	\end{tabular}
	\caption{Movement Libras - PMA}
	\label{MLIB-PMA}
\end{table}
			\begin{table}[H]
	\centering
	\begin{tabular}{l|lll}
		Nombre        & Tasa de acierto(\%) & Tasa de reducción(\%) & Tiempo(s) \\ \hline
		Partición 1-1 & 66.32124352331606   & 47.12230215827338     & 1790.677  \\
		Partición 1-2 & 62.69430051813472   & 48.56115107913669     & 1768.997  \\
		Partición 2-1 & 60.62176165803109   & 47.48201438848921     & 1755.608  \\
		Partición 2-2 & 62.69430051813472   & 47.12230215827338     & 1687.165  \\
		Partición 3-1 & 69.43005181347151   & 48.92086330935252     & 1772.775  \\
		Partición 3-2 & 62.69430051813472   & 52.15827338129496     & 1785.188  \\
		Partición 4-1 & 59.58549222797927   & 45.32374100719424     & 1728.802  \\
		Partición 4-2 & 68.9119170984456    & 50.35971223021583     & 1913.634  \\
		Partición 5-1 & 65.28497409326425   & 52.15827338129496     & 1709.091  \\
		Partición 5-2 & 59.58549222797927   & 52.51798561151079     & 1892.485  \\ \hline
		Media         & 63.78238341968913   & 49.1726618705036      & 1780.4422
	\end{tabular}
	\caption{Arrhythmia - PMA}
	\label{ARRH-PMA}
\end{table}
		\subsection{Algotitmo Memético Parcial Elitista (\textbf{EPMA})}
			\begin{table}[H]
	\centering
	\begin{tabular}{l|lll}
		Nombre        & Tasa de acierto(\%) & Tasa de reducción(\%) & Tiempo(s) \\ \hline
		Partición 1-1 & 95.77464788732394   & 50.0                  & 385.797   \\
		Partición 1-2 & 95.43859649122807   & 23.333333333333332    & 361.738   \\
		Partición 2-1 & 95.07042253521126   & 46.666666666666664    & 322.149   \\
		Partición 2-2 & 94.73684210526316   & 60.0                  & 351.292   \\
		Partición 3-1 & 94.36619718309859   & 40.0                  & 398.329   \\
		Partición 3-2 & 96.49122807017544   & 60.0                  & 352.329   \\
		Partición 4-1 & 96.83098591549296   & 46.666666666666664    & 385.95    \\
		Partición 4-2 & 94.73684210526316   & 43.333333333333336    & 401.164   \\
		Partición 5-1 & 95.77464788732394   & 26.666666666666668    & 465.509   \\
		Partición 5-2 & 95.08771929824562   & 40.0                  & 387.92    \\ \hline
		Media         & 95.43081294786262   & 43.66666666666667     & 381.2177
	\end{tabular}
	\caption{WDBC - EPMA}
	\label{WDBC-EPMA}
\end{table}
			\begin{table}[H]
	\centering
	\begin{tabular}{l|lll}
		Nombre        & Tasa de acierto(\%) & Tasa de reducción(\%) & Tiempo(s) \\ \hline
		Partición 1-1 & 76.11111111111111   & 55.55555555555556     & 489.381   \\
		Partición 1-2 & 66.66666666666667   & 61.111111111111114    & 510.745   \\
		Partición 2-1 & 66.66666666666667   & 46.666666666666664    & 408.092   \\
		Partición 2-2 & 70.55555555555556   & 42.22222222222222     & 390.532   \\
		Partición 3-1 & 71.11111111111111   & 45.55555555555556     & 515.766   \\
		Partición 3-2 & 68.33333333333333   & 67.77777777777777     & 475.058   \\
		Partición 4-1 & 72.22222222222223   & 44.44444444444444     & 386.418   \\
		Partición 4-2 & 66.11111111111111   & 52.22222222222222     & 377.024   \\
		Partición 5-1 & 70.55555555555556   & 57.77777777777778     & 376.772   \\
		Partición 5-2 & 75.0                & 42.22222222222222     & 445.0     \\ \hline
		Media         & 70.33333333333333   & 51.55555555555556     & 437.4788
	\end{tabular}
	\caption{Movement Libras - EPMA}
	\label{MLIB-EPMA}
\end{table}
			\begin{table}[H]
	\centering
	\begin{tabular}{l|lll}
		Nombre        & Tasa de acierto(\%) & Tasa de reducción(\%) & Tiempo(s) \\ \hline
		Partición 1-1 & 65.28497409326425   & 48.56115107913669     & 2228.532  \\
		Partición 1-2 & 62.69430051813472   & 47.12230215827338     & 1893.633  \\
		Partición 2-1 & 59.58549222797927   & 55.03597122302158     & 1768.696  \\
		Partición 2-2 & 66.32124352331606   & 48.201438848920866    & 1797.324  \\
		Partición 3-1 & 61.6580310880829    & 50.35971223021583     & 1835.627  \\
		Partición 3-2 & 65.80310880829016   & 45.68345323741007     & 1891.617  \\
		Partición 4-1 & 61.13989637305699   & 43.52517985611511     & 1847.484  \\
		Partición 4-2 & 58.031088082901555  & 48.201438848920866    & 1861.277  \\
		Partición 5-1 & 61.13989637305699   & 51.798561151079134    & 1737.024  \\
		Partición 5-2 & 63.21243523316062   & 49.280575539568346    & 1761.401  \\ \hline
		Media         & 62.48704663212435   & 48.77697841726619     & 1862.2615
	\end{tabular}
	\caption{Arrhythmia - EPMA}
	\label{ARRH-EPMA}
\end{table}
		\subsection{Comparación de resultados}
			\begin{table}[H]
	\centering
	\small
	\begin{tabular}{l|lll|lll|lll}
				& 			& WDBC 		&			&	  Mov	& ement 	& Libras	&			& Arrhyt	& hmia		\\ \hline
				& \%\_clas	& \%\_red	& T			& \%\_clas	& \%\_red	& T			& \%\_clas	& \%\_red	& T			\\ \hline
		SFS		& 93.1814	& 88.6667	& 1.8121	& 49.4444	& 94.3333	& 8.0742	& 66.4249	& 97.8058	& 128.5797	\\ \hline
		TMA		& 94.4131	& 46.0		& 361.4899	& 69.6667	& 51.1111	& 494.7511	& 62.4870	& 48.7770	& 2026.6029	\\ \hline
		PMA		& 95.0089	& 44.3333	& 390.7531	& 70.6111	& 49.1111	& 509.2745	& 63.7824	& 49.1727	& 1780.4422	\\ \hline
		EPMA	& 95.4308	& 43.6667	& 381.2177	& 70.3333	& 51.5556	& 437.4788	& 62.4870	& 48.7770	& 1862.2615
		
	\end{tabular}
	\caption{Comparación de resultados}
	\label{Compare}
\end{table}
			
			Viendo los resultados de arriba, podemos concluir que:
			\begin{itemize}
				\item Los mayores índices de reducción se obtienen con el \textit{SFS}, obteniendo en
				todos los casos una tasa de reducción muy alta, mientras que los resultados de los
				\textit{MA} rondan el 50\%.
				
				\item En el caso de \textit{WBDC}, todos los algoritmos tienen una alta tasa de acierto,
				manteniéndose los resultados de los \textit{MA} en torno al 95\% de acierto, dando los
				mejores resultados en media el algoritmo \textit{EPMA} y los peores, \textit{TMA} con
				apenas 1\% de diferencia.
				
				\item En el caso de \textit{Movement Libras}, los algoritmos \textit{MA} siguen logrando
				resultados medios próximos al 70\% frente al 50\% de \textit{SFS}. Destacando como el
				mejor \textit{PMA}, seguido de \textit{EPMA} y de \textit{TMA}. A pesar de esto,
				ocasionalmente se obtenienen en ellos, un 75\% e incluso un 80\% de clasificación, que
				es el mayor porcentaje obtenido en dicho archivo incluyendo algoritmos de prácticas
				anteriores.
				
				\item Sin duda alguna, para el caso de \textit{Arrhythmia}, el mejor algoritmo sería
				el de \textit{SFS}, ya que tiene el mejor índice de acierto y de reducción.
				
				\item En cuanto a los tiempos de ejecución, podemos observar que el más veloz sería el
				algoritmo greedy \textit{SFS} aunque podemos observar que de entre los \textit{MA}(los
				cuáles suelen tener un resultado significativamente mejor que \textit{SFS}), podemos
				ver que \textit{TMA} es el más lento. Esto nos hace pensar que la búsqueda local es
				más pesada en términos de tiempo que las operaciones poblacionales, pues \textit{TMA}
				realiza más búsquedas locales que \textit{PMA} y \textit{EPMA}, los cuales realizan
				más generaciones.
				
				\item Un hecho que me parece curioso es el hecho de que sobre \textit{Arrhytmia}, tanto
				\textit{TMA} como \textit{EPMA} llegan a las mismas soluciones o, al menos, soluciones
				igual de buenas en todas las particiones realizadas.
			\end{itemize}
			
			La diferencia entre las tasas de reducción de los distintos algoritmos se deben a que \textit{SFS}
			comienza con una máscara en la que ninguna característica está seleccionada mientras que
			los algoritmos \textit{MA} toman una población de padres con unas máscaras en las cuáles
			la probabilidad de que una característica esté seleccionada es 50\%.
			
			Como los algoritmos \textit{MA} heredan gran parte de los métodos de los algoritmos \textit{GA},
			podemos razonar de forma análoga que ni el operador de selección ni el de reemplazo, alteran
			las máscaras con las que trabaja la población y que los operadores de cruce y de mutación
			alteran la máscara pero como no imponen condiciones sobre la tasa de reducción, no habrá
			un límite de dicho valor a una constante predeterminada. Este resultado se respalda con los
			resultados obtenidos, donde se han obtenido tasas de reducción comprendidas entre el 23\% y
			el 67\%.
			
			Referente a los tiempos de ejecución, estaba claro que \textit{SFS} sería el más veloz pues
			solamente aplica una vez la heurística mientras los \textit{MA} requieren agotar las 15000
			iteraciones para acabar aunque convergen rápidamente a buenas soluciones, por lo que si no
			las agotara e hicieran un número mucho menor de iteraciones, obtendrían una solución de
			similar grado de calidad.
			
	\input{Compilación.tex}
	\newpage
	
	\begin{thebibliography}{10}
	\expandafter\ifx\csname url\endcsname\relax
	  \def\url#1{\texttt{#1}}\fi
	\expandafter\ifx\csname urlprefix\endcsname\relax\def\urlprefix{URL }\fi
	\expandafter\ifx\csname href\endcsname\relax
	  \def\href#1#2{#2} \def\path#1{#1}\fi
	
	\bibitem{KNN}
	WEKA (The University of Waikato)\\
	  \url{http://weka.sourceforge.net/doc.dev/weka/classifiers/lazy/IBk.html}\\
	  \url{http://weka.sourceforge.net/doc.dev/weka/core/neighboursearch/NearestNeighbourSearch.html}
	  
  	\bibitem{ARFFReader}
  	WEKA (The University of Waikato)\\
	  \url{http://weka.sourceforge.net/doc.dev/weka/core/converters/ArffLoader.ArffReader.html}
	  
	\end{thebibliography}

\end{document}
