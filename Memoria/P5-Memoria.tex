%%
% Plantilla de Memoria
% Modificación de una plantilla de Latex de Nicolas Diaz para adaptarla 
% al castellano y a las necesidades de escribir informática y matemáticas.
%
% Editada por: Mario Román
%
% License:
% CC BY-NC-SA 3.0 (http://creativecommons.org/licenses/by-nc-sa/3.0/)
%%

%%%%%%%%%%%%%%%%%%%%%
% Thin Sectioned Essay
% LaTeX Template
% Version 1.0 (3/8/13)
%
% This template has been downloaded from:
% http://www.LaTeXTemplates.com
%
% Original Author:
% Nicolas Diaz (nsdiaz@uc.cl) with extensive modifications by:
% Vel (vel@latextemplates.com)
%
% License:
% CC BY-NC-SA 3.0 (http://creativecommons.org/licenses/by-nc-sa/3.0/)
%
%%%%%%%%%%%%%%%%%%%%%

%----------------------------------------------------------------------------------------
%	PAQUETES Y CONFIGURACIÓN DEL DOCUMENTO
%----------------------------------------------------------------------------------------

%% Configuración del papel.
% microtype: Tipografía.
% mathpazo: Usa la fuente Palatino.
\documentclass[a4paper, 11pt]{article}
\usepackage[protrusion=true,expansion=true]{microtype}
\usepackage{mathpazo}

% Indentación de párrafos para Palatino
\setlength{\parindent}{0pt}
  \parskip=8pt
\linespread{1.05} % Change line spacing here, Palatino benefits from a slight increase by default


%% Castellano.
% noquoting: Permite uso de comillas no españolas.
% lcroman: Permite la enumeración con numerales romanos en minúscula.
% fontenc: Usa la fuente completa para que pueda copiarse correctamente del pdf.
\usepackage[spanish,es-noquoting,es-lcroman]{babel}
\usepackage[utf8]{inputenc}
\usepackage[T1]{fontenc}
\selectlanguage{spanish}


%% Gráficos
\usepackage{graphicx} % Required for including pictures
\usepackage{wrapfig} % Allows in-line images
\usepackage[usenames,dvipsnames]{color} % Coloring code

% % Enlaces
\usepackage[hidelinks]{hyperref}

%% Matemáticas
\usepackage{amsmath}

% Para algoritmos
\usepackage{algorithm}
\usepackage{algorithmic}
\usepackage{amsthm}
\floatname{algorithm}{Algoritmo}
\renewcommand{\listalgorithmname}{Lista de algoritmos}
\renewcommand{\algorithmicrequire}{\textbf{Entrada:}}
\renewcommand{\algorithmicensure}{\textbf{Salida:}}
\renewcommand{\algorithmicend}{\textbf{fin}}
\renewcommand{\algorithmicif}{\textbf{si}}
\renewcommand{\algorithmicthen}{\textbf{entonces}}
\renewcommand{\algorithmicelse}{\textbf{en otro caso}}
\renewcommand{\algorithmicelsif}{\algorithmicelse,\ \algorithmicif}
\renewcommand{\algorithmicendif}{\algorithmicend\ \algorithmicif}
\renewcommand{\algorithmicfor}{\textbf{para }}
\renewcommand{\algorithmicforall}{\textbf{para cada}}
\renewcommand{\algorithmicdo}{\textbf{}}
\renewcommand{\algorithmicendfor}{\algorithmicend\ \algorithmicfor}
\renewcommand{\algorithmicwhile}{\textbf{mientras}}
\renewcommand{\algorithmicendwhile}{\algorithmicend\ \algorithmicwhile}
\renewcommand{\algorithmicloop}{\textbf{repetir}}
\renewcommand{\algorithmicendloop}{\algorithmicend\ \algorithmicloop}
\renewcommand{\algorithmicrepeat}{\textbf{repetir}}
\renewcommand{\algorithmicuntil}{\textbf{hasta que}}
\renewcommand{\algorithmicprint}{\textbf{imprimir}} 
\renewcommand{\algorithmicreturn}{\textbf{devolver}} 
\renewcommand{\algorithmictrue}{\textbf{true }} 
\renewcommand{\algorithmicfalse}{\textbf{false }} 
\renewcommand{\algorithmicand}{\textbf{y}}
\renewcommand{\algorithmicor}{\textbf{o}}


%% Bibliografía
\makeatletter
\renewcommand\@biblabel[1]{\textbf{#1.}} % Change the square brackets for each bibliography item from '[1]' to '1.'
\renewcommand{\@listI}{\itemsep=0pt} % Reduce the space between items in the itemize and enumerate environments and the bibliography


%----------------------------------------------------------------------------------------
%	TÍTULO
%----------------------------------------------------------------------------------------
% Configuraciones para el título.
% El título no debe editarse aquí.
\renewcommand{\maketitle}{
  \begin{flushright} % Center align
  {\LARGE\@title} % Increase the font size of the title
  
  \vspace{50pt} % Some vertical space between the title and author name
  
  {\large\@author} % Author name
  \\\@date % Date
  \vspace{40pt} % Some vertical space between the author block and abstract
  \end{flushright}
}

% Título
\title{\textbf{Metaheurísticas: Selección de Características}\\ % Title
P-5: Algoritmos Meméticos} % Subtitle

\author{\textsc{Óscar Bermúdez Garrido\\
\href{http://www.github.com/oxcar103}{@oxcar103}} % Author
\\{\textit{Universidad de Granada}}} % Institution

\date{\today} % Date


%----------------------------------------------------------------------------------------
%	DOCUMENTO
%----------------------------------------------------------------------------------------

\begin{document}

\maketitle % Print the title section

% Resumen (Descomentar para usarlo)
\renewcommand{\abstractname}{Resumen} % Uncomment to change the name of the abstract to something else
%\begin{abstract}
% Resumen aquí
%\end{abstract}

% Palabras clave
%\hspace*{3,6mm}\textit{Keywords:} lorem , ipsum , dolor , sit amet , lectus % Keywords
%\vspace{30pt} % Some vertical space between the abstract and first section


% Índice
{\parskip=2pt
  \tableofcontents
}
\pagebreak

%% Inicio del documento
	\input{Introducción.tex}
	\input{CaracterísticasComunes.tex}
	
	\section{Heurísticas implementadas}\footnote{En la implementación de las heurísticas desarrolladas
	en esta práctica, se utilizan la búsqueda de soluciones por la heurística \textbf{SFS} para comparar
	los resultados obtenidos y las optimizaciones de la \textbf{LS} para mejorar los resultados obtenidos
	por estas nuevas heurísticas multiarranque. Dado que ya se explicaron en \textbf{P1-Memoria}, he
	considerado ahorrar su explicación en este documento para permitir una mayor limpieza del mismo.}
	
		\subsection{Algoritmos Meméticos}
		\subsection{Algoritmo Memético Total}
		\subsection{Algoritmo Memético Parcial}
		\subsection{Algotitmo Memético Parcial Elitista}
	
	\section{Resultados}\footnote{En este documento, se comparan los resultados obtenidos por las
	heurísticas desarrolladas en esta práctica con la búsqueda de soluciones por la heurística
	\textbf{SFS} cuyos resultados ya se expusieron en \textbf{P1-Memoria}, por lo que también he
	obviado su inclusión en este documento en toda su extensión mas sí han sido incluidas las medias
	de ejecución en la tabla comparativa.}
	
		\subsection{Algoritmos Meméticos}
		\subsection{Algoritmo Memético Total}
		\subsection{Algoritmo Memético Parcial}
		\subsection{Algotitmo Memético Parcial Elitista}
		
	
	\input{Compilación.tex}
	\newpage
	
	\begin{thebibliography}{10}
	\expandafter\ifx\csname url\endcsname\relax
	  \def\url#1{\texttt{#1}}\fi
	\expandafter\ifx\csname urlprefix\endcsname\relax\def\urlprefix{URL }\fi
	\expandafter\ifx\csname href\endcsname\relax
	  \def\href#1#2{#2} \def\path#1{#1}\fi
	
	\bibitem{KNN}
	WEKA (The University of Waikato)\\
	  \url{http://weka.sourceforge.net/doc.dev/weka/classifiers/lazy/IBk.html}\\
	  \url{http://weka.sourceforge.net/doc.dev/weka/core/neighboursearch/NearestNeighbourSearch.html}
	  
  	\bibitem{ARFFReader}
  	WEKA (The University of Waikato)\\
	  \url{http://weka.sourceforge.net/doc.dev/weka/core/converters/ArffLoader.ArffReader.html}
	  
	\end{thebibliography}

\end{document}
