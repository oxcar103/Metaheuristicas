%%
% Plantilla de Memoria
% Modificación de una plantilla de Latex de Nicolas Diaz para adaptarla 
% al castellano y a las necesidades de escribir informática y matemáticas.
%
% Editada por: Mario Román
%
% License:
% CC BY-NC-SA 3.0 (http://creativecommons.org/licenses/by-nc-sa/3.0/)
%%

%%%%%%%%%%%%%%%%%%%%%
% Thin Sectioned Essay
% LaTeX Template
% Version 1.0 (3/8/13)
%
% This template has been downloaded from:
% http://www.LaTeXTemplates.com
%
% Original Author:
% Nicolas Diaz (nsdiaz@uc.cl) with extensive modifications by:
% Vel (vel@latextemplates.com)
%
% License:
% CC BY-NC-SA 3.0 (http://creativecommons.org/licenses/by-nc-sa/3.0/)
%
%%%%%%%%%%%%%%%%%%%%%

%----------------------------------------------------------------------------------------
%	PAQUETES Y CONFIGURACIÓN DEL DOCUMENTO
%----------------------------------------------------------------------------------------

%% Configuración del papel.
% microtype: Tipografía.
% mathpazo: Usa la fuente Palatino.
\documentclass[a4paper, 11pt]{article}
\usepackage[protrusion=true,expansion=true]{microtype}
\usepackage{mathpazo}

% Indentación de párrafos para Palatino
\setlength{\parindent}{0pt}
  \parskip=8pt
\linespread{1.05} % Change line spacing here, Palatino benefits from a slight increase by default


%% Castellano.
% noquoting: Permite uso de comillas no españolas.
% lcroman: Permite la enumeración con numerales romanos en minúscula.
% fontenc: Usa la fuente completa para que pueda copiarse correctamente del pdf.
\usepackage[spanish,es-noquoting,es-lcroman]{babel}
\usepackage[utf8]{inputenc}
\usepackage[T1]{fontenc}
\selectlanguage{spanish}


%% Gráficos
\usepackage{graphicx} % Required for including pictures
\usepackage{wrapfig} % Allows in-line images
\usepackage[usenames,dvipsnames]{color} % Coloring code

% % Enlaces
\usepackage[hidelinks]{hyperref}

%% Matemáticas
\usepackage{amsmath}

% Para algoritmos
\usepackage{algorithm}
\usepackage{algorithmic}
\usepackage{amsthm}
\floatname{algorithm}{Algoritmo}
\renewcommand{\listalgorithmname}{Lista de algoritmos}
\renewcommand{\algorithmicrequire}{\textbf{Entrada:}}
\renewcommand{\algorithmicensure}{\textbf{Salida:}}
\renewcommand{\algorithmicend}{\textbf{fin}}
\renewcommand{\algorithmicif}{\textbf{si}}
\renewcommand{\algorithmicthen}{\textbf{entonces}}
\renewcommand{\algorithmicelse}{\textbf{en otro caso}}
\renewcommand{\algorithmicelsif}{\algorithmicelse,\ \algorithmicif}
\renewcommand{\algorithmicendif}{\algorithmicend\ \algorithmicif}
\renewcommand{\algorithmicfor}{\textbf{para }}
\renewcommand{\algorithmicforall}{\textbf{para cada}}
\renewcommand{\algorithmicdo}{\textbf{}}
\renewcommand{\algorithmicendfor}{\algorithmicend\ \algorithmicfor}
\renewcommand{\algorithmicwhile}{\textbf{mientras}}
\renewcommand{\algorithmicendwhile}{\algorithmicend\ \algorithmicwhile}
\renewcommand{\algorithmicloop}{\textbf{repetir}}
\renewcommand{\algorithmicendloop}{\algorithmicend\ \algorithmicloop}
\renewcommand{\algorithmicrepeat}{\textbf{repetir}}
\renewcommand{\algorithmicuntil}{\textbf{hasta que}}
\renewcommand{\algorithmicprint}{\textbf{imprimir}} 
\renewcommand{\algorithmicreturn}{\textbf{devolver}} 
\renewcommand{\algorithmictrue}{\textbf{true }} 
\renewcommand{\algorithmicfalse}{\textbf{false }} 
\renewcommand{\algorithmicand}{\textbf{y}}
\renewcommand{\algorithmicor}{\textbf{o}}


%% Bibliografía
\makeatletter
\renewcommand\@biblabel[1]{\textbf{#1.}} % Change the square brackets for each bibliography item from '[1]' to '1.'
\renewcommand{\@listI}{\itemsep=0pt} % Reduce the space between items in the itemize and enumerate environments and the bibliography


%----------------------------------------------------------------------------------------
%	TÍTULO
%----------------------------------------------------------------------------------------
% Configuraciones para el título.
% El título no debe editarse aquí.
\renewcommand{\maketitle}{
  \begin{flushright} % Center align
  {\LARGE\@title} % Increase the font size of the title
  
  \vspace{50pt} % Some vertical space between the title and author name
  
  {\large\@author} % Author name
  \\\@date % Date
  \vspace{40pt} % Some vertical space between the author block and abstract
  \end{flushright}
}

% Título
\title{\textbf{Metaheurísticas: Selección de Características}\\ % Title
P-5: Algoritmos Meméticos} % Subtitle

\author{\textsc{Óscar Bermúdez Garrido\\
\href{http://www.github.com/oxcar103}{@oxcar103}} % Author
\\{\textit{Universidad de Granada}}} % Institution

\date{\today} % Date


%----------------------------------------------------------------------------------------
%	DOCUMENTO
%----------------------------------------------------------------------------------------

\begin{document}

\maketitle % Print the title section

% Resumen (Descomentar para usarlo)
\renewcommand{\abstractname}{Resumen} % Uncomment to change the name of the abstract to something else
%\begin{abstract}
% Resumen aquí
%\end{abstract}

% Palabras clave
%\hspace*{3,6mm}\textit{Keywords:} lorem , ipsum , dolor , sit amet , lectus % Keywords
%\vspace{30pt} % Some vertical space between the abstract and first section


% Índice
{\parskip=2pt
  \tableofcontents
}
\pagebreak

%% Inicio del documento
	\input{Introducción.tex}
	\input{CaracterísticasComunes.tex}
	
	\section{Heurísticas implementadas}\footnote{En la implementación de las heurísticas desarrolladas
	en esta práctica, se utilizan la búsqueda de soluciones por la heurística \textbf{SFS} para comparar
	los resultados obtenidos y las optimizaciones de la \textbf{LS} para mejorar los resultados obtenidos
	por estas nuevas heurísticas multiarranque. Dado que ya se explicaron en \textbf{P1-Memoria}, he
	considerado ahorrar su explicación en este documento para permitir una mayor limpieza del mismo.}
	
		\subsection{Algoritmos Meméticos}
		\subsection{Algoritmo Memético Total}
		\subsection{Algoritmo Memético Parcial}
		\subsection{Algotitmo Memético Parcial Elitista}
	
	\section{Resultados}\footnote{En este documento, se comparan los resultados obtenidos por las
	heurísticas desarrolladas en esta práctica con la búsqueda de soluciones por la heurística
	\textbf{SFS} cuyos resultados ya se expusieron en \textbf{P1-Memoria}, por lo que también he
	obviado su inclusión en este documento en toda su extensión mas sí han sido incluidas las medias
	de ejecución en la tabla comparativa.}
	
		\subsection{Algoritmo Memético Total}
			\begin{table}[H]
	\centering
	\begin{tabular}{l|lll}
		Nombre        & Tasa de acierto(\%) & Tasa de reducción(\%) & Tiempo(s)          \\ \hline
		Partición 1-1 & 95.77464788732394   & 46.666666666666664    & 393.624            \\
		Partición 1-2 & 93.6842105263158    & 40.0                  & 341.458            \\
		Partición 2-1 & 95.4225352112676    & 46.666666666666664    & 367.4              \\
		Partición 2-2 & 93.33333333333333   & 56.666666666666664    & 353.936            \\
		Partición 3-1 & 93.30985915492958   & 46.666666666666664    & 353.209            \\
		Partición 3-2 & 94.03508771929825   & 40.0                  & 348.396            \\
		Partición 4-1 & 97.53521126760563   & 40.0                  & 399.221            \\
		Partición 4-2 & 94.73684210526316   & 50.0                  & 370.773            \\
		Partición 5-1 & 95.4225352112676    & 43.333333333333336    & 313.611            \\
		Partición 5-2 & 90.87719298245614   & 50.0                  & 373.271            \\ \hline
		Media         & 94.41314553990608   & 45.99999999999999     & 361.48990000000003
	\end{tabular}
	\caption{WDBC - TMA}
	\label{WDBC-TMA}
\end{table}
			\begin{table}[H]
	\centering
	\begin{tabular}{l|lll}
		Nombre        & Tasa de acierto(\%) & Tasa de reducción(\%) & Tiempo(s)          \\ \hline
		Partición 1-1 & 72.22222222222223   & 47.77777777777778     & 483.078           \\
		Partición 1-2 & 68.33333333333333   & 51.111111111111114    & 413.394           \\
		Partición 2-1 & 69.44444444444444   & 51.111111111111114    & 425.151           \\
		Partición 2-2 & 71.11111111111111   & 55.55555555555556     & 407.325           \\
		Partición 3-1 & 75.0                & 44.44444444444444     & 365.033           \\
		Partición 3-2 & 66.66666666666667   & 52.22222222222222     & 602.561           \\
		Partición 4-1 & 66.66666666666667   & 43.333333333333336    & 568.858           \\
		Partición 4-2 & 72.77777777777777   & 54.44444444444444     & 531.985           \\
		Partición 5-1 & 65.0                & 55.55555555555556     & 590.151           \\
		Partición 5-2 & 69.44444444444444   & 55.55555555555556     & 559.975           \\ \hline
		Media         & 69.66666666666666   & 51.11111111111111     & 494.7511000000001
	\end{tabular}
	\caption{Movement Libras - TMA}
	\label{MLIB-TMA}
\end{table}
			\begin{table}[H]
	\centering
	\begin{tabular}{l|lll}
		Nombre        & Tasa de acierto(\%) & Tasa de reducción(\%) & Tiempo(s)          \\ \hline
		Partición 1-1 & 65.28497409326425   & 48.56115107913669     & 2804.111           \\
		Partición 1-2 & 62.69430051813472   & 47.12230215827338     & 2239.5             \\
		Partición 2-1 & 59.58549222797927   & 55.03597122302158     & 2203.75            \\
		Partición 2-2 & 66.32124352331606   & 48.201438848920866    & 1997.665           \\
		Partición 3-1 & 61.6580310880829    & 50.35971223021583     & 1916.014           \\
		Partición 3-2 & 65.80310880829016   & 45.68345323741007     & 1894.98            \\
		Partición 4-1 & 61.13989637305699   & 43.52517985611511     & 1854.196           \\
		Partición 4-2 & 58.031088082901555  & 48.201438848920866    & 1854.773           \\
		Partición 5-1 & 61.13989637305699   & 51.798561151079134    & 1746.643           \\
		Partición 5-2 & 63.21243523316062   & 49.280575539568346    & 1754.397           \\ \hline
		Media         & 62.48704663212435   & 48.77697841726619     & 2026.6028999999999
	\end{tabular}
	\caption{Arrhythmia - TMA}
	\label{ARRH-TMA}
\end{table}
		\subsection{Algoritmo Memético Parcial}
			\begin{table}[H]
	\centering
	\begin{tabular}{l|lll}
		Nombre        & Tasa de acierto(\%) & Tasa de reducción(\%) & Tiempo(s)          \\ \hline
		Partición 1-1 & 94.71830985915493   & 50.0                  & 408.058            \\
		Partición 1-2 & 95.78947368421052   & 60.0                  & 342.7              \\
		Partición 2-1 & 95.4225352112676    & 40.0                  & 381.02             \\
		Partición 2-2 & 94.3859649122807    & 36.666666666666664    & 394.217            \\
		Partición 3-1 & 93.66197183098592   & 36.666666666666664    & 448.106            \\
		Partición 3-2 & 95.78947368421052   & 53.333333333333336    & 442.366            \\
		Partición 4-1 & 96.12676056338029   & 50.0                  & 375.013            \\
		Partición 4-2 & 95.08771929824562   & 33.333333333333336    & 372.708            \\
		Partición 5-1 & 95.4225352112676    & 50.0                  & 322.973            \\
		Partición 5-2 & 93.6842105263158    & 33.333333333333336    & 420.37             \\ \hline
		Media         & 95.00889547813196   & 44.33333333333333     & 390.75309999999996
	\end{tabular}
	\caption{WDBC - PMA}
	\label{WDBC-PMA}
\end{table}
			\begin{table}[H]
	\centering
	\begin{tabular}{l|lll}
		Nombre        & Tasa de acierto(\%) & Tasa de reducción(\%) & Tiempo(s) \\ \hline
		Partición 1-1 & 70.0                & 51.111111111111114    & 463.453   \\
		Partición 1-2 & 72.77777777777777   & 51.111111111111114    & 441.972   \\
		Partición 2-1 & 67.77777777777777   & 46.666666666666664    & 552.651   \\
		Partición 2-2 & 66.11111111111111   & 50.0                  & 500.743   \\
		Partición 3-1 & 63.333333333333336  & 47.77777777777778     & 418.51    \\
		Partición 3-2 & 80.55555555555556   & 46.666666666666664    & 453.781   \\
		Partición 4-1 & 71.11111111111111   & 44.44444444444444     & 542.437   \\
		Partición 4-2 & 70.0                & 54.44444444444444     & 576.512   \\
		Partición 5-1 & 70.0                & 48.888888888888886    & 622.136   \\
		Partición 5-2 & 74.44444444444444   & 50.0                  & 520.55    \\ \hline
		Media         & 70.61111111111111   & 49.11111111111111     & 509.2745
	\end{tabular}
	\caption{Movement Libras - PMA}
	\label{MLIB-PMA}
\end{table}
			\begin{table}[H]
	\centering
	\begin{tabular}{l|lll}
		Nombre        & Tasa de acierto(\%) & Tasa de reducción(\%) & Tiempo(s) \\ \hline
		Partición 1-1 & 66.32124352331606   & 47.12230215827338     & 1790.677  \\
		Partición 1-2 & 62.69430051813472   & 48.56115107913669     & 1768.997  \\
		Partición 2-1 & 60.62176165803109   & 47.48201438848921     & 1755.608  \\
		Partición 2-2 & 62.69430051813472   & 47.12230215827338     & 1687.165  \\
		Partición 3-1 & 69.43005181347151   & 48.92086330935252     & 1772.775  \\
		Partición 3-2 & 62.69430051813472   & 52.15827338129496     & 1785.188  \\
		Partición 4-1 & 59.58549222797927   & 45.32374100719424     & 1728.802  \\
		Partición 4-2 & 68.9119170984456    & 50.35971223021583     & 1913.634  \\
		Partición 5-1 & 65.28497409326425   & 52.15827338129496     & 1709.091  \\
		Partición 5-2 & 59.58549222797927   & 52.51798561151079     & 1892.485  \\ \hline
		Media         & 63.78238341968913   & 49.1726618705036      & 1780.4422
	\end{tabular}
	\caption{Arrhythmia - PMA}
	\label{ARRH-PMA}
\end{table}
		\subsection{Algotitmo Memético Parcial Elitista}
		\subsection{Comparación de resultados}
			\begin{table}[H]
	\centering
	\small
	\begin{tabular}{l|lll|lll|lll}
				& 			& WDBC 		&			&	  Mov	& ement 	& Libras	&			& Arrhyt	& hmia		\\ \hline
				& \%\_clas	& \%\_red	& T			& \%\_clas	& \%\_red	& T			& \%\_clas	& \%\_red	& T			\\ \hline
		SFS		& 93.1814	& 88.6667	& 1.8121	& 49.4444	& 94.3333	& 8.0742	& 66.4249	& 97.8058	& 128.5797	\\ \hline
		TMA		& 94.4131	& 46.0		& 361.4899	& 69.6667	& 51.1111	& 494.7511	& 62.4870	& 48.7770	& 2026.6029	\\ \hline
		PMA		& 95.0089	& 44.3333	& 390.7531	& 70.6111	& 49.1111	& 509.2745	& 63.7824	& 49.1727	& 1780.4422	\\ \hline
		EPMA	& 95.4308	& 43.6667	& 381.2177	& 70.3333	& 51.5556	& 437.4788	& 62.4870	& 48.7770	& 1862.2615
		
	\end{tabular}
	\caption{Comparación de resultados}
	\label{Compare}
\end{table}
		
	
	\input{Compilación.tex}
	\newpage
	
	\begin{thebibliography}{10}
	\expandafter\ifx\csname url\endcsname\relax
	  \def\url#1{\texttt{#1}}\fi
	\expandafter\ifx\csname urlprefix\endcsname\relax\def\urlprefix{URL }\fi
	\expandafter\ifx\csname href\endcsname\relax
	  \def\href#1#2{#2} \def\path#1{#1}\fi
	
	\bibitem{KNN}
	WEKA (The University of Waikato)\\
	  \url{http://weka.sourceforge.net/doc.dev/weka/classifiers/lazy/IBk.html}\\
	  \url{http://weka.sourceforge.net/doc.dev/weka/core/neighboursearch/NearestNeighbourSearch.html}
	  
  	\bibitem{ARFFReader}
  	WEKA (The University of Waikato)\\
	  \url{http://weka.sourceforge.net/doc.dev/weka/core/converters/ArffLoader.ArffReader.html}
	  
	\end{thebibliography}

\end{document}
